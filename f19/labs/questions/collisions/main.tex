\section*{Finding collisions in hash functions}

\bgroup
\begin{wrapfigure}{r}{.38\textwidth}
\vspace{-.2in}
{\footnotesize
~\quad$\begin{array}{|lp{10pt}l|lp{10pt}l|lp{10pt}l|}\hline
\multicolumn{9}{|c|}{\textit{Partial ASCII Table}}\\\hline
     32 & 20 & \textrm{\textvisiblespace} &  64 & 40 & \texttt{@} &  96 & 60 & \texttt{`} \\
     33 & 21 & \texttt{!}      &  65 & 41 & \texttt{A}      &  97 & 61 & \texttt{a} \\
     34 & 22 & \texttt{"}      &  66 & 42 & \texttt{B}      &  98 & 62 & \texttt{b} \\
     35 & 23 & \texttt{\#}     &  67 & 43 & \texttt{C}      &  99 & 63 & \texttt{c} \\
     36 & 24 & \texttt{\$}     &  68 & 44 & \texttt{D}      & 100 & 64 & \texttt{d} \\
     37 & 25 & \texttt{\%}     &  69 & 45 & \texttt{E}      & 101 & 65 & \texttt{e} \\
     38 & 26 & \texttt{\&}     &  70 & 46 & \texttt{F}      & 102 & 66 & \texttt{f} \\
     39 & 27 & \texttt{'}      &  71 & 47 & \texttt{G}      & 103 & 67 & \texttt{g} \\
     40 & 28 & \texttt{(}      &  72 & 48 & \texttt{H}      & 104 & 68 & \texttt{h} \\
     41 & 29 & \texttt{)}      &  73 & 49 & \texttt{I}      & 105 & 69 & \texttt{i} \\
     42 & 2A & \texttt{*}      &  74 & 4A & \texttt{J}      & 106 & 6A & \texttt{j} \\
     43 & 2B & \texttt{+}      &  75 & 4B & \texttt{K}      & 107 & 6B & \texttt{k} \\
     44 & 2C & \texttt{,}      &  76 & 4C & \texttt{L}      & 108 & 6C & \texttt{l} \\
     45 & 2D & \texttt{-}      &  77 & 4D & \texttt{M}      & 109 & 6D & \texttt{m} \\
     46 & 2E & \texttt{.}      &  78 & 4E & \texttt{N}      & 110 & 6E & \texttt{n} \\
     47 & 2F & \texttt{/}      &  79 & 4F & \texttt{O}      & 111 & 6F & \texttt{o} \\
     48 & 30 & \texttt{0}      &  80 & 50 & \texttt{P}      & 112 & 70 & \texttt{p} \\
     49 & 31 & \texttt{1}      &  81 & 51 & \texttt{Q}      & 113 & 71 & \texttt{q} \\
     50 & 32 & \texttt{2}      &  82 & 52 & \texttt{R}      & 114 & 72 & \texttt{r} \\
     51 & 33 & \texttt{3}      &  83 & 53 & \texttt{S}      & 115 & 73 & \texttt{s} \\
     52 & 34 & \texttt{4}      &  84 & 54 & \texttt{T}      & 116 & 74 & \texttt{t} \\
     53 & 35 & \texttt{5}      &  85 & 55 & \texttt{U}      & 117 & 75 & \texttt{u} \\
     54 & 36 & \texttt{6}      &  86 & 56 & \texttt{V}      & 118 & 76 & \texttt{v} \\
     55 & 37 & \texttt{7}      &  87 & 57 & \texttt{W}      & 119 & 77 & \texttt{w} \\
     56 & 38 & \texttt{8}      &  88 & 58 & \texttt{X}      & 120 & 78 & \texttt{x} \\
     57 & 39 & \texttt{9}      &  89 & 59 & \texttt{Y}      & 121 & 79 & \texttt{y} \\
     58 & 3A & \texttt{:}      &  90 & 5A & \texttt{Z}      & 122 & 7A & \texttt{z} \\
     59 & 3B & \texttt{;}      &  91 & 5B & \texttt{[}      & 123 & 7B & \texttt{\{} \\
     60 & 3C & \texttt{<}      &  92 & 5C & \texttt{\textbackslash}       & 124 & 7C & \texttt{|} \\
     61 & 3D & \texttt{=}      &  93 & 5D & \texttt{]}      & 125 & 7D & \texttt{\}} \\
     62 & 3E & \texttt{>}      &  94 & 5E & \texttt{\textasciicircum}       & 126 & 7E & \sim \\
     63 & 3F & \texttt{?}      &  95 & 5F & \texttt{\_}     &  &  &  \\
\hline
\end{array}$ }
\end{wrapfigure}


Recall that a hash function $h(k)$ takes a key $k$ as its argument and
returns some integer, a \emph{hash value}; we can then use
$\textrm{abs}(h(k) \% m)$ as an index into our hash table.  In this
lab you will be examining various hash functions and exploiting their
inefficiencies to make them collide.

It will be convenient to will denote a string of length $n$ (for $n>0$) as
$s_0s_1s_2...s_{n-2}s_{n-1}$, where $s_i$ is the ASCII value of
character $i$ in string $s$.  (A partial ASCII table is given to the
right.)  We define four hash functions as follows:

\begin{description}
\labelwidth=1em
\item\lstinline'hash_add':
$h(s) = s_0 + s_1 + s_2 + \dots + s_{n-2} + s_{n-1}$

\item\lstinline'hash_mul32':

\hspace*{-1.9em}%
$h(s) = ( \dots ((s_0 \times 32 + s_1) \times 32 + s_2) \times 32 \dots + s_{n-2}) \times 32 + s_{n-1}$

\item\lstinline'hash_mul31':

\hspace*{-1.9em}%
$h(s) = ( \dots ((s_0 \times 31 + s_1) \times 31 + s_2) \times 31 \dots + s_{n-2}) \times 31 + s_{n-1}$

\item\lstinline'hash_lcg':

$h(s) = f(f( \dots f(f(f(s_0) + s_1) + s_2) \dots + s_{n-2}) + s_{n-1})$

\hspace*{-1.9em}%
where $f(x) = 1664525 \times x + 1013904223$
\end{description}

\medskip
These four hash functions have been implemented for you and can be run
from the command line like this, for example:

\begin{lstlisting}[language={[coin]C}]
% hash_add
Enter a string to hash: bar
   hash value = 309
   hashes to index 309 in a table of size 1024
Another? (empty line quits):
\end{lstlisting}
Note that the command line hashing tool also reports where the element
with the given key will hash to given a table size of 1024. This is
important because hash tables have a limited size, so we want to
minimize collisions within said size.
\egroup

\newpage
The first exercise requires you to mathematically reverse-engineer one of the
simpler hash functions:

\begin{part}\TAGS{hashing, string}
  Find three or more strings, each string containing three or more
  characters, that would always collide because they have the same
  hash value using \lstinline'hash_add'.

\onePT
\end{part}

\begin{solution}
Example Solution: ADG, BDF, CDE\\
Strategy: add to one character while subtracting from the other.
\end{solution}

Now, let's work through a more complicated real-world example: hashing
an entire dictionary. We would like to know which hashing function
would be the best to hash the Scrabble dictionary. We define a hashing
function to be ``better'' based on how efficiently it spreads out the
words over the buckets. Obviously, this depends on the size of our
hash table: if we have a smaller hash table, there will naturally be
more collisions.  That's why we can use a visualizer (implemented for
you in file \lstinline'visualizer.c0') to see how many words hash to
each bucket for a given hash function.

\begin{part}\TAGS{hashing}
  Implement your own version of \lstinline'hash_mul32' in
  \lstinline'hash-a.c0' so that the function
  \lstinline'hash_string(s)' returns an integer representing the hash
  value for \lstinline's' using the formula given on the previous
  page. The
  \href{http://c0.typesafety.net/doc/c0-libraries.pdf}{string library}
  may be helpful in this.  You can compile your code and run it with
  the following command:
\begin{lstlisting}[language={[coin]C}, basicstyle=\smallbasicstyle,
                   belowskip=0pt]
  % cc0 hash-a.c0 hash-dictionary.c0 visualizer.c0
  % ./a.out -o mul32.png
  % display mul32.png
\end{lstlisting}
This will output a graphical visualization of your hash function on
the dictionary for a table of size 1024, with the vertical lines
showing how many values hashed to that index in the table.  If you are
ssh'ing remember to ssh with \lstinline'-Y' or \lstinline'-X'! You can
run your program with the \lstinline'-n' flag followed by a different
table size if you like.  You can see just how ineffective
\lstinline'hash_mul32' is!
\end{part}

\begin{solution}
\begin{lstlisting}[language={C}]
int hash(string s) {
    int len = string_length(s);
    if (len == 0) {
        return 0;
    }
    char c = string_charat(s, len-1);
    return char_ord(c) + 32*hash(string_sub(s, 0, len-1));
}
\end{lstlisting}
\end{solution}

\begin{part}\TAGS{hashing}
  Now, similarly implement \lstinline'hash_lcg' in
  \lstinline'hash-b.c0', and compile it for the dictionary:
\begin{lstlisting}[language={[coin]C}, basicstyle=\smallbasicstyle,
                   belowskip=0pt]
  % cc0 hash-b.c0 hash-dictionary.c0 visualizer.c0
\end{lstlisting}
Run it like above to see how well it hashes the dictionary. Compare
this to \lstinline'hash_mul32'.

\twoPT
\end{part}

\begin{solution}
\begin{lstlisting}[language={C}]
int hash(string s) {
    int len = string_length(s);
    if (len == 0) {
        return 0;
    }
    char c = string_charat(s, len-1);
    return 166425*(char_ord(c) + hash(string_sub(s, 0, len-1))) + 1013904223;
}
\end{lstlisting}
\end{solution}
