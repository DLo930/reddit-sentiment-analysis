\section*{Dr.~Evil's passwords}

Genius supervillian Dr.~Evil is on the loose! Known for a series of
devilishly tricky yet completely vulnerable assembly bombs, Dr.~Evil
has left a trail of destruction across Carnegie Mellon's undergraduate
computer science curriculum.  Authorities have been unable to track
the whereabouts of this mastermind, but we have new intelligence on
Dr.~Evil's Super Secret Evil Plan\textsuperscript{TM} to investigate.

You have been hired as an agent to crack the code of Dr.~Evil's Super
Secret Evil Plan. It seems that she left her secret plans in a
password protected c0 binary file, accessible to you on the cluster
computers by typing \lstinline'evilplan'. She also accidentally left her
C0VM bytecode in a public folder! She seems to have deleted most of
the helpful comments, though, so we'll need help figuring out the
passwords by hand. We were also able to acquire the \lstinline'main'
function's source code in \lstinline'password-main.c0', but it relies on
functions that only appear in the bytecode file \lstinline'password.bc0'.

You'll need to read through \lstinline'password.bc0' to figure out
some of the function calls --- namely, the function calls
\lstinline'password1()', \lstinline'password2()',
\lstinline'password3()', etc.

Each of the password functions either takes in a password as input,
and returns a boolean, or simply returns the password as an
integer. Some passwords are numbers while others are strings.  For the
first four passwords the user's input is passed to
\lstinline'parse_int', but for the last password the string is
passed directly to the function.  We've filled in the bytecode file
with all the intelligence we have, so you'll have to figure out the
rest.

\emph{We highly suggest you keep the C0VM writeup handy as you
  complete this lab.}

To check if you're correct, just run the password binary file, and
type in the passwords you think are correct:

\begin{lstlisting}[language={[coin]C}, stringstyle=\basicstyle]
% evilplan
Welcome to Dr. Evil's Super Secret Evil Plan Terminal
This terminal should only be run by Dr. Evil to read the
Super Secret Evil Plan.
If you are anyone else, get OUT.
Password1:
\end{lstlisting}

\clearpage
\begin{wrapfigure}{r}{.38\textwidth}
\vspace{-.2in}
{\footnotesize
~\quad$\begin{array}{|lp{10pt}l|lp{10pt}l|lp{10pt}l|}\hline
\multicolumn{9}{|c|}{\textit{Partial ASCII Table}}\\\hline
     32 & 20 & \textrm{\textvisiblespace} &  64 & 40 & \texttt{@} &  96 & 60 & \texttt{`} \\
     33 & 21 & \texttt{!}      &  65 & 41 & \texttt{A}      &  97 & 61 & \texttt{a} \\
     34 & 22 & \texttt{"}      &  66 & 42 & \texttt{B}      &  98 & 62 & \texttt{b} \\
     35 & 23 & \texttt{\#}     &  67 & 43 & \texttt{C}      &  99 & 63 & \texttt{c} \\
     36 & 24 & \texttt{\$}     &  68 & 44 & \texttt{D}      & 100 & 64 & \texttt{d} \\
     37 & 25 & \texttt{\%}     &  69 & 45 & \texttt{E}      & 101 & 65 & \texttt{e} \\
     38 & 26 & \texttt{\&}     &  70 & 46 & \texttt{F}      & 102 & 66 & \texttt{f} \\
     39 & 27 & \texttt{'}      &  71 & 47 & \texttt{G}      & 103 & 67 & \texttt{g} \\
     40 & 28 & \texttt{(}      &  72 & 48 & \texttt{H}      & 104 & 68 & \texttt{h} \\
     41 & 29 & \texttt{)}      &  73 & 49 & \texttt{I}      & 105 & 69 & \texttt{i} \\
     42 & 2A & \texttt{*}      &  74 & 4A & \texttt{J}      & 106 & 6A & \texttt{j} \\
     43 & 2B & \texttt{+}      &  75 & 4B & \texttt{K}      & 107 & 6B & \texttt{k} \\
     44 & 2C & \texttt{,}      &  76 & 4C & \texttt{L}      & 108 & 6C & \texttt{l} \\
     45 & 2D & \texttt{-}      &  77 & 4D & \texttt{M}      & 109 & 6D & \texttt{m} \\
     46 & 2E & \texttt{.}      &  78 & 4E & \texttt{N}      & 110 & 6E & \texttt{n} \\
     47 & 2F & \texttt{/}      &  79 & 4F & \texttt{O}      & 111 & 6F & \texttt{o} \\
     48 & 30 & \texttt{0}      &  80 & 50 & \texttt{P}      & 112 & 70 & \texttt{p} \\
     49 & 31 & \texttt{1}      &  81 & 51 & \texttt{Q}      & 113 & 71 & \texttt{q} \\
     50 & 32 & \texttt{2}      &  82 & 52 & \texttt{R}      & 114 & 72 & \texttt{r} \\
     51 & 33 & \texttt{3}      &  83 & 53 & \texttt{S}      & 115 & 73 & \texttt{s} \\
     52 & 34 & \texttt{4}      &  84 & 54 & \texttt{T}      & 116 & 74 & \texttt{t} \\
     53 & 35 & \texttt{5}      &  85 & 55 & \texttt{U}      & 117 & 75 & \texttt{u} \\
     54 & 36 & \texttt{6}      &  86 & 56 & \texttt{V}      & 118 & 76 & \texttt{v} \\
     55 & 37 & \texttt{7}      &  87 & 57 & \texttt{W}      & 119 & 77 & \texttt{w} \\
     56 & 38 & \texttt{8}      &  88 & 58 & \texttt{X}      & 120 & 78 & \texttt{x} \\
     57 & 39 & \texttt{9}      &  89 & 59 & \texttt{Y}      & 121 & 79 & \texttt{y} \\
     58 & 3A & \texttt{:}      &  90 & 5A & \texttt{Z}      & 122 & 7A & \texttt{z} \\
     59 & 3B & \texttt{;}      &  91 & 5B & \texttt{[}      & 123 & 7B & \texttt{\{} \\
     60 & 3C & \texttt{<}      &  92 & 5C & \texttt{\textbackslash}       & 124 & 7C & \texttt{|} \\
     61 & 3D & \texttt{=}      &  93 & 5D & \texttt{]}      & 125 & 7D & \texttt{\}} \\
     62 & 3E & \texttt{>}      &  94 & 5E & \texttt{\textasciicircum}       & 126 & 7E & \sim \\
     63 & 3F & \texttt{?}      &  95 & 5F & \texttt{\_}     &  &  &  \\
\hline
\end{array}$ }
\end{wrapfigure}

~

\vspace{-43pt}
\begin{part}\TAGS{c0vm}
  Dr.~Evil's first password function seems pretty simple. It seems to
  return an integer. What is it?

\onePT

\begin{solution}
  1031
\end{solution}
\end{part}

\begin{part}\TAGS{c0vm}
  Dr.~Evil's second password is a bit more complicated. It uses
  \lstinline'vload' and \lstinline'vstore' to store some local
  variables. Figure out what integer \lstinline'password2' returns!

\begin{solution}
  35
\end{solution}
\end{part}

\begin{part}\TAGS{c0vm}
  Dr.~Evil's third password is definitely more complicated. It uses
  \lstinline'ildc' to load integers from the integer pool. What's
  going on there? For this password, note that returning \lstinline'1'
  is equivalent to returning \lstinline'true', and returning
  \lstinline'0' is equivalent to returning \lstinline'false'.

\twoPT

\begin{solution}
  46 + 367k
\end{solution}
\end{part}

\begin{part}\TAGS{c0vm}
  Dr.~Evil's fourth password has a loop! The function jumps around,
  doing something to an integer input. What's the password?

\begin{solution}
  122
\end{solution}
\end{part}

\begin{part}\TAGS{c0vm}
  Dr.~Evil's fifth and final password calls a helper function,
  func5. Figure out what it's doing, and crack the last password!  The
  ASCII table to the right, which includes both integer and hex
  values, may come in handy.

\threePT

\begin{solution}
  ``the interface''
\end{solution}
\end{part}

\begin{part}\TAGS{c0vm}
  For the most clever of agents, Dr.~Evil seems to have left a hidden
  6th password. She didn't activate it in the source code file, which
  means it must have been so complicated even she didn't want to deal
  with it! Figure it out through the bytecode, and tell your TA if you
  think you got it.

\begin{solution}
  ``Respect'' followed by any permutation of 0 through 5, followed by
  anything (including the empty string)
\end{solution}
\end{part}
