\section*{Using generic hash tables}

In this lab, we'll be using the hash dictionaries discussed in
lecture, but we'll be implementing a slightly different
dictionary interface than what we saw in class.
\begin{quote}
\begin{lstlisting}
/*** Client interface ***/

typedef void* key;
typedef void* value;

typedef bool key_equiv_fn(key x, key y);
typedef int key_hash_fn(key x);

/*** Library interface ***/

// typedef ______* hdict_t;
typedef struct hdict_header* hdict_t;

hdict_t hdict_new(int capacity, key_equiv_fn* equiv, key_hash_fn* hash)
  /*@requires capacity > 0 && equiv != NULL && hash != NULL; @*/
  /*@ensures \result != NULL; @*/ ;

value hdict_lookup(hdict_t H, key k)
  /*@requires H != NULL; @*/ ;

void hdict_insert(hdict_t H, key k, value v)
  /*@requires H != NULL && v != NULL; @*/
  /*@ensures hdict_lookup(H, k) == v; @*/ ;
\end{lstlisting}
\end{quote}
Our sample application will be used in checking student attendance.
Your code for this should go in a file called \lstinline'rollcall.c1'. %NEW

\begin{part}\TAGS{dictionary, void-star}
  Define a struct that represents students.  Its fields should include
  \lstinline'andrew_id' (\lstinline'string'), \lstinline'days_present'
  (\lstinline'int'), and \lstinline'days_absent' (\lstinline'int').
  You can include other fields if you want, but you need these fields
  with these types.

  Write out the definition of this struct.  Include a
  \lstinline'typedef' so that you can allocate structs with
  \lstinline'alloc(student)'.

\onePT
\begin{solution}
\begin{lstlisting}[numbers=left, name="rollcall"]
struct student_info {
  string andrew_id;
  int days_present;
  int days_absent;
};
typedef struct student_info student;

\end{lstlisting}
\end{solution}
\end{part}

\begin{part}\TAGS{interface, hashing, void-star}
  Write client functions for a hashtable based on student
  information. For this lab we will think of our keys as being
  Andrew IDs, and therefore be using pointers to
  \lstinline'string's (\lstinline'string*') to represent them. We will
  think of the entries as being students, and therefore
  use pointers to \lstinline'student's (\lstinline'student*') to represent
  the value.

  \emph{Hint:} Your functions should have the requirement that
  \lstinline'x' and \lstinline'y' are both non-\lstinline'NULL' and
  have \lstinline'string*' as their tag.
 %The hash function should create a hash value based \emph{only} on the \lstinline'andrew_id' string, and the equivalence function should check \emph{only} the \lstinline'andrew_id' fields for equality.
\begin{quote}
\begin{lstlisting}
int hash_student(key x);
bool students_same_andrewid(key x, key y);
\end{lstlisting}
\end{quote}

\begin{solution}
\begin{lstlisting}[numbers=left, name="rollcall"]
int hash_student(key x)
//@requires x != NULL && \hastag(string*, x);
{
    string s = *(string*)x;
    int h = 0;
    for (int i = 0; i < string_length(s); i++) {
        h = 31 * h + char_ord(string_charat(s, i));
    }
    return h;
}

bool students_equal(key x, key y)
//@requires x != NULL && \hastag(string*, x);
//@requires y != NULL && \hastag(string*, y);
{
    string s1 = *(string*)x;
    string s2 = *(string*)y;
    return string_equal(s1, s2);
}
\end{lstlisting}
\end{solution}
\end{part}

\begin{part}\TAGS{function-pointer, interface}
  Write a function that initializes a \lstinline'hdict_t' with students
  that have no attendance record. Don't worry about what happens if
  there are duplicates in this array.
\begin{quote}
\begin{lstlisting}[belowskip=0pt]
hdict_t new_roster(string[] andrew_ids, int len)
//@requires \length(andrew_ids) == len;
\end{lstlisting}
\twoPT
\end{quote}

\begin{solution}
\begin{lstlisting}[numbers=left, name="rollcall"]
hdict_t new_roster(string[] andrew_ids, int len)
//@requires \length(andrew_ids) == len;
{
    hdict_t H = hdict_new(len, &students_equal, &hash_student);
    for (int i = 0; i < len; i++) {
        student* s = alloc(student);
        s->andrew_id = andrew_ids[i];
        string* v = alloc(string);
        *v = andrew_ids[i];
        hdict_insert(H, (void*)v, (void*)s);
    }
    return H;
}
\end{lstlisting}
\end{solution}
\end{part}

At this point, you should create a trivial \lstinline'main()' function
inside \lstinline'rollcall.c1'
just to make sure your code compiles:
\begin{lstlisting}[language={[coin]C}]
cc0 -d hdict.c1 rollcall.c1
\end{lstlisting}
You'll need to delete this \lstinline'main()'
function before compiling with \lstinline'test-rollcall.c1' below.

\begin{part}\TAGS{application, dictionary, interface}
  Write functions that increment a student's attendance record.
\begin{quote}
\begin{lstlisting}
void mark_present(hdict_t H, string andrew_id)
//@requires H != NULL;

void mark_absent(hdict_t H, string andrew_id)
//@requires H != NULL;
\end{lstlisting}
\end{quote}
These functions should manipulate the \lstinline'days_present' and
\lstinline'days_absent' fields stored in the hash table, so that
\lstinline'hdict_lookup' can access these fields later on.

\threePT
\begin{solution}
\begin{lstlisting}[numbers=left, name="rollcall"]
void mark_present(hdict_t H, string andrew_id)
//@requires H != NULL;
{
    string* k = alloc(string);
    *k = andrew_id;
    student* s = (student*)hdict_lookup(H, (void*)k);
    if (s == NULL) return; //bail out if we can't find the student
    s->days_present++;
}

void mark_absent(hdict_t H, string andrew_id)
//@requires H != NULL;
{
    string* k = alloc(string);
    *k = andrew_id;
    student* s = (student*)hdict_lookup(H, (void*)k);
    if (s == NULL) return; //bail out if we can't find the student
    s->days_absent++;
}
\end{lstlisting}
\end{solution}
\end{part}

You can compile and run your code with \lstinline'test-rollcall.c1':
\begin{lstlisting}[language={[coin]C}]
% cc0 -d hdict.c1 rollcall.c1 test-rollcall.c1
% ./a.out
Enrolling bovik, rjsimmon, fp, and niveditc... done.
Student gburdell is not enrolled...
Student bovik is enrolled...
Student rjsimmon is enrolled...
Student twm is not enrolled...

Student bovik: 5 present, 4 absent...
Student rjsimmon: 8 present, 1 absent...
Student niveditc: 8 present, 1 absent...
Student fp: 2 present, 7 absent...
Done!
\end{lstlisting}
\enlargethispage{5ex}
