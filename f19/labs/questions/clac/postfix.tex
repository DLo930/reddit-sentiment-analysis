\section*{Postfix expressions}

You are used to infix arithmetic expressions where the operator is in
between its two operands (e.g., \lstinline'3 + 4'). In \emph{postfix}
expressions, the operand follows (``post'') its two operands (e.g.,
\lstinline'3 4 +').  Postfix expressions can be used as operands in
other postfix expressions without the need for parentheses. Here are
some examples:

\begin{lstlisting}[basicstyle=\smallbasicstyle, belowskip=0pt]
INFIX               POSTFIX
1 + 2 * 3 - 4       1 2 3 * + 4 -
(1 + 2) * 3 - 4     1 2 + 3 * 4 -
1 + 2 * (3 - 4)     1 2 3 4 - * +
\end{lstlisting}
In an infix expression, the order of operation is determined by
precedence conventions and the use of parentheses.  In a postfix
expression, it is determined by the position of the operators.

% Note that order of operations determines what is converted from infix
% to postfix first. Also, postfix expressions never have parentheses.

To evaluate a postfix expression, we can treat it as a queue of
\emph{tokens} of operands and operators (the front of the queue is on
the left and the back on the right), and then use a stack to evaluate
it.  For each token in the postfix expression, if it is an operand
(e.g., \lstinline'1'), it is pushed on the stack. If it is an
operator, the top two operands are popped from the stack, evaluated
using that operator, and the result is pushed back on the stack. Once
all tokens are processed from the queue (from left to right), the
final result of the computation should be at the top of the stack.

\begin{part}\TAGS{compilation, stack}
Convert the infix expression

\lstinline'125 - 15 * (3 + 2) / (6 * 4 + 1)'

to postfix by hand, and then trace the algorithm described above to compute the
value of the postfix expression. The result should be the same as if you
calculated the infix expression directly.
\end{part}
\onePT

\begin{solution}
\lstinline'125 15 3 2 + * 6 4 * 1 + / -'
\end{solution}
