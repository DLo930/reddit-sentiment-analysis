\section*{Implementing the algorithms}

\begin{part}\TAGS{correctness, loop-invariant, safety}
  Now you have two good sets of loop invariants: in
  \lstinline'reverse.c0', use the algorithm from (1.a) to implement a
  function that reverses the decimal digits in a seven-digit
  nonnegative number using only mathematic and bitwise operators, and
  call it \lstinline'reverse_math' (we've provided a skeleton function
  for you). You shouldn't have to use \texttt{POW} outside of
  contracts.  \textbf{Treat a number with fewer than seven digits as
    if it has leading zeroes.}

\begin{lstlisting}[language={[coin]C}, basicstyle=\smallbasicstyle]
  % coin -d reverse.c0
  --> reverse_math(7654321);
  1234567 (int)
  --> reverse_math(1512200);
  22151 (int)
  --> reverse_math(42);
  2400000 (int)
\end{lstlisting}

You can test your code by running %
\lstinline[language={[coin]C}]'cc0 -d -x reverse.c0 test-math-rev.c0' .

\begin{solution}
\begin{lstlisting}[numbers=left]
int reverse_math(int x)
//@requires 0 <= x && x <= POW(10,8);
//@ensures 0 <= \result && \result <= POW(10,8);
{
  int y = 0;

  for (int i = 0; i < 7; i++)
  //@loop_invariant 0 <= i && i <= 7;
  //@loop_invariant y <= POW(10, i);
  //@loop_invariant x <= POW(10, 7-i);
  {
    y *= 10;
    y += x % 10;
    x /= 10;
  }

  return y;
}
\end{lstlisting}
\end{solution}
\end{part}


\begin{part}\TAGS{array, correctness, loop-invariant, safety}
  In \lstinline'reverse.c0', write a function
  \lstinline'reverse_array' that reverses all the decimal digits of a
  seven-digit nonnegative number using arrays and the algorithm from
  (1.b). Be sure to use the loop invariants you wrote above!

\begin{lstlisting}[language={[coin]C}, basicstyle=\smallbasicstyle]
  % coin -d reverse.c0
  --> reverse_array(7654321);
  1234567 (int)
  --> reverse_array(1512200);
  22151 (int)
  --> reverse_array(42);
  2400000 (int)
\end{lstlisting}

You can test your code by running %
\lstinline[language={[coin]C}]'cc0 -d -x reverse.c0 test-array-rev.c0' .

\begin{solution}
\begin{lstlisting}[numbers=left]
int reverse_array(int x) {
  int[] A = alloc_array(int, 7);
  int y = 0;

  for (int i = 0; i < 7; i++) {
    int a = x % 10;
    A[i] = a;
    x /= 10;
  }

  for (int j = 0; j < 7; j++)
  //@loop_invariant 0 <= j && j <= \length(A);
  //@loop_invariant j == 0 ? y == 0 : y % 10 == A[j-1];
  {
    y *= 10;
    y += A[j];
  }

  return y;
}
\end{lstlisting}
\end{solution}
\end{part}
