Please do not look at your \lstinline'reflect.c0' or \lstinline'blur.c0' code
in this lab!  However, \emph{just for the duration of this lab}, you
can collaborate on writing test cases. (Unless we make an explicit
exception like this, the academic integrity policy for 122 doesn't
allow test cases to be shared, since they are code.)

\begin{part}\TAGS{testing}
Write unit tests in \lstinline'images-test.c0' for either or both of the
following functions described in the images assignment:
\end{part}
\begin{lstlisting}
pixel[] reflect(pixel[] A, int width, int height)
pixel[] blur(pixel[] pixels, int width, int height, int[] mask, int maskwidth)
\end{lstlisting}

We don't have good tools for testing what happens when you give
precondition-violating inputs to these functions, so your test cases
should be valid inputs, and you should confirm that the outputs are
what you expect them to be. Here are some testing tips:
\begin{itemize}
\item%
  For \lstinline'reflect': construct a 1x1 image and a 2x3 image with 4
  distinct pixels. Calculate where you expect these pixels to end up
  in the resulting array. Check that the resulting pixels are what you
  expect using \lstinline'assert()' statements.
\item%
  For \lstinline'blur': construct a 1x1 mask and/or a 3x3 mask
  alongside 1x1, 2x3, 3x4, and 4x5 test images. Calculate what you
  expect the mask calculation to return, and check that it does so
  using \lstinline'assert()' statements.
\end{itemize}

If you've started writing \lstinline'reflect.c0' or \lstinline'blur.c0',
you can compile your test cases locally using

\begin{lstlisting}[language={[coin]C}]
% cc0 -d pixel.c0 imageutil.c0 reflect.c0 blur.c0 images-test.c0
% ./a.out
\end{lstlisting}

(You don't have to include \lstinline'blur.c0' if you're only writing
  tests for \lstinline'reflect', and you don't have to include
  \lstinline'reflect.c0' if you're only writing tests for \lstinline'blur'.)
