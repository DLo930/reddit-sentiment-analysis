\section*{Testing for the images assignment}

\textbf{Setup:}
In this activity, you will write some unit tests for a few of the
image processing functions in the current homework assignment.
The code you'll need for this assignment is the
\lstinline'images-handout' directory for the images programming
assignment. (Download this code if you haven't already.) You should
write your code in a new file, \lstinline'images-test.c0', in this
directory.

%%% Match the programming homework
%Please do not look at your \lstinline'rotate.c0' or \lstinline'mask.c0' code
in this lab!  However, \emph{just for the duration of this lab}, you
can collaborate on writing test cases. (Unless we make an explicit
exception like this, the academic integrity policy for 122 doesn't
allow test cases to be shared, since they are code.)

\begin{part}\TAGS{testing}
Write unit tests in \lstinline'images-test.c0' for either or both of the
following functions described in the images assignment:
\end{part}
\begin{lstlisting}
pixel[] rotate(pixel[] A, int width, int height)
int[] apply_mask(pixel[] pixels, int width, int height, int[] mask, int maskwidth)
\end{lstlisting}

We don't have good tools for testing what happens when you give
precondition-violating inputs to these functions, so your test cases
should be valid inputs, and you should confirm that the outputs are
what you expect them to be. Here are some testing tips:
\begin{itemize}
\item%
  For \lstinline'rotate': construct a 1x1 image and a 2x2 image with 4
  distinct pixels. Calculate where you expect these pixels to end up
  in the resulting array. Check that the resulting pixels are what you
  expect using \lstinline'assert()' statements.
\item%
  For \lstinline'apply_mask': construct a 1x1 mask and/or a 3x3 mask
  alongside 1x1, 2x3, 3x4, and 4x5 test images. Calculate what you
  expect the mask calculation to return, and check that it does so
  using \lstinline'assert()' statements.
\end{itemize}

If you've started writing \lstinline'rotate.c0' or \lstinline'mask.c0',
you can compile your test cases locally using

\begin{lstlisting}[language={[coin]C}]
% cc0 -d pixel.c0 imageutil.c0 rotate.c0 mask.c0 images-test.c0
% ./a.out
\end{lstlisting}

Please do not look at your \lstinline'reflect.c0' or \lstinline'blur.c0' code
in this lab!  However, \emph{just for the duration of this lab}, you
can collaborate on writing test cases. (Unless we make an explicit
exception like this, the academic integrity policy for 122 doesn't
allow test cases to be shared, since they are code.)

\begin{part}\TAGS{testing}
Write unit tests in \lstinline'images-test.c0' for either or both of the
following functions described in the images assignment:
\end{part}
\begin{lstlisting}
pixel[] reflect(pixel[] A, int width, int height)
pixel[] blur(pixel[] pixels, int width, int height, int[] mask, int maskwidth)
\end{lstlisting}

We don't have good tools for testing what happens when you give
precondition-violating inputs to these functions, so your test cases
should be valid inputs, and you should confirm that the outputs are
what you expect them to be. Here are some testing tips:
\begin{itemize}
\item%
  For \lstinline'reflect': construct a 1x1 image and a 2x3 image with 4
  distinct pixels. Calculate where you expect these pixels to end up
  in the resulting array. Check that the resulting pixels are what you
  expect using \lstinline'assert()' statements.
\item%
  For \lstinline'blur': construct a 1x1 mask and/or a 3x3 mask
  alongside 1x1, 2x3, 3x4, and 4x5 test images. Calculate what you
  expect the mask calculation to return, and check that it does so
  using \lstinline'assert()' statements.
\end{itemize}

If you've started writing \lstinline'reflect.c0' or \lstinline'blur.c0',
you can compile your test cases locally using

\begin{lstlisting}[language={[coin]C}]
% cc0 -d pixel.c0 imageutil.c0 reflect.c0 blur.c0 images-test.c0
% ./a.out
\end{lstlisting}

(You don't have to include \lstinline'blur.c0' if you're only writing
  tests for \lstinline'reflect', and you don't have to include
  \lstinline'reflect.c0' if you're only writing tests for \lstinline'blur'.)


If you have not started writing your own images implementation yet, you can
make sure your test cases compile before submitting to Autolab by running

\begin{lstlisting}[language={[coin]C}]
% cc0 /afs/andrew/course/15/122/misc/lab-images/stubs.c0 images-test.c0
\end{lstlisting}

You should also submit \lstinline'images-test.c0' (just that file, not a
\lstinline'.tgz' file) to the ungraded Autolab autograder created for this lab.
This is the one called ``Lab0\TheNumber''.  In the autograder, we will
run your tests against three kinds of implementations:
\begin{enumerate}
\item%
  A correct implementation, which must pass all your tests.
\item%
  Some implementations that fail a contract when given certain
  (reasonable) inputs. We call such bugs \emph{contract failures}.
\item%
  Some implementations that will satisfy all the reasonable
  postconditions we might expect you to write, but which nevertheless
  do obviously wrong things.  We call such bugs \emph{contract
    exploits}.
\end{enumerate}
The autograder for the images assignment will also accept your
\lstinline'images-test.c0' file and record the number of buggy
implementations you catch on the scoreboard!
