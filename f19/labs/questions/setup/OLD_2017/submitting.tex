\section*{Submitting Assignments}

In this class you will be submitting your assignments using two tools
that may be new to you.  This activity has the purpose to make sure
you know what you are doing.

\begin{part}
\subsection*{Programming Assignments}

You will submit your \emph{programming} assignments (and some lab
solutions) using \autolab{}, the site where you downloaded the handout
from.  Let's see how that works on the file \lstinline'factorial.c0'
where you have earlier written your name and section number.

We will typically submit compressed archive files.  You do so by
typing the following at the Unix prompt:

\lstinline[language={[coin]C}]'% tar cfzv testing.tgz factorial.c0'

Next, point your browser to \autolab, go to this lab's entry and
upload the newly created file \lstinline'testing.tgz' using the
``Submit File'' button.

(There is also a way to submit from the prompt.  You'll see that later
on.)

Finally, check that you submitted the right file by clicking on the
``view source'' icon.
\end{part}


\begin{part}
\subsection*{Written Assignments}

You will submit your \emph{written} assignments using
\href{\gradescope}{Gradescope}.  But first we need to register at
\href{https://gradescope.com}{https://gradescope.com} \textbf{using
  YOUR ANDREW EMAIL} and entry code \textbf{\gradescopeEntryCode}.
You will then use the file \lstinline'written-test.pdf', which was
also in the handout.  You will write just your name and section number
and then submit.

You can do the writing in two ways:
\begin{enumerate}
\item%
  By entering annotations in \lstinline'written-test.pdf' and then
  saving.  Several (but not all) PDF viewers support annotations.
  Examples are \lstinline'preview' on Mac, \lstinline'iAnnotate' on
  iOS and Android, and \lstinline'Acrobat Pro' on pretty much
  anything --- \lstinline'Acrobat Pro' is installed in all non-CS cluster
  machines.
\item%
  By printing \lstinline'written-test.pdf', writing your solution by
  hand, and then scanning it back to PDF\@.  \emph{This is pretty
    laborious: you will want to get the first option to work for you.}
\end{enumerate}
You will use the first option since we are on a cluster machine.

Now that you have a ``solved'' version of
\lstinline'written-test.pdf', point your browser to
\href{\gradescope}{Gradescope} and upload your solution file.  That's
it!  If you didn't manage to find an editor that works for you, simply
submit the blank writeup for now.

\end{part}
