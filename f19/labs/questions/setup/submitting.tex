\newpage
\section*{Submitting assignments}

In this class you will be submitting your assignments using two tools
that may be new to you.  This activity has the purpose to make sure
you know what you are doing.


\subsection*{Written assignments}
\begin{part}\TAGS{course-policies}
You will submit your \emph{written} assignments using
\href{\gradescope}{Gradescope}.  But first we need to register at
\href{https://gradescope.com}{https://gradescope.com} \textbf{using
  YOUR ANDREW EMAIL} and entry code \textbf{\gradescopeEntryCode}.

Because Gradescope doesn't support posting homeworks, we do so on
\writtenDistributor{}.  In a browser, go to \writtenDistributor, click on %
``\lstinline'Written 0 (UNGRADED)''' and then on %
``\lstinline'View writeup'''.  Save a copy as \lstinline'written-test.pdf'.
You will write just your name and section number and then submit.

You can do the writing in two ways:
\begin{enumerate}
\item%
  By entering annotations in \lstinline'written-test.pdf' and then
  saving.  The easiest way is to do so online by using
  \url{pdfescape.com} in your browser.  Alternatively, several (but
  not all) PDF viewers support annotations.  Examples are
  \lstinline'preview' on Mac, \lstinline'iAnnotate' on iOS and
  Android, and \lstinline'Acrobat Pro' on pretty much anything ---
  \lstinline'Acrobat Pro' is installed in all non-CS cluster machines.

\item%
  By printing \lstinline'written-test.pdf', writing your solution by
  hand, and then scanning it back to PDF\@.  \emph{This is pretty
    laborious: you will want to get the first option to work for you.}
\end{enumerate}
Unless you are near a scanner, you will use the first option.

Now that you have a ``solved'' version of
\lstinline'written-test.pdf', point your browser to
\href{\gradescope}{Gradescope} and upload your solution file.  That's
it!  If you didn't manage to find a PDF editor that works for you,
simply submit the blank writeup for now.
\end{part}


\subsection*{Programming assignments}
\begin{part}\TAGS{course-policies}
You will submit your \emph{programming} assignments (and some lab
solutions) using \autolab{}, the site where you downloaded the handout
from.  Let's see how that works on the file \lstinline'factorial.c0'
where you earlier wrote your name and section number.

We will typically submit compressed archive files.  You do so by
typing the following at the Unix prompt:

\lstinline[language={[coin]C}]'% tar cfzv testing.tgz factorial.c0'

The above command puts your file \lstinline'factorial.c0' into a
compressed file named \lstinline'testing.tgz'.

\begin{description}
\item[{\em If you are on your own laptop}], you need to move the
    \lstinline'testing.tgz' file
    from the andrew server that you ssh'd into to your own laptop.
    If your laptop is running Linux or Mac, use the following command
    to do so:
        \begin{lstlisting}[language={[coin]C}]
% scp <your_id>@unix.andrew.cmu.edu:private/15122/lab01/handout/testing.tgz .
        \end{lstlisting}
Note that you need to run this command from a terminal where
    you are \emph{not} ssh'd into Andrew (i.e., open a fresh terminal
    and run the command there).

If you're on Windows, MobaXTerm has a way to help you transfer the file.
Ask a TA to help you if you can't find the tool.
\end{description}

Next, point your browser to \autolab, go to this lab's entry and
upload the newly created file \lstinline'testing.tgz' using the
``Submit File'' button.

(There is also a way to submit from the prompt.  You'll see that later
on.)

Finally, check that you submitted the right file by clicking on the
``view source'' icon.  \textbf{Always check your submissions!}
\end{part}


% At this point you are done with the lab, and you are free to go.
%  \textbf{(4.c)} is strictly optional


% \begin{part}\TAGS{unix}
%   If you have your laptop, you can try to work through the first step
%   of C0 at CMU, which can be found at
%   \url{http://c0.typesafety.net/tutorial/C0-at-CMU.html},
%   \emph{Connecting through SSH}, on your laptop. However, we'll have a
%   laptop session on \lapsessionTime{} in \lapsessionPlace{} for you to
%   get your laptop set up.
% \end{part}
