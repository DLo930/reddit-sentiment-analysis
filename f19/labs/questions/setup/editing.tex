\subsection*{Editing your program}

You can use any editor you wish to write and edit your programs, but
we highly recommend you try out \lstinline[language={[coin]C}]'vim'
and \lstinline[language={[coin]C}]'emacs' since these editors can do
much more than just help you edit your code (as you will see). For
this lab, \textbf{try both of them} by following the
instructions below.  Later, use the one you like best\footnote{
See the \href{http://www.cs.cmu.edu/~15122/about.shtml}{course website}
for some links to learn more about using these editors. They both
are capable of a lot more than we describe here.}.

\begin{part}\TAGS{unix}
  Open the file \lstinline[language={[coin]C}]'factorial.c0' that you
  downloaded from the previous part of the lab:

  \textbf{VIM:}\\ \lstinline[language={[coin]C}]'% vim factorial.c0'

  \textbf{EMACS:}\\ \lstinline[language={[coin]C}]'% emacs factorial.c0'
\end{part}

If a file doesn't exist, the editor will start with a new empty file.
You should see the editor start in the Terminal window, and you should
see a program that looks like it computes factorial. The program is
written in C0, the language we'll be using to start the semester.

\begin{part}\TAGS{unix}
  Edit the program and add your name and section letter at the
  appropriate locations. Use the instructions below for the editor
  you're using.

  \textbf{VIM:} This editor has two modes, \emph{insert mode} where
  you can insert text, and \emph{command mode} where you can enter
  commands.  It starts in command mode so you can't edit
  immediately. Use the arrow keys to move around the file. While in
  command mode, if you press ``\lstinline'i''', the editor changes you
  to insert mode, allowing you to type text. Go to insert mode and
  add your name and section letter.
  Press the Escape (ESC) key while in insert mode to return
  to command mode.

  \textbf{EMACS:} You can just start typing and editing without
  hitting special keys. You can use the arrow keys to navigate around
  the file to insert code. There are many shortcuts and built-in
  features to emacs but you don't need them right now. In the file,
  insert your name and your section letter in the appropriate comments
  in your program.

\end{part}

\begin{part}\TAGS{unix}
  Save your changes and exit the editor.

  \textbf{VIM:} Make sure you're in command mode by pressing ESC\@.
  Then, you can save your work and exit vim by entering the
  sequence ``\lstinline':wq''' followed by pressing Enter.
  (If you have unsaved changes you would like to discard,
  you'll have to enter the sequence ``\lstinline':q!''' followed by Enter.
  You can also save without exiting by entering the sequence
  ``\lstinline':w''' followed by Enter.)

  \textbf{EMACS:} Once you're ready to save, press Ctrl-x (the Control
  key and the ``\lstinline'x''' key at the same time) followed by
  Ctrl-s. You can exit by pressing Ctrl-x followed by Ctrl-c.  If you
  have not saved before exiting, Emacs will ask you whether you want
  to save your file (since you changed it) --- press ``\lstinline'y'''
  for yes. (Press ``\lstinline'n''' instead if you don't want to save
  your changes).

\end{part}

\begin{part}\TAGS{unix}
Try out your new editing skills on the file \lstinline'favorite_number.c0'.
You should see where the file tells you to add a line that returns your
favorite number, like this:
\begin{lstlisting}[language={[coin]C}, belowskip=0pt]
int my_favorite_number() {
    /* add a line below that returns your favorite number */
    return 17; // this is *my* favorite number. Choose your own.
}
\end{lstlisting}
\end{part}
