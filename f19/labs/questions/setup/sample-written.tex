\newpage
\section*{Written Assignments}

\begin{part}\TAGS{course-policies}
You will submit your \emph{written} assignments using
\href{\gradescope}{Gradescope}.  But first we need to register at
\href{https://gradescope.com}{https://gradescope.com} \textbf{using
  YOUR ANDREW EMAIL} and entry code \textbf{\gradescopeEntryCode}.

Written homeworks are posted on \writtenDistributor{}.  In a browser,
go to \writtenDistributor{}, and click on \linebreak%
``\lstinline'Written 0 (UNGRADED)''' under ``Written Homeworks''.  %
The printer icon in the top
right corner will give you a version you can save as a PDF\@.  Save a
copy as \lstinline'written-test.pdf'.  You will write just your name
and section number and then submit.

You can do the writing in two ways:
\begin{enumerate}
\item%
  By entering annotations in \lstinline'written-test.pdf' and then
  saving.  The easiest way is to do so online by using
  \url{pdfescape.com} in your browser.  Alternatively, several
  PDF viewers support annotations, including
  \lstinline'preview' on Mac, \lstinline'iAnnotate' on iOS and
  Android, and \lstinline'Acrobat Pro' on pretty much anything ---
  \lstinline'Acrobat Pro' is installed in all non-CS cluster machines.

\item%
  By printing \lstinline'written-test.pdf', writing your solution by
  hand, and then scanning it back to PDF\@.  \emph{This is pretty
    laborious: you will want to get the first option to work for you.}
\end{enumerate}

Now that you have a ``solved'' version of
\lstinline'written-test.pdf', go to \href{\gradescope}{Gradescope} and
upload your solution file.  \textbf{\em Always check your submission
  to be sure it looks correct!}  If you didn't manage to find a PDF
editor that works for you, simply submit the blank writeup for now.
\end{part}
