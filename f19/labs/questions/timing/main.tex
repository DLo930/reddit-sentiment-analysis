\section*{Timing code}

In Unix, there is a way to determine the actual running time of a
program.  You use the \lstinline'time' command followed by the program
name (and its arguments) that you want to time.  For example, to time
an executable \lstinline'a.out' in your current directory, you would
enter:

\begin{lstlisting}[language={[coin]C}]
% time ./a.out
Testing with n = 1000... Done. 0
4.602u 0.015s 0:04.63 99.5% 0+0k 0+0io 0pf+0w
\end{lstlisting}
The first number (the one with a \lstinline'u' after it) is the best
one to track for this activity. The ``u'' is for \emph{user}, which is
closer to what we want than the system time (the amount of time the
program handed control over to the operating system) or the wall clock
time.

\begin{part}\TAGS{complexity, testing}
  In the lab directory, run the following commands:

\begin{lstlisting}[language={[coin]C}]
% make LFtest
% make memotest
\end{lstlisting}

This will create two programs, \lstinline'LFtest' and
\lstinline'memotest' that take an argument \lstinline'n', and print
the result of lagged fibonacci performed on the arguments
\lstinline'(n, 1, 2)' --- these are actually just the regular Fibonacci
numbers!  You can use these to time both the specification function
and your new function. For example, to time the LF specification
function using input 10, you would enter:

\begin{lstlisting}[language={[coin]C}]
% time ./LFtest 10
\end{lstlisting}

Your job is to determine the asymptotic complexity (runtime) of each
program expressed using big-O notation as a function of $n$, in its
simplest, tightest form.
\end{part}

You can determine the complexity of a program by running it for
varying values of $n$ and then plotting your results (or looking for
obvious patterns).

Keep in mind that the specification function should have a
\emph{significantly} worse asymptotic complexity than your
\lstinline'fast_LF' implementation.

\threePT

\begin{solution}\par
\begin{minipage}{0.5\textwidth}
  \lstinline'LFtest': exponential --- $O(2^n)$
\\\lstinline'memotest': linear --- $O(n)$
\end{minipage}
\end{solution}
