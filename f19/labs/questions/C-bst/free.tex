\section*{Memory leaks}

As we learned in lecture, C doesn't do memory management like was done
in C0.  As a result, we need to use the \lstinline'free' function to
free memory for use by the operating system.

We can't free everything, though! As the implementation (because our
data structure is generic), we have no way of knowing how to free the
data provided by the client. As such, our freeing function takes a
function pointer from the client that frees elements (or is
\lstinline'NULL' in the case of the client wishing to not free the
memory).

\begin{part}\TAGS{c-memory}
Extend the BST interface with a new type and a new function:

\begin{lstlisting}
typedef void free_fn(void* x);
void bst_free(bst_t B, free_fn* F)
  /*@requires B != NULL; @*/ ;
\end{lstlisting}

Implement this function that frees all the BST's internal memory, and
that, if \lstinline'F' is not \lstinline'NULL', also runs the provided
function to free all the \lstinline'void*' pointers stored in the
tree's \lstinline'data' fields.

\begin{solution}
  TODO
\end{solution}
\end{part}
