\bgroup
\begin{wrapfigure}{r}{.38\textwidth}
\vspace{-.2in}
{\footnotesize
~\quad$\begin{array}{|lp{10pt}l|lp{10pt}l|lp{10pt}l|}\hline
\multicolumn{9}{|c|}{\textit{Partial ASCII Table}}\\\hline
     32 & 20 & \textrm{\textvisiblespace} &  64 & 40 & \texttt{@} &  96 & 60 & \texttt{`} \\
     33 & 21 & \texttt{!}      &  65 & 41 & \texttt{A}      &  97 & 61 & \texttt{a} \\
     34 & 22 & \texttt{"}      &  66 & 42 & \texttt{B}      &  98 & 62 & \texttt{b} \\
     35 & 23 & \texttt{\#}     &  67 & 43 & \texttt{C}      &  99 & 63 & \texttt{c} \\
     36 & 24 & \texttt{\$}     &  68 & 44 & \texttt{D}      & 100 & 64 & \texttt{d} \\
     37 & 25 & \texttt{\%}     &  69 & 45 & \texttt{E}      & 101 & 65 & \texttt{e} \\
     38 & 26 & \texttt{\&}     &  70 & 46 & \texttt{F}      & 102 & 66 & \texttt{f} \\
     39 & 27 & \texttt{'}      &  71 & 47 & \texttt{G}      & 103 & 67 & \texttt{g} \\
     40 & 28 & \texttt{(}      &  72 & 48 & \texttt{H}      & 104 & 68 & \texttt{h} \\
     41 & 29 & \texttt{)}      &  73 & 49 & \texttt{I}      & 105 & 69 & \texttt{i} \\
     42 & 2A & \texttt{*}      &  74 & 4A & \texttt{J}      & 106 & 6A & \texttt{j} \\
     43 & 2B & \texttt{+}      &  75 & 4B & \texttt{K}      & 107 & 6B & \texttt{k} \\
     44 & 2C & \texttt{,}      &  76 & 4C & \texttt{L}      & 108 & 6C & \texttt{l} \\
     45 & 2D & \texttt{-}      &  77 & 4D & \texttt{M}      & 109 & 6D & \texttt{m} \\
     46 & 2E & \texttt{.}      &  78 & 4E & \texttt{N}      & 110 & 6E & \texttt{n} \\
     47 & 2F & \texttt{/}      &  79 & 4F & \texttt{O}      & 111 & 6F & \texttt{o} \\
     48 & 30 & \texttt{0}      &  80 & 50 & \texttt{P}      & 112 & 70 & \texttt{p} \\
     49 & 31 & \texttt{1}      &  81 & 51 & \texttt{Q}      & 113 & 71 & \texttt{q} \\
     50 & 32 & \texttt{2}      &  82 & 52 & \texttt{R}      & 114 & 72 & \texttt{r} \\
     51 & 33 & \texttt{3}      &  83 & 53 & \texttt{S}      & 115 & 73 & \texttt{s} \\
     52 & 34 & \texttt{4}      &  84 & 54 & \texttt{T}      & 116 & 74 & \texttt{t} \\
     53 & 35 & \texttt{5}      &  85 & 55 & \texttt{U}      & 117 & 75 & \texttt{u} \\
     54 & 36 & \texttt{6}      &  86 & 56 & \texttt{V}      & 118 & 76 & \texttt{v} \\
     55 & 37 & \texttt{7}      &  87 & 57 & \texttt{W}      & 119 & 77 & \texttt{w} \\
     56 & 38 & \texttt{8}      &  88 & 58 & \texttt{X}      & 120 & 78 & \texttt{x} \\
     57 & 39 & \texttt{9}      &  89 & 59 & \texttt{Y}      & 121 & 79 & \texttt{y} \\
     58 & 3A & \texttt{:}      &  90 & 5A & \texttt{Z}      & 122 & 7A & \texttt{z} \\
     59 & 3B & \texttt{;}      &  91 & 5B & \texttt{[}      & 123 & 7B & \texttt{\{} \\
     60 & 3C & \texttt{<}      &  92 & 5C & \texttt{\textbackslash}       & 124 & 7C & \texttt{|} \\
     61 & 3D & \texttt{=}      &  93 & 5D & \texttt{]}      & 125 & 7D & \texttt{\}} \\
     62 & 3E & \texttt{>}      &  94 & 5E & \texttt{\textasciicircum}       & 126 & 7E & \sim \\
     63 & 3F & \texttt{?}      &  95 & 5F & \texttt{\_}     &  &  &  \\
\hline
\end{array}$ }
\end{wrapfigure}

\section*{Storing and using strings in C}

Load the file \lstinline'ex1.c' into a text editor. Read through the file and write down what you think the output will be before you run the program:

\lstinline[language={[coin]C}]'word string: '\shortanswerline{ace\hspace*{9.1em}}

\lstinline[language={[coin]C}]'word ASCII values: '%
\shortanswerline{97\hspace*{0.9em}}~~\shortanswerline{99\hspace*{0.9em}}~~\shortanswerline{101\hspace*{0.4em}}~~\shortanswerline{0\hspace*{1.4em}}

Once you have done this, compile with the following command (all on one line):

\texttt{\% gcc -Wall -Wextra -Werror -Wshadow -std=c99}\\
\hspace*{0.5in} \texttt{-pedantic -g ex1.c}

\begin{solution}\\
word string: \texttt{ace}\\
word ASCII values: \texttt{97 99 101 0}
\end{solution}

\begin{part}\TAGS{c-numbers, c-memory, string}
  Which parts differed from what you expected?

\end{part}

\begin{part}\TAGS{c-memory, string, testing}
  Change the `\texttt{\textbackslash0}' character in the array to
  something else, like `\texttt{d}'. Predict how this will change the
  answer, and then compile and see if you're right.

\begin{solution}\\
word string: \texttt{aced<??gibberish??>}\\
word ASCII values: \texttt{97 99 101 100}
\end{solution}
\end{part}

\begin{part}\TAGS{c-memory, correctness, string, testing}
  Run the modified code under \lstinline'valgrind', and read through
  its output to see which lines in \lstinline'ex1.c' are given as part
  of the output.

\begin{solution}
\begin{lstlisting}[language={[coin]C}]
Conditional jump or move depends on uninitialised value(s)
      at 0x4E7DA94: vfprintf (in /usr/lib64/libc-2.17.so)
      by 0x4E86C18: printf (in /usr/lib64/libc-2.17.so)
      by 0x40055D: main (ex1.c:10)
\end{lstlisting}
\end{solution}
\end{part}

At this point, compare notes with people around you to see if you have
the same answers for (1.b) and (1.c).  Ask a TA if there's anything
you're unsure about!

\onePT
\egroup