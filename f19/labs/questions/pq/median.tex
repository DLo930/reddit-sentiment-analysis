\section*{Finding the median}

We know how to return the element with highest priority out of a
priority queue.  Now, let's find the element with the $k$-th highest
priority (if $k$ is 1, it returns the element of \textbf{highest priority}).

\begin{part}\TAGS{application, interface, pq}
  In file \lstinline'pqmedian.c1', complete the implementation of the
  function \lstinline'k_priority(H,k)' that returns the $k$-th
  priority element in priority queue \lstinline'H'.  On return,
  \lstinline'H' should contain the same elements as when the function
  was called.
\begin{solution}
\begin{lstlisting}
elem  k_priority(pq_t H, int k)
//@requires H != NULL && !pq_empty(H);
//@requires 1 <= k && k <= pq_size(H);
{
  // allocate a k-element array
  elem[] temp = alloc_array(elem, k);

  // save the first k elements of H in temp
  for (int i = 0; i < k; i++) {
    temp[i] = pq_rem(H);
  }
  // temp[k-1] contains the k-th priority element

  // put everything back into H
  for (int i = 0; i < k; i++) {
     pq_add(H, temp[i]);
  }

  return temp[k-1];
}
\end{lstlisting}
\end{solution}
\end{part}


The \emph{median} of a collection $H$ is the element $m$
in $H$ so that half of the other elements of $H$ are larger than or
equal to $m$ and the other half is smaller or equal to $m$.  If $H$
contains an even number of elements, this definition is ambiguous
since it asks us to take ``half'' of an odd number of elements.  In
this case, we (inaccurately) let the ``smaller half''
have one element more than the ``larger half''.\footnote{You can find
  the actual definition of median online.  Why can't we use it in this
  exercise?}


\begin{part}\TAGS{application, interface, pq}
  Also in file \lstinline'pqmedian.c1', complete the implementation of
  the function \lstinline'median(H)' that returns the median element
  in priority queue \lstinline'H'.  On return, \lstinline'H' should
  contain the same elements as when the function was called.  Hint:
  the function \lstinline'pq_size' may come handy.
\begin{solution}
\begin{lstlisting}
elem median(pq_t H)
//@requires H != NULL && !pq_empty(H);
{
  return k_priority(H, pq_size(H)/2);
}

\end{lstlisting}
\end{solution}
\end{part}

Compile and test your code by running this command:

\lstinline[language={[coin]C}]'% cc0 -d -x pq.c1 pqmedian.c1'

\threePT