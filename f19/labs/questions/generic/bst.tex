\section*{Reviewing the BST implementation}

Recall that we implemented a binary search tree in lecture as nodes, structs
containing data and two (possibly \lstinline'NULL') pointers
\lstinline'left' and \lstinline'right'.

\bgroup
\smalllistings
\begin{lstlisting}
  typedef void* elem;
  typedef int compare_fn(elem x, elem y)
      /*@requires x != NULL && y != NULL; @*/;

  typedef struct tree_node tree;
  struct tree_node {
      elem data;
      tree* left;
      tree* right;
  };

  typedef struct bst_header bst;
  struct bst_header {
      tree* root;
      compare_fn* compare; // Function pointer must be non-NULL
  };
\end{lstlisting}

Functions that the client will call are usually implemented as a
function that takes a \lstinline'bst*'.  This function then calls an
internal helper function that actually does the work, node by node,
hiding the internal representation of the tree.  These helper
functions are often implemented recursively due to their naturally
recursive nature.

\egroup