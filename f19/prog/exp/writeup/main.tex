\documentclass[12pt]{exam}
\newcommand{\hwnumber}{5}
\newcommand{\hwname}{Clac and EXP}
\newcommand{\duedate}{\formatdate{28}{02}{\YEAR} by \progDueTime}

\usepackage{../misc/latex/edition}  % Course semester
\usepackage{../misc/latex/c0}       % Listings style for c0
\usepackage{amsmath}
\usepackage{enumerate}
\usepackage[normalem]{ulem}
\usepackage{verbatim}
\usepackage[left=1in, right=1in, top=1in, bottom=1in]{geometry}
\usepackage{graphicx}
\usepackage{hyperref}
\usepackage{tikz}     \usetikzlibrary{shapes}
\usepackage{fancybox}
\usepackage[all]{xy}
\usepackage{wrapfig}
\usepackage{fancyvrb}
\usepackage{datetime}
\usepackage{etoolbox}
\usepackage{calc}
\usepackage[nomessages]{fp}
\usepackage{import}  % Like input and include, but respects subdirectories

\newcommand{\defaultQuestionLocation}{questions}
\newcommand{\inputQuestion}[2][\defaultQuestionLocation/]{%
  \subimport{#1}{#2}
}
% Subdirectories of \defaultQuestionLocation containing code and pictures
\newcommand{\code}{code}
\newcommand{\img}{img}


%%% ic: frontmatter macros
\newcommand{\specialInstructions}{}
\newcommand{\HWNUMBER}
{\ifdefempty{\hwnumber}{__}{%
  \ifnumless{\hwnumber}{10}{0\hwnumber}{\hwnumber}}}
\newcommand{\hwtype}{Written Homework}

%%% ic: 'exam' tweaks
\renewcommand{\half}{.5} % Half points

\newcommand{\Question}[2][]
 {\ifstrempty{#1}
    {\question{\bf #2}}
    {\question[#1]{\bf #2}}
  \immediate\write\rubricfile{}%
  \immediate\write\rubricfile{Question \thequestiontitle:}%
  \immediate\write\rubricfile{==========}
 }

%%% ic: Support for editable PDF
% counter name (some viewers misbehave if always the same)
\newcounter{editable}
\newcommand{\nextField}{\addtocounter{editable}{1}q\arabic{editable}}
\newcommand{\NextField}
 {\makebox[0pt][r]{\scalebox{0.1}{\color{White}\nextField}}}

% Color of edit area
\newcommand{\editAreaColor}{red}
% Single line answer:   \editableLine[extra parameters (optional)]{line width}
\newcommand{\editableLine}[2][]
{\textcolor{\editAreaColor}{%
 \underline{\hspace*{-0.25em}%
 \raisebox{-0.5ex}{%
 \TextField[width=#2, borderwidth=0, #1]{\NextField}}}}%
}
% Single line answer for code:  \editableLine[extra parameters (optional)]{line width}
\newcommand{\editableCodeLine}[2][]
{\textcolor{\editAreaColor}{%
 \underline{%
 \TextField[width=#2, height=1.5ex, borderwidth=0, #1]{\NextField}}}}
% Multiline answer:  \editableLine[extra parameters (optional)]{box height}
\newcommand{\editableBox}[2][]
{\leavevmode\hspace*{-0.1em}%
\TextField[height=#2, width=\linewidth,
           multiline=true, borderwidth=0.1, bordercolor=\editAreaColor,
           #1]{\NextField}}

%%%%% Same answer format as exams
\renewcommand{\rmdefault}{ppl}
\renewcommand{\sfdefault}{phv}
\newcommand{\answerColor}{Blue}

\ifprintanswers
\newcommand{\answer}[2]{\makebox[#1][c]{\color{\answerColor}#2}}
\else
\newcommand{\answer}[2]{\makebox[#1][c]{}\makebox[0pt]{\phantom{|}}}
\fi
\newcommand{\uanswer}[2]{\underline{\answer{#1}{#2}}}


%%% Write rubric snippet.  Usage:
% \RUBRIC
% any multi-line text (including \, #, %, whatever)
% ENDRUBRIC
%% (ENDRUBRIC should be on a line by itself)
\makeatletter
\def\RUBRIC
 {%
  \begingroup
  \let\do\@makeother\dospecials
  \endlinechar=`\^^J
  \@tofile%
 }
\def\ENDRUBRIC{ENDRUBRIC}
\def\@tofile#1^^J{%
  \def\@test{#1}%
  \ifx\@test\ENDRUBRIC
    \immediate\write\rubricfile{}  % End with an empty line
    \expandafter\@firstoftwo
  \else
    \expandafter\@secondoftwo
  \fi
  {\endgroup}%
  {\toks@{#1}%
   \begingroup\endlinechar=\m@ne
   \everyeof{\noexpand}%
   \xdef\@temp{\scantokens\expandafter{\the\toks@}}%
   \endgroup
   \immediate\write\rubricfile{\@temp}%
   \@tofile}%
}
\makeatother

%% Displays tags for an exercise in 'answer' mode
\newcommand{\TAGS}[1]
{\ifprintanswers%
  \rule{0em}{0ex}%
  \marginpar{\footnotesize%
    \fcolorbox{black}{Gray!25}{%
      \parbox[t]{2cm}{\raggedright\textbf{TAGS:}\\#1}}}%
  \ignorespaces%
 \fi}%


%% Page layout
\pagestyle{headandfoot}

\headrule
\header{\textbf{\courseNumber{} \hwtype{} \hwnumber}}
       {}
       {\textbf{Page \thepage\ of \numpages}}
\footrule
\footer{}{}{\COPYRIGHT}

\renewcommand{\partlabel}{\textbf{\thequestion.\thepartno}}
%\renewcommand{\partlabel}{\textbf{Task \thepartno}}
\renewcommand{\subpartlabel}{\textbf{\thesubpart.}}
\renewcommand{\thepartno}{\arabic{partno}}
\renewcommand{\thesubpart}{\alph{subpart}}
\pointpoints{pt}{pts}
\pointformat{\raisebox{0ex}[\height][0pt]{\fcolorbox{black}{yellow}{\themarginpoints}}}
\bonuspointformat{\raisebox{0ex}[\height][0pt]{\fcolorbox{black}{red}{\themarginpoints}}}
\marginpointname{\points}
\pointsinmargin
%\boxedpoints

\setlength\answerlinelength{2in}
\setlength\answerskip{0.3in}

\newcommand{\mkWrittenTitle}[1]{#1}
\newcommand{\mkDueDate}[1]{#1}
\newcommand{\mkEvalSummary}[1]{#1}
\newcommand{\mkGradetable}[1]{#1}



% This fixes an issue with the exam package version 2.6 and after,
% where 'framed' has been renamed to 'examframed' to avoid a conflict.
\ifcsmacro{examframed}{%
\newenvironment{framed}
{\begin{examframed}}
{\end{examframed}}
}{}

\lstdefinelanguage[clac]{C}[C0]{C}{%
  numbers=none,
  deletekeywords={if},
  alsoletter={+, -, *, /, \%, <, :, ;},
%  morekeywords={print, quit, +, -, *, /, \%, <, drop, swap, rot, if, pick, skip},
  morekeywords={:, ;},
  moredelim=[l][\shellstyle]{\%},
  moredelim=[l][\coinheaderstyle]{Clac\ top\ level},
  moredelim=[l][\coinheaderstyle]{EXP\ top\ level},
  moredelim=**[l][\coininputstyle]{clac>>},
  moredelim=**[l][\coininputstyle]{exp>>},
  moredelim=[l][\coinerrorstyle]{Abort\ trap:},
}

\begin{document}
\hwTitle

\noindent
In this assignment, you will investigate stacks, queues, and
dictionaries by implementing two programming languages. The first
language, Clac, is a strange new programming language, but it is one
that is easy to implement. The second language, EXP, looks more
familiar.

\bigskip
\noindent
The code handout for this assignment is on \autolab{} and at
\begin{center}
\whereisthetgz{exp-handout.tgz}
\end{center}
The file \lstinline'README.txt' in the code handout goes over the
contents of the handout and explains how to hand the assignment in.
There is a SEVEN (7) PENALTY-FREE HANDIN LIMIT.
Every additional handin will incur a small (5\%) penalty (even if
using a late day).

\paragraph{Reminder about respecting interfaces:}
In this assignment, we will take interfaces seriously. When using the
implementations of stacks and queues in the \lstinline'lib' directory of
the handout, you should \emph{only} use the interfaces. We will
enforce this, in the autograder, by compiling your code against
different implementations of the interfaces. Similarly, when working
on EXP in Section 3, you should respect the interface of the dictionaries
you implemented in Section 2.

\paragraph{Style Grading:}
In this assignment, we will continue emphasizing \emph{programming
  style}.  We will be looking at your code and evaluating it based on
the criteria outlined at
\url{http://cs.cmu.edu/~15122/misc/styleguide.pdf}.  We will make
comments on your code via Autolab, and will assign an overall passing
or failing style grade.  A failing style grade will be temporarily
represented as a score of -15 points.  This -15 will be reset to 0
once you:
\begin{enumerate}
\itemsep=-3pt
\item%
  fix the style issues,
\item%
  see a member of the course staff during office hours \textbf{within
    5 days} after the grades are released, and
\item%
  briefly discuss the style issues and how they were addressed.
\end{enumerate}
We will evaluate your code for style in two ways. We will use
\lstinline'cc0' with the \lstinline'-w' flag that gives style warnings
--- code that raises warnings with this flag is almost certain to fail
style grading.  Because the \lstinline'-w' flag does not check for good
variable names, appropriate comments, or appropriate use of
functions, these issues will be checked by hand.

\newpage
\section{Introducing Clac}
\label{sec:demo}

Clac is a new stack-based programming language developed by a
Pittsburgh-area startup called Reverse Polish Systems (RPS).  Any
similarities of Clac with Forth or PostScript are purely coincidental.

Clac works like an interactive calculator.  When it runs, it maintains
an \emph{operand stack}.  Entering numbers will simply push them onto
the operand stack.  When an operation such as addition \lstinline'+' or
multiplication \lstinline'*' is encountered, it will be applied to the top
elements of the stack (consuming them in the process) and the result
is pushed back onto the stack.  When a newline is read, the number on
top of the stack will be printed.  This is an example where we start Clac
and type \lstinline'3 4 *', followed by a newline:
\begin{quote}
\begin{lstlisting}[language={[clac]C}]
% ./clac
Clac top level
clac>> 3 4 *
Stack is now [12]
\end{lstlisting}
\end{quote}
Clac responded by printing the stack, which now contains only \lstinline'12'.
We now enter \lstinline'-9 2 /' and a newline:
\begin{quote}
\begin{lstlisting}[language={[clac]C}]
clac>> -9 2 /
Stack is now [12,-4]
\end{lstlisting}
\end{quote}
At this point the stack contains two integers: \lstinline'12' (the result of
the multiplication) and \lstinline'-4' (the result of the integer division,
which is at the top of the stack). We can add them simply by typing
\lstinline'+' and a newline.
\begin{quote}
\begin{lstlisting}[language={[clac]C}]
clac>> +
Stack is now [8]
\end{lstlisting}
\end{quote}
The stack contains only \lstinline'8', since $12+(-4) = 8$.  Note that the
integers and operators we input into Clac (they are called \emph{tokens}
below) work like a queue: Clac processes them left to right and we enter new
ones on the right.


\subsection{Rules of Clac}

Every Clac operation that you will implement removes a single string
(a \emph{token}) from an input queue, and then modifies the operand stack. To specify operations, we use the notation
$$
S \longrightarrow S'
$$
to mean that the stack $S$ transitions to become stack $S'$.
Stacks are written with the \emph{top element at the right end}!
For example, the action of subtraction is stated as
$$
\mbox{\texttt{-}} : S, x, y \longrightarrow S, x{-}y
$$
which means: \emph{``When you dequeue the token \texttt{-} from the
  queue, pop the top element ($y$) and the next element ($x$) from the
  stack, subtract $y$ from $x$, and push the result $x - y$ back onto
  the stack.''} The fact that we write $S$ in the rule above means
that there can be many other integers on the stack that will not be
affected by the operation.

The following table gives all the Clac operations that you will
implement in this assignment.
$$
\begin{array}{lclcll}
\mbox{\bf Token} & & \mbox{\bf Before} & & \mbox{\bf After} & \mbox{\bf Note} \\ \hline
n & : & S & \longrightarrow & S, n & \mbox{for $-2^{31} \leq n < 2^{31}$ in decimal} \\ \hline
\mbox{\texttt{+}} & : & S, x, y & \longrightarrow & S, x\,{+}\,y \\
\mbox{\texttt{-}} & : & S, x, y & \longrightarrow & S, x\,{-}\,y \\
\mbox{\texttt{*}} & : & S, x, y & \longrightarrow & S, x\,{*}\,y \\
\mbox{\texttt{/}} & : & S, x, y & \longrightarrow & S, x\,{/}\,y &
\mbox{Error if $y = 0$ or if $x = -2^{31}$ and $y = -1$}\\
\mbox{\texttt{**}} & : & S, x, y & \longrightarrow & S, x^y &
\mbox{Error if $y < 0$}\\
\end{array}
$$
The addition, subtraction, multiplication, and exponent operations
should give their answer in terms of 32-bit two's complement modular
arithmetic, which makes it easy to implement Clac's math operations in
terms of C0's math operations. Whenever the instructions say that Clac
should raise an error, you should call the function
\lstinline'error()' with an appropriate error message. If a rule
requires two integers to be removed from the stack and fewer than two
integers are presently on the stack, \lstinline'error()' should also
be called.

The \lstinline'error()' function takes a string as its
argument and is built in to C0, like \lstinline'assert()'. User errors
(errors in Clac code) should always cause \lstinline'error()' to be
called; assertions should only be used for programmer errors.

You can use the function \lstinline'parse_int(tok, 10)' to
check whether a token is an integer in base 10. If the function returns
\lstinline'NULL', then the string stored as \lstinline'tok' does not
represent an integer. Otherwise, dereferencing the returned pointer
will give you the integer that the string stored in \lstinline'tok'
represents.

\subsection{Using the Reference Implementation}
\label{sec:clacref}

We can see how our introductory series of instructions affected the
stack and the queue in detail by running the \lstinline'clac-ref' binary
that is available on the shared Unix servers, giving it the extra
argument \lstinline'-trace'. Using this argument, we can see the state of
the stack and the queue after every step.
\begin{quote}
\begin{lstlisting}[language={[clac]C}, columns=fixed]
% clac-ref -trace
Clac top level
clac>> 3 4 * -9 2 / +
                       stack || queue
                             || 3 4 * -9 2 / +
                           3 || 4 * -9 2 / +
                         3 4 || * -9 2 / +
                          12 || -9 2 / +
                       12 -9 || 2 / +
                     12 -9 2 || / +
                       12 -4 || +
                           8 ||
8
\end{lstlisting}
\end{quote}
We can see that \lstinline'8' was printed out at the end because that
was what was at the top of the stack after the queue had been completely
consumed.

The full Clac implementation contains many other tokens for
calculation, manipulating the stack and the queue, and even defining
custom functions. However, the small set of operations above will be
enough for this assignment.


\subsection{Implementation}

\begin{task}[6]
\TAGS{correctness, interface, queue, safety, stack}
In the file \lstinline'clac.c0', you should implement the function
\lstinline'eval':
\begin{quote}
\begin{lstlisting}[numbers=none]
void eval(istack_t S, queue_t Q)
//@requires S != NULL && Q != NULL;
//@ensures queue_empty(Q);
\end{lstlisting}
\end{quote}
\end{task}
Note that the type is \lstinline'istack_t'. In the code handout, you
have both an interface of stacks of integers (\lstinline'istack_t', in
\lstinline'lib/stack_of_int.c0') and stacks of strings
(\lstinline'stack_t', in \lstinline'lib/stack_of_string.c0').

The \lstinline'main' function in file \lstinline'clac-main.c0' takes
lines of input and converts them to a \emph{queue of tokens}. Each
token is just a string.  This part of the Clac implementation has
already been programmed for you, and you are welcome to examine it,
but you should not change this code.

When \lstinline'eval' is first called, the \emph{stack of integers}
$S$ will be empty.  But since the input is processed line-by-line, the
\lstinline'eval' function may also be called with nonempty stacks,
representing the values from prior computations.  The \lstinline'eval'
function should dequeue tokens from the queue $Q$ and process them
according to the Clac definition.  When the queue is empty,
\lstinline'eval' should return leaving the stack in whatever state it
was already in. If \lstinline'error()' is ever called, it doesn't
matter what remains on the stack or on the queue.

It is possible to write a lot of long and confusing code to safely and
efficiently implementing this task, but it is also possible to
use helper functions to write very clear and concise code. Try to
think about how you would structure your code if you knew you were
going to implement full Clac, which has a dozen more tokens. (No
matter how good your code looks, your \lstinline'eval()' function will
probably still look like a loop containing a long series of
\lstinline'if'-\lstinline'else' statements.)

You can reuse code from lecture and the lecture notes, but if you do
that, cite your sources!

\newpage
\section{Dictionaries}

The series of Clac tokens ``\lstinline'3 4 * -9 2 / +''' from
Section~\ref{sec:clacref} performs the computation that we would write
down on paper as $3 \times 4 + -9 \div 2$. Correctly interpreting this
mathematical computation relies on understanding
\emph{precedence} or \emph{order of operations}. Specifically, we
know that multiplication and division need to be done before addition,
because these operations have higher precedence.  The order of
operations is clear when we look at our Clac program: the tokens
\lstinline'*' and \lstinline'/' are earlier in the queue than the
\lstinline'+' token.

In this section, you will implement a \emph{dictionary} that allows
us to take a mathematical operator (represented as a string, like
\lstinline'"**"'), and find its precedence as an integer (say,
\lstinline'2'). When we initially create dictionaries with the
function \lstinline'dict_new()', we populate them with an array of strings,
where each string \lstinline'A[i]' contains a whitespace-separated
list of all the tokens that should be given precedence
\lstinline'i'. Here's an example:
\begin{quote}
\begin{lstlisting}[numbers=none]
string[] A = alloc_array(string, 3);
A[2] = "**";  // '**' has the highest precedence, 2
A[1] = "* /"; // '*' and '/' have the same precedence, 1
A[0] = "+ -"; // '+' and '-' have the same precedence, 0
dict_t D = dict_new(A, 3);
\end{lstlisting}
\end{quote}
If you haven't already done so, you should look at the
\lstinline'num_tokens()' and \lstinline'parse_tokens()' functions from
the C0 library named \lstinline'parse'. The
\lstinline'is_infix_array()' function provided to you in the file
\lstinline'lib/utils.c0' gives an example of using these functions.

%\clearpage
\begin{task}[6]
\TAGS{array, dictionary, safety, search, string}
In the file \lstinline'dict.c0', implement a precedence dictionary
with the following interface:
\begin{quote}
\begin{lstlisting}[numbers=none]
//typedef ______* dict_t;

dict_t dict_new(string[] A, int n)
  /*@requires \length(A) == n; @*/
  /*@requires is_infix_array(A, n); @*/ ;

int dict_lookup(dict_t D, string oper)
  /*@requires D != NULL; @*/
  /*@ensures \result >= -1; @*/ ;
\end{lstlisting}
\end{quote}

The function \lstinline'dict_new(A, n)' must return \lstinline'NULL'
if the same token appears twice in the input. The function
\lstinline'dict_lookup(D, x)' should return \lstinline'-1' if
the string stored as \lstinline'x' is not present in the dictionary
\lstinline'D'. Efficiency is not a primary concern for this task.  See
the file \lstinline'dict-test.c0' for examples.
\end{task}

The implementation is completely up to you, but it should define and
make appropriate use of a thorough specification function
\lstinline'is_dict'.  Feel free to re-use ideas from previous
assignments.  If you use code from previous handouts or lecture, make
sure to cite your sources.

Whatever you do, you should document your design clearly in comments,
write and deploy reasonable data structure invariants, and use contracts
to ensure safety and some degree of correctness for your code.

\newpage
\section{The EXP Language}

Reverse Polish Systems has hired you to build a more user-friendly
language, called EXP, leveraging their existing Clac technology.  The
EXP language has all the mathematical operations that the mini-Clac you
implemented has: exponents, multiplication, division, addition, and
subtraction.  The difference is that EXP allows us to write
expressions as infix. In EXP, we can write
%
``\lstinline'27 / 3**2 - 2**3''' and the result will be $-5$, because
we evaluate the exponents first, getting $27 / 9 - 8$, and then
evaluate the division, getting $3 - 8 = -5$.

\subsection{Parsing EXP Programs}
\label{sec:parse-exp}

Both Clac and EXP use the same process for turning a string into
a queue of tokens.
The EXP program above will be transformed into a queue of
tokens with the following contents:
\begin{quote}
\lstinline'"27"',
\lstinline'"/"',
\lstinline'"3"',
\lstinline'"**"',
\lstinline'"2"',
\lstinline'"-"',
\lstinline'"2"',
\lstinline'"**"',
\lstinline'"3"'
\end{quote}
We won't evaluate EXP programs directly. Directly evaluating infix
expressions that are presented in this way is difficult at
best. Instead, we will approach the problem indirectly: we will use an
algorithm to transforms an EXP program into a Clac program. The example
above will be transformed into the Clac
program
%
``\lstinline'27 3 2 ** / 2 3 ** -'''.  We already have an interpreter
that can evaluate this Clac program, so we can use that existing
interpreter to run our EXP program!

You can see this happening using the \lstinline'exp-ref' binary
that is also available on the shared Unix servers:
\begin{quote}
\begin{lstlisting}[language={[clac]C}]
% exp-ref
EXP top level
EXP>> 3*4 + -9/2
Corresponding Clac program: 3 4 * -9 2 / +
EXP>> -4*2/3 - 7*7
Corresponding Clac program: -4 2 * 3 / 7 7 * -
\end{lstlisting}
\end{quote}

The algorithm for translating infix EXP programs into postfix Clac
programs will need a dictionary in order to determine which
tokens should be treated as infix mathematical operations. We'll
use the word \textbf{operator} to describe
a token that appears in the dictionary (and that therefore has a
known precedence).

In addition to a dictionary, our translation algorithm will use
a stack of operators and two queues. The input queue
contains the EXP program we're reading, and the output queue contains
the Clac program we're writing. Numbers always get moved immediately
from the input queue to the output queue. Operators cannot be
immediately moved to the output queue, though. They are placed on the
operator
stack until it is time to put them onto the output queue.

The algorithm, named the \emph{shunting-yard algorithm} by Edsger
Dijkstra,\footnote{The algorithm is also described at
  \url{https://en.wikipedia.org/wiki/Shunting-yard_algorithm}. You're
  welcome to read the Wikipedia article, but don't refer to other
  sources.} has two phases: we start in the integer phase, and switch
phases every time we remove something from the input queue.

\begin{description}
\item[Integer phase:] There must be a token on the input queue, and it
  must be an integer. Put that token on the
  output queue and switch to the \textbf{operator} phase.

\item[Operator phase: ] In this phase, you will sometimes pop
  operators off of the stack. \emph{Whenever you pop an operator off
    the stack, you should add it immediately to the output queue.}

  If there is a token \emph{tok} on the input queue, it must be an
  operator (it must be in the precedence dictionary). Pop just enough
  operators off of the stack so that the operator on the top of the
  stack (if there is one) has precedence \emph{strictly less} than
  \emph{tok}'s precedence. (If the operator stack is initially
  empty, or if the operator on the top of the stack already has a
  lower precedence, then there's nothing to do.) Finally, push
  \emph{tok} on the stack and switch to the \textbf{integer} phase.

  If there are no tokens on the input queue,
  then you are done. Pop every operator off the stack (and
  put it on the output queue).
\end{description}

%\clearpage
\paragraph{Example}
To give a few examples of how the operator phase works, imagine that the
operator stack initially contains the following three operators, with
\lstinline'"**"' at the top of the stack:
\begin{quote}
\lstinline'"+"' (precedence 0), \lstinline'"/"' (precedence 1),
\lstinline'"**"' (precedence 2)
\end{quote}
The behavior of the operator phase depends on what's on the
queue. Here are three possible outcomes:
\begin{itemize}
\item
If the input queue is empty, then enqueue \lstinline'"**"', then
\lstinline'"/"', and then \lstinline'"+"' onto the output queue in
that order. Return the output queue.

\item If the token at the front of the input queue is \lstinline'"42"', then
  the input was not a valid EXP program.  Return \lstinline'NULL'.  The same
  happens when passed an invalid token like \lstinline'"x"' or \lstinline'"@"'.

\item
If the token at the front of the input queue is \lstinline'"*"' and
the dictionary gave this token precedence 1, first pop
\lstinline'"**"' off the stack and add it to the output queue. Then
pop \lstinline'"/"' off the stack and add it to the output queue. Now
the precedence of the token on the top of the stack is strictly less
than 1, so push \lstinline'"*"' onto the stack. Switch
to the integer phase and repeat.
\end{itemize}

\paragraph{Associativity}

This version of the shunting-yard algorithm results in all operators
being treated as \emph{left-associative}. We definitely want to
interpret subtraction and division as left associative. As an example,
consider the expression $1 - 2 - 3$. This is always evaluated the same
way as $(1 - 2) - 3$, or ``\lstinline'1 2 - 3 -''' in Clac; it would
be incorrect to read it as $1 - (2 - 3)$, or \lstinline'1 2 3 - -' in
Clac. However, the exponentiation operator is often treated as
\emph{right associative}. (Python does this, for instance.)  EXP
treats exponents as left associative to make your implementation
simpler.

\subsection{Invariants}
Before you start implementing this algorithm, think about the
invariants that are at work! Operationally, before you add a new
operator to the input stack, you remove items from the stack until the
new operator has a higher precedence than the operator on the top of the
stack. We can write a specification function that captures this
operational behavior as an invariant.

\begin{task}[2]
\TAGS{dictionary, ds-invariant, interface, stack}
In a new file \lstinline'parse.c0', write a function which checks that
a stack \lstinline'S' contains valid operators with strictly
increasing precedence. The highest-precedence operation must be on the
top of the stack.
\begin{quote}
\begin{lstlisting}[numbers=none]
bool is_precstack(dict_t D, stack_t S)
//@requires D != NULL && S != NULL;
\end{lstlisting}
\end{quote}
When the function returns, the contents of the stack should be the
same as they were when the function was called.
\end{task}

\paragraph{Important:}

Even though \lstinline'is_precstack(D, S)' will be an important loop
invariant of our algorithm, we are not usually allowed to write it in
a \lstinline'//@loop_invariant' contract! This is because C0 has
checks to make sure that programs will always run the same way with or
without contracts enabled.

The function \lstinline'is_precstack()' must actually modify the stack
\lstinline'S' in order to respect the stack interface. The C0 compiler
assumes that calling a function that modifies the stack in any way could
cause the program to behave differently when contracts are
enabled. Even though your function returns the stack to its original
condition, the C0 compiler cannot prove this and therefore refuses to
allow \lstinline'is_precstack()' in a contract.

If you want to use this function in a loop invariant to help you debug
your code, you can call the compiler with the extra argument
\lstinline'--no-purity-check'. We'll compile your code with this
option on Autolab --- but that means you have to be extra careful that
your \lstinline'is_precstack()' function returns the stack to its
original state!


\newpage
\subsection{Implementation}

\begin{task}[7]
\TAGS{correctness, dictionary, interface, queue, safety, stack}
In the file \lstinline'parse.c0', implement the algorithm described in
Section~\ref{sec:parse-exp}:
\begin{quote}
\begin{lstlisting}[numbers=none]
queue_t parse(dict_t D, queue_t input)
//@requires D != NULL && input != NULL;
//@ensures \result == NULL || queue_empty(input);
\end{lstlisting}
\end{quote}
The function should return \lstinline'NULL' if the input is not a
well-formed EXP program according to the operators in the dictionary
\lstinline'D'. If the function returns \lstinline'NULL' it doesn't
matter what is left on the input queue.
\end{task}

The \lstinline'exp-main.c0' file parses programs based on the
following precedence table:
\begin{center}
\begin{tabular}{rl}
   1. & \lstinline'||'
\\ 2. & \lstinline'&&'
\\ 3. & \lstinline'<', \lstinline'>', \lstinline'==', \lstinline'!='
\\ 4. & \lstinline'<<', \lstinline'>>'
\\ 5. & \lstinline'+', \lstinline'-'
\\ 6. & \lstinline'*', \lstinline'/', \lstinline'%'
\\ 7. & \lstinline'**'
\end{tabular}
\end{center}
If you run your \lstinline'exp' program with the \lstinline'-x'
option, it will attempt to run the parsed EXP program with your Clac
implementation.
\begin{quote}
\begin{lstlisting}[language={[clac]C}]
% ./exp -x
EXP top level
EXP>> 3 * 4 + -9 / 2
Corresponding Clac program: 3 4 * -9 2 / +
Result: 8
\end{lstlisting}
\end{quote}
This will only work if you restrict yourself to EXP programs using
\lstinline'**', \lstinline'*', \lstinline'/', \lstinline'+', or
\lstinline'-'.  Otherwise, your Clac implementation will crash when it
finds a token like \lstinline'<' that it doesn't know how to process.

\subsection{Extending EXP using Clac}

For the last part of this assignment, you will write some code in
Clac, instead of writing code in C0. You will need to use the full
Clac implementation described in Figure~\ref{fig:allclac}, which is
implemented by the \lstinline'clac-ref' program.

The code you write will be in the form of Clac \textit{definitions},
which you have already encountered in the lab activity on Clac. For
example, the Clac definition ``\lstinline': dup 1 pick ;''' defines a
new token \lstinline'dup' that takes whatever is at the top of the
stack and duplicates it.  Therefore, the user-defined \lstinline'dup'
token implements the rule $S, x \longrightarrow S, x, x$.

\begin{figure}
\fbox{%
\parbox{\linewidth}{%
$$
\begin{array}{lclcll}
\mbox{\bf Token} & & \mbox{\bf Before} & & \mbox{\bf After} & \mbox{\bf Note} \\ \hline
\mbox{\lstinline'>'} & : & S, x, y & \longrightarrow & S, 1 &
     \mbox{If $x > y$}\\
\mbox{\lstinline'>'} & : & S, x, y & \longrightarrow & S, 0 &
     \mbox{If $x \leq y$}\\
\mbox{\lstinline'=='} & : & S, x, y & \longrightarrow & S, 1 &
     \mbox{If $x = y$}\\
\mbox{\lstinline'=='} & : & S, x, y & \longrightarrow & S, 0 &
     \mbox{If $x \neq y$}\\
\mbox{\lstinline'!='} & : & S, x, y & \longrightarrow & S, 1 &
     \mbox{If $x \neq y$}\\
\mbox{\lstinline'!='} & : & S, x, y & \longrightarrow & S, 0 &
     \mbox{If $x = y$}\\
\mbox{\lstinline'||'} & : & S, x, y & \longrightarrow & S, 1 &
     \mbox{If either $x \neq 0$ or $y \neq 0$}\\
\mbox{\lstinline'||'} & : & S, x, y & \longrightarrow & S, 0 &
     \mbox{If both $x$ and $y$ are 0}\\
\mbox{\lstinline'\&\&'} & : & S, x, y & \longrightarrow & S, 1 &
     \mbox{If both $x \neq 0$ and $y \neq 0$}\\
\mbox{\lstinline'\&\&'} & : & S, x, y & \longrightarrow & S, 0 &
     \mbox{If either $x$ or $y$ is 0}\\
\mbox{\lstinline'\<\<'} & : & S, x, y & \longrightarrow & S, x \times 2^y &
     \mbox{If $0 \leq y < 32$}\\
\mbox{\lstinline'\>\>'} & : & S, x, y & \longrightarrow & S, x / 2^y &
     \mbox{If $x \geq 0$ and $ 0 \leq y < 32$}
%\\
%\mbox{\lstinline{\%}} & : & S, x, y & \longrightarrow & S, x\,{\%}\,y &
%     \mbox{Error if $y = 0$ or if $x = -2^{31}$ and $y = -1$}\\
\end{array}
$$
}}

\caption{Clac functions to implement in Task 5}
\label{fig:newclac}
\end{figure}


\begin{task}[4]
\TAGS{application, compilation}
  In a file \lstinline'exp-defs.clac', write Clac definitions that
  implement the Clac tokens described in Figure~\ref{fig:newclac}.
  The file already includes definitions of \lstinline'>',
  \lstinline'==', and \lstinline'!=' to get you started.
\end{task}

\begin{ectask} (Extra challenge 1) %
\TAGS{application, compilation}
  Make the shift operators always raise an error when $y < 0$ or $y
  \geq 32$.
\end{ectask}

\begin{ectask} (Extra challenge 2) %
\TAGS{application, compilation}
  Make the right-shift operator perform division that rounds towards
  negative infinity, so that it works correctly on negative $x$ as
  well as nonnegative $x$.
\end{ectask}

With your definitions, you can parse any valid EXP program into Clac,
and then run it with the Clac reference interpreter:

\begin{quote}
\begin{lstlisting}[language={[clac]C}]
% ./exp
EXP top level
EXP>> 3 * 4 > -9 / 2 && 3 << 2 == 24 >> 2 - 1
Corresponding Clac program: 3 4 * -9 2 / > 3 2 << 24 2 1 - >> == &&
% clac-ref exp-defs.clac
clac>> 3 4 * -9 2 / > 3 2 << 24 2 1 - >> == &&
1
\end{lstlisting}
\end{quote}


\begin{figure}[p]
\fbox{%
\parbox{\linewidth}{%
\newcommand{\mmid}{\mid\mid}
\newcommand{\cc}[1]{\mbox{\texttt{#1}}}
\newcommand{\tok}{\mathit{tok}}
$$
\begin{array}{@{}r@{\;\;\mmid\;\;}l@{\;\;\longrightarrow\;\;}r@{\;\;\mmid\;\;}ll@{}}
   \multicolumn{2}{c}{\bf Before\hspace*{5.8em}}
 & \multicolumn{2}{c}{\bf \hspace*{3.0em}After}
\\ \multicolumn{1}{r@{\;\;\;\;\;\;}}{\mbox{\bf Stack}} & \mbox{\bf Queue}
 & \multicolumn{1}{r@{\;\;\;\;\;\;}}{\mbox{\bf Stack}} & \mbox{\bf Queue}
 & \mbox{\bf Cond/Effect} \\ \hline
%%%%
%%%% Already implemented
%%%%
   S                        & n, Q
 & S, n                     & Q
\\ S, n                     & \cc{print}, Q
 & S                        & Q
 & \mbox{See note \#1}
\\ S                        & \cc{quit}, Q
 & S                        & Q
 & \mbox{See note \#2}
\\ \hline
%%%%
%%%% Arith operations
%%%%
   S, x, y                  & \cc{+}, Q
 & S, x\,{+}\,y             & Q
 & \mbox{See note \#3}
\\ S, x, y                  & \cc{-}, Q
 & S, x\,{-}\,y             & Q
 & \mbox{See note \#3}
\\ S, x, y                  & \cc{*}, Q
 & S, x \times y            & Q
 & \mbox{See note \#3}
\\ S, x, y                  & \cc{/}, Q
 & S, x\,{/}\,y             & Q
 & \mbox{See note \#4}
\\ S, x, y                  & \cc{\%}, Q
 & S, x\,{\%}\,y            & Q
 & \mbox{See note \#4}
\\ S, x, y                  & \cc{**}, Q
 & S, x^y                   & Q
 & \mbox{See \#3, \#4}
\\ S, x, y                  & \cc{<}, Q
 & S, 1                     & Q
 & \mbox{if $x < y$}
\\ S, x, y                  & \cc{<}, Q
 & S, 0                     & Q
 & \mbox{if $x \geq y$}
\\ \hline
%%%%
%%%% Stack and queue manipulations
%%%%
   S, x                     & \cc{drop}, Q
 & S                        & Q
\\ S, x, y                  & \cc{swap}, Q
 & S, y, x                  & Q
\\ S, x, y, z               & \cc{rot}, Q
 & S, y, z, x               & Q
\\
%   S, x         & \cc{dup}, Q
% & S, x, x      & Q
%\\
%   S, n         & \cc{skip}, \tok_1, \ldots, \tok_n, Q
% & S            & Q
% & \mbox{$n \geq 0$}
%\\
   S, x                     & \cc{if}, Q
 & S                        & Q
 & \mbox{if $x \neq 0$}
\\ S, x                     & \cc{if}, \tok_1, \tok_2, \tok_3, Q
 & S                        & Q
 & \mbox{if $x = 0$}
\\ S, x_n, \ldots, x_1, n   & \cc{pick}, Q
 & S, x_n, \ldots, x_1, x_n & Q
 & \mbox{See note \#5}
\\ S, n                     & \cc{skip}, \tok_1, \ldots, \tok_n, Q
 & S                        & Q
 & \mbox{See note \#5}
\end{array}
$$
\bgroup
\small\em

Clac should raise an error whenever there are not enough tokens on the
stack or the queue for an operation to be performed, or whenever the
token on the top of the queue is not one of the ones listed
above. Tokens are case sensitive, so \lstinline'Print' and \lstinline'PRINT' are
not defined, though \lstinline'print' is.

\medskip

Notes:
\begin{enumerate}
\item%
  The \lstinline'print' token causes $n$ to be printed, followed by a
  newline.
\item%
  The \lstinline'quit' token causes the interpreter to stop. The
  \lstinline'eval' function should then return \lstinline'false' to
  indicate that we should just stop, rather than asking for more
  input.
\item%
  This is a 32 bit, two's complement language, so addition,
  subtraction, multiplication, and exponentiation should behave just
  as in C0 without raising any overflow errors.
\item%
  Division or modulus by 0, or division/modulus of
  \lstinline'int_min()' by -1, which would result in an arithmetic
  error according to the definition of C0 (see page 3 of the
  \href{http://c0.typesafety.net/doc/c0-reference.pdf}{C0 Reference}),
  should raise an error in Clac. Negative exponents are undefined and
  should also raise an error.
\item%
  The \lstinline'pick' token should raise an error if $n$, the value
  on the top of the stack, is not strictly positive. The
  \lstinline'skip' token should raise an error if $n$ is negative; 0
  is acceptable.
\end{enumerate}
\egroup
}}

\caption{Clac reference}
\label{fig:allclac}
\end{figure}



\end{document}
