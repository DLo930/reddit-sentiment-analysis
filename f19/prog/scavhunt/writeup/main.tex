\documentclass[12pt]{exam}
\newcommand{\hwnumber}{1}
\newcommand{\hwname}{Scavenger hunt}
\newcommand{\duedate}{\formatdate{5}{9}{\YEAR} by \progDueTime} % day-month-year

\usepackage{import}  % Like input and include, but respects subdirectories

% copied from palatino.sty, except left ttdefault as CMTT
\renewcommand{\rmdefault}{ppl}
\renewcommand{\sfdefault}{phv}
% \renewcommand{\ttdefault}{pcr}
% end palatino.sty

\usepackage{amsmath,amsthm,amssymb}
%\usepackage{enumerate}  Use enumitem, diderot doesn't like enumerate
\usepackage{enumitem}  % control enumerate/itemized lists, used
\usepackage{etoolbox} % Subsumes ifthen
\usepackage{graphicx}
\usepackage{comment}
\usepackage{makeidx}
\usepackage{minitoc}

\usepackage[breaklinks=true,
  colorlinks=true,
  citecolor=blue,
  linkcolor=blue,
  urlcolor=blue]{hyperref}

\usepackage{fancyhdr}
\pagestyle{fancyplain}
\setlength{\headheight}{14pt}
\addtolength{\oddsidemargin}{30pt}
\addtolength{\evensidemargin}{-22pt}
\renewcommand{\headrulewidth}{0.4pt}
\newcommand{\gototoc}{\quad\hyperlink{toc}{\phantom{$\Uparrow$}}\quad}
\lhead[\thepage\gototoc]{\fancyplain{}{\leftmark}}
\chead{}
\rhead[\rightmark]{\fancyplain{}{\gototoc\thepage}}
\rfoot{}
\cfoot{\fancyplain{\thepage}{\em Draft of \today}}
\lfoot{}


% Directories containing code and pictures
\newcommand{\code}{code}
\newcommand{\img}{img}

%%%%%%%%%%%%%%%%%%%%%%%%%%%%%%%%%%%%%%%%%%%%%%%%%%%%%%%%%%%%%%%%%%%%%%
% Environments
%%%%%%%%%%%%%%%%%%%%%%%%%%%%%%%%%%%%%%%%%%%%%%%%%%%%%%%%%%%%%%%%%%%%%%

\newtheorem{exercise}{Exercise}[chapter]
\newtheorem{example}{Example}
\newtheorem{theorem}{Theorem}[chapter]
\newtheorem{lemma}[theorem]{Lemma}
\newtheorem{definition}[theorem]{Definition}

\newenvironment{Exercise}[1]
{\begin{exercise}[sample solution on page~\pageref{#1-solved}]%
    \label{#1}}
{\end{exercise}\vspace{-0.5ex}}

\newenvironment{Solution}[1]
{\par\noindent
  \textbf{Solution of exercise~\ref{#1}.}~\bgroup\itshape
  \label{#1-solved}%
}
{\egroup\par\bigskip}

\renewenvironment{proof}{\trivlist \item[\hskip \labelsep{\bf
Proof:}]}{\hfill$\Box$ \endtrivlist}
\newenvironment{attempt}{\trivlist \item[\hskip \labelsep{\bf
Proof attempt:}]}{\hfill$\Diamond$ \endtrivlist}

\newcommand{\ednote}[1]{\footnote{\it #1}\message{ednote!}}
\newenvironment{note}{\begin{quote}\message{note!}\it}{\end{quote}}
\newcommand{\highlight}[1]{\par%\vspace{1ex}%
\framebox{\addtolength{\linewidth}{-1em}\begin{minipage}{\linewidth}#1\end{minipage}}}

\newcommand{\m}[1]{\mathsf{#1}}

%% Displays tags in 'answer' mode
\providecommand{\ismeta}{false}
\newcommand{\TAGS}[2][0ex]
{\ifbool{\ismeta}%
  {%\rule{0em}{0ex}%
   \marginpar{\vspace*{-5.5ex}\footnotesize%
   \raisebox{#1}{%
     \fcolorbox{black}{Gray!25}{%
       \parbox[t]{2cm}{\raggedright\textbf{TAGS:}\\#2}}}}%
   \ignorespaces}%
  {}}%
\newcommand{\exerciseTAGS}[1]{\TAGS[-10ex]{#1}}
\newcommand{\chapterTAGS}[1]
{\ifbool{\ismeta}%
  {\chead{\fancyplain{%
   \fcolorbox{black}{Gray!25}{%
     \parbox[t]{0.985\linewidth}{\raggedright\textbf{TAGS:} #1}}%
   }}}%
  {}}%

%% Macros for rendering C0, C1, C, C0VM listings
\usepackage{../../misc/latex/c0}


% Must be loaded after listings
\usepackage{tikz}


\setcounter{tocdepth}{1}
\newenvironment{preface}
 {\rhead[\sc Preface]{\fancyplain{}{\thepage}}%
  \clearpage%
  \vspace*{2cm}%
  \noindent%
  {\Huge \textbf{Preface}}%
  \par%
  \vspace*{1.5cm}%
  \noindent}
 {}

%% Enter chapter heading in list of figures -- only if there are figures
\newcommand{\currentChapter}{}
\newcommand{\lofchapter}
 {\addtocontents{lof}{\noindent{\bfseries\chaptername~\thechapter:~~\currentChapter}}}

%% Is this the whole book or individual chapters?
\newcommand{\main}{true}
\newcommand{\ifmain}[2]{\ifdefstring{\main}{true}{#1}{#2}}

%% Macros to undefine if this is the main
\newenvironment{abstract}{}{\medskip}


% \ifcsname all\endcsname%
%  \relax
% \else
%  \typeout{}
%  \typein[\all]{What to compile? <enter> for everything, or chapter/chapter_name}
% \fi

% \newcommand{\ifall}[2]{\ifdefstring{\all}{}{#1}{#2}}

% \ifall{}
% {\renewcommand{\lectitle}[1]{\chapter{#1}}
% \renewcommand{\date}[1]{}
% \includeonly{\all}}

% \ifall{
\makeindex
% }{}

\begin{document}
\hwTitle

\noindent
Welcome to 15-122!  This first programming homework is designed
as an opportunity for you to tour the course tools and
workflow that we will use in 15-122.

The assignment takes the form of a scavenger hunt; you will add pieces
of code to a file, \lstinline'scavhunt.c0', and then turn this file in
through \autolab. There are two ways to do so. From the terminal on
Andrew Linux (via cluster or ssh) type:
\begin{quote}
\begin{lstlisting}[language={[coin]C}]
% handin scavhunt scavhunt.c0 puzzle.c0
\end{lstlisting}
\end{quote}
Your score will then be available on the \autolab{} website.
Your submission can also be made through the web interface of \autolab.
To do so, please create a zipped tarball (\lstinline'tgz'), for example:
\begin{quote}
\begin{lstlisting}[language={[coin]C}]
% tar -czvf handin.tgz scavhunt.c0 puzzle.c0
\end{lstlisting}
\end{quote}
and then upload it to \autolab.


\clearpage
\section{Obtaining the handout code}

\vspace{-\bigskipamount}
\begin{task}[2]
\TAGS{interpreter, unix}
 Obtain the handout file \lstinline'scavhunt.c0',
containing a function \lstinline'greet'.
\end{task}

The handout code for this assignment is a gzipped tarball file at
\begin{center}
  \whereisthetgz{scavhunt-handout.tgz}
\end{center}
Download it and unpack it.  \textbf{IMPORTANT:} if you are using
Andrew AFS, make sure that you only put code for 15-122 into your
\lstinline'private' directory or into another directory with
appropriate AFS permissions set. Failing to do this will cause you to
run afoul of the course's academic integrity policy.

\begin{quote}
\begin{lstlisting}[language={[coin]C},mathescape=false]
% cd $HOME/private/15122
% tar xzvf scavhunt-handout.tgz
scavhunt-handout/
scavhunt-handout/puzzle.c0
scavhunt-handout/puzzle-test.c0
scavhunt-handout/scavhunt-main.c0
scavhunt-handout/scavhunt.c0
% cd scavhunt-handout
% ls -la
drwxr-xr-x 2 iliano users 2048 Aug 16 17:30 .
drwxr-xr-x 3 iliano users 2048 Aug 16 17:48 ..
-rw-r--r-- 1 iliano users 1587 May 16  2015 puzzle.c0
-rw-r--r-- 1 iliano users 1155 Jan 15  2015 puzzle-test.c0
-rw-r--r-- 1 iliano users  405 Jan 12  2015 scavhunt.c0
-rw-r--r-- 1 iliano users  196 Sep  1  2014 scavhunt-main.c0
\end{lstlisting}
\end{quote}

You can't use \lstinline'scavhunt-main.c0' yet, but you can test the
\lstinline'greet' function contained in \lstinline'scavhunt.c0' using
\lstinline'coin'.

\begin{quote}
\begin{lstlisting}[language={[coin]C},lastline=13]
% coin -d scavhunt.c0
C0 interpreter (coin) 0.3.3 'Nickel' (r590, Mon Aug 29 12:04:13 UTC 2016)
Type `#help' for help or `#quit' to exit.
--> greet("Hello", "world");
"Hello, world!" (string)
--> println(greet("Hello", "world"));
Hello, world!
(void)
--> print(greet("Hi", "friend"));
(void)
--> print("\n");
Hi, friend!
(void)
--> #quit
\end{lstlisting}
\end{quote}

This demonstrates something important (and potentially confusing)
about C0 programs. When you use the \lstinline'print' function to
print strings, the output is \emph{buffered} and does not generally
get printed out until a newline is printed. The escape sequence
\lstinline'\n' represents a newline, so printing the string
\lstinline'"\n"' empties the buffer, printing \lstinline'Hi, friend!'.


\section{Using the C0 tutorial}

\vspace{-\bigskipamount}
\begin{task}[2]
\TAGS{syntax, unix}
The C0 tutorial's page on ``Statements'' contains the code for a C0
file named \lstinline'fact.c0'. Copy all the contents of this file (it's
just one function) into \lstinline'scavhunt.c0'.
\end{task}

The C0 tutorial at \url{http://c0.typesafety.net/} will help with
early assignments. The point about buffered output above is explained
on the page ``\href{http://c0.typesafety.net/tutorial/Debugging-C0-Programs.html}{Debugging C0 Programs}'' in the C0 tutorial.


\section{Viewing images from AFS}

\vspace{-\bigskipamount}
\begin{task}[2]
\TAGS{compilation, unix}
The file
\lstinline'/afs/andrew/course/15/122/misc/scavhunt/snippet.png' is an image
file that, when viewed, contains a C0 function. Copy the code in that image
into \lstinline'scavhunt.c0'.
\end{task}

For the next two programming assignments, it will be very helpful for
you to already know how to view images that live on AFS on your own
computer.  The PNG file above, which is publicly readable when you're
logged into a Linux cluster machine or connected with SSH to
\lstinline'linux.andrew.cmu.edu', can be viewed using a program like
\lstinline'display'.
%, \lstinline'gpicview', \lstinline'qiv', \lstinline'eog',
% or \lstinline'gthumb'.
To use one of these programs, you will either need to be on a Linux
cluster or you will need to use \lstinline'ssh -X' to log on to
\lstinline'linux.andrew.cmu.edu' with X11 forwarding (if this doesn't
work, try \lstinline'ssh -Y').  You can also copy the file by
transferring it from AFS to your computer with the \lstinline'scp'
command-line program or with a program like WinSCP and viewing the
image with whatever built-in image-viewing software your operating
system uses.  Play around and find a method you like.

Once you have added the function from \lstinline'snippet.png', you can use
the \lstinline'cc0' compiler and the provided \lstinline'scavhunt-main.c0'
program to compile and run your scavenger hunt code.

\begin{quote}
\begin{lstlisting}[language={[coin]C}]
% cc0 -d -o scavhunt scavhunt.c0 scavhunt-main.c0
% ./scavhunt
\end{lstlisting}
\end{quote}

The \lstinline'-d', which also appeared in the call to
\lstinline'coin', makes sure contracts are checked. The arguments
\lstinline'-o scavhunt' tells the compiler to produce an executable
file named \lstinline'scavhunt'. You could omit this argument and the
executable file would be named \lstinline'a.out'.

\begin{quote}
\begin{lstlisting}[language={[coin]C}]
% cc0 -d scavhunt.c0 scavhunt-main.c0
% ./a.out
\end{lstlisting}
\end{quote}


\section{Updates, clarifications and questions on \qatoolName}

\vspace{-\bigskipamount}
\begin{task}[3]
\TAGS{compilation}
Modify the function from \lstinline'snippet.png' as described in the
``How to use \qatoolName{} (effectively)'' post on 15-122 \qatoolName.
\end{task}

\qatoolName{} can be found at \url{\qatoolBaseURL}. Please read
through the whole ``How to use \qatoolName{} (effectively)'' post, as
it explains some of the guidelines for how we will be using
\qatoolName.


\section{Getting feedback from Autolab}

To proceed, you'll need to \textbf{submit the work you have so far to
  Autolab}. The instructions for doing this are on the first page of
this writeup. Autolab always allows multiple submissions, and your
\textbf{last submission} is the one we count.  For this assignment,
there is no limit to the number of times you are allowed to submit (but
there will be in future homework).

\begin{task}[4]
\TAGS{testing}
Add a function (that takes no arguments and returns a string) to
\lstinline'scavhunt.c0'. It does not matter what string this function
returns.
\end{task}

You'll be able to figure out what the name of the missing function
should be by looking at the output that Autolab gives you.  The
autograder we use for 15-122 doesn't give a ton of feedback, but it
does give some feedback when tests fail. First, it says why the test
failed (for instance, \lstinline'File did not compile'), and then it gives
a hint. These hints aren't perfect, and they can sometimes be
unintentionally misleading. (Usually, this is because of a mistake
that we didn't anticipate, or because an error in an earlier part of
the assignment triggered another error later on.)

Also, the hints almost always assume that the test compiled, so if the
test reports that the file did not compile, you should look to the
first part of the autograder's input to see the output from the C0
compiler and get an idea of what went wrong, rather than paying
attention to the hint. You'll need to do this to learn the name of the
mystery function needed to complete this assignment. \emph{Don't make
  a habit of doing things this way.} Autolab is not your compiler, and
you should usually test your code before you hand in your work.  In
fact, future programming assignments will limit the number of
submissions you are allowed.

There are two other critical features in Autolab. First, the
\emph{Gradebook} link lets you see your performance, including the
number of late days you have used. Second, you can view the code you
handed in with Autolab; when there's a manually-graded piece of a
programming assignment you'll get comments from TAs on Autolab.


\newpage
\section{Puzzle hunt}

This last task will require you to think a little bit harder, and also
read through the C0 tutorial page on Strings and Characters at
\url{http://c0.typesafety.net/}, as well as the C0 library
documentation. You must implement three functions:
\begin{lstlisting}[basicstyle=\smallbasicstyle]
int common_prefix(string s1, string s2)
//@ensures 0 <= \result && \result <= string_length(s1);
//@ensures 0 <= \result && \result <= string_length(s2);
//@ensures string_equal(string_sub(s1, 0, \result), string_sub(s2, 0, \result));
/*@ensures \result == string_length(s1)
  @     || \result == string_length(s2)
  @     || string_charat(s1, \result) != string_charat(s2, \result); @*/

int find_nonrepeating_position(string s)
//@requires string_length(s) > 0;
//@requires string_charat(s, 0) != string_charat(s, string_length(s) - 1);
//@ensures 0 <= \result && \result < string_length(s)-1;
//@ensures string_charat(s, \result) == string_charat(s, 0);
//@ensures string_charat(s, \result+1) != string_charat(s, 0);

string same_start_end(string s)
//@ensures is_substring(\result, s);
/*@ensures string_length(s) < 128
  @     || (string_length(\result) > 1
  @         && string_charat(\result, 0)
  @            == string_charat(\result, string_length(\result) - 1)); @*/
\end{lstlisting}
Note: some of these postconditions rely on \emph{short-circuiting
  evaluation}, which is discussed in the C0 tutorial page on Booleans
and in the first recitation notes.

It is possible to write loop invariants to prove that each function
satisfies its postcondition. You aren't required to do that for this
assignment, but you are encouraged to try!
You will definitely want to write test cases for these functions and
run them; it's a good idea to enable contract checking when you do so:
\begin{quote}
\begin{lstlisting}[language={[coin]C}]
% cc0 -d -o puzzle puzzle.c0 puzzle-test.c0
% ./puzzle
\end{lstlisting}
\end{quote}
For the first function, some test cases are provided. For the second
function, the autograder's feedback will indicate what string your
function failed on. For the third function, which involves finding two
characters in the string that are the same, you'll need to think
through test cases on your own. You might want to look up the
\emph{pigeonhole principle} to understand why the postcondition for
that function looks like it does. (Hint: there are 127 different
characters that can appear in C0 strings.)

\begin{task}[12]
\TAGS{correctness, safety, syntax}
Fill in the three functions in \lstinline'puzzle.c0'. Any implementation
that always satisfies the postconditions will be accepted, even though
there may be multiple correct implementations that give different
answers.
\end{task}

\end{document}
