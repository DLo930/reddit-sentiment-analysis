\section*{Deletion of the highest-priority element from a heap%
\TAGS{pq, safety}}

You may need the following functions: \lstinline'is_heap_safe',
\lstinline'ok_above'
\begin{lstlisting}[numbers=left]
void sift_down(heap* H)
//@requires [*\answerline{is\_heap\_safe(H) \&\& H->next > 1 \&\& is\_heap\_except\_down(H, 1)}*];
//@ensures is_heap(H);
{
  int i = 1;
  while ([*\answerline{2*i < H->next}*])
  //@loop_invariant 1 <= i && i < H->next;
  //@loop_invariant is_heap_except_down(H, i);
  //@loop_invariant grandparent_check(H, i);
  {
    int left = 2*i;
    int right = left+1;
    if ([*\answerline{ok\_above(H, i, left) \&\& (right >= H->next || ok\_above(H, i, right))}*])
      return;
    if ([*\answerline{right >= H->next || ok\_above(H, left, right)}*]) {
      swap_up(H, left);
      i = left;
    } else {
      //@assert [*\shortanswerline{right < H->next \&\& ok\_above(H, right, left)\hspace{-4.6em}}*];
      swap_up(H, right);
      i = right;
    }
  }
}

elem pq_rem(heap* H)
//@requires is_heap(H) && !pq_empty(H);
//@ensures is_heap(H);
{
  elem min = H->data[1];
  (H->next)--;
  if (H->next > 1) {
    H->data[1] = H->data[H->next];
    sift_down(H);
  }
  return min;
}
\end{lstlisting}


\checkpoint*{\TAGS{correctness, loop-invariant}}

\begin{enumerate}
\item%
  Check that the preconditions imply the loop invariants hold
  initially, and that they are satisfied when \lstinline'sift_down' is
  called from \lstinline'pq_rem'.

\begin{solution}
  Because initially i is 1 (line 5), %
  \lstinline'1 <= i' %
  is trivial and %
  \lstinline'i < H->next' %
  follows from 2.b %
  (\lstinline'H->next > 1').

  The second loop invariant similarly follows from 5 and 2.c.

  The grandparent check is trivally true for i = 1.
\end{solution}

\item%
  Show that the grandparent check is necessary as a loop invariant.

\begin{solution} Counterexample, 4 is being inserted
\begin{verbatim}
       3          3
      /          / <-- violation, precluded by grandparent check
     4     =>   2
    ? ?        / \
   2   7      4   7
  / \        ? ?
 3   5      3   5

\end{verbatim}
\end{solution}

\item%
  Prove that the loop invariants imply the postcondition for the
  return on line 14 and on line 23.

\begin{solution}
  The return on line 14 is straightforward: the check on line 13
  checks all the places that aren't checked by the loop invariant on
  line 8, so line 8 + line 13 = \lstinline'is_heap'.

  The negation of the loop guard (line 6) and the loop invariant on
  line 7 imply that \lstinline'i' is a leaf, which means that
  \lstinline'is_heap_except_down' is equivalent to
  \lstinline'is_heap'.
\end{solution}
\end{enumerate}
