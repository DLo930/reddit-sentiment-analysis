\section*{\lstinline'structs' on the stack%
\TAGS{c-memory, struct}}

In C0 and C1, if we ever wanted to create a \lstinline'struct', we had
to explicitly allocate memory for it using
\lstinline'alloc'. C doesn't have this restriction ---
you can declare \lstinline'struct' variables on the stack, just like
\lstinline'int''s.  We set a field of a \lstinline'struct' with
dot-notation, below. Recall that when we had a \emph{pointer} \lstinline'p' to a
\lstinline'struct', we accessed its fields with
\lstinline'p->data'. This is just syntactic sugar for
\lstinline'(*p).data'.
\checkpoint*{\TAGS{c-memory, struct}}
\bgroup
\smalllistings
\lstinputlisting{\code/structs.c}
\egroup

\begin{solution}
This prints out:\\
\lstinline'a.x, a.y: 3, c'\\
\lstinline'b.x, b.y: 4, c'

In C, struct assignment does a shallow copy of the entire struct. Of course,
we're more concerned that the students understand how to use the dot notation of
structs in this problem.
\end{solution}
