\section*{Converting between binary and decimal%
\TAGS{ints}}

\bgroup
\newcommand{\base}[1]{\textcolor{Brown}{#1}}
\newcommand{\bin}[1]{\textcolor{blue}{#1}}
\newcommand{\dec}[1]{\textcolor{red}{#1}}
\newcommand{\blankSize}{2.5em}
\newcommand{\blank}[1][]{\underline{\makebox[\blankSize][r]{#1\;\;}}}
\newcommand{\blankrow}[3]{%
  \blank[#1] \,\times\, \base{2} \,+\, \blank[\bin{#2}] \;=\; \blank[#3]}

To easily convert a number represented in binary notation, such as
$\bin{10100}_{[\base{2}]}$, we can employ \emph{Horner's algorithm}. At
each step, we multiply the previous result by \base{2}, and add the next
bit in the number. To convert in the other direction, we divide by \base{2}
and write the remainder at each step from bottom to top. We can see
the conversion between $\bin{10100}_{[\base{2}]}$ and \dec{20} (or
$\dec{20}_{[\base{10}]}$ to be extra-decimaly) below.
$$
\begin{array}{@{}l@{\hspace{0.9em}}
             |@{\hspace{0.9em}}l@{\hspace{0.9em}}
             |@{\hspace{0.9em}}l@{}}
\mathstrut
   \blankrow{}{}{}      & \blankrow{}{}{} & \blankrow{}{}{}
\\ \blankrow{}{}{}      & \blankrow{}{}{} & \blankrow{}{}{}
\\ \blankrow{}{1}{1}    & \blankrow{}{}{} & \blankrow{}{}{}
\\ \blankrow{1}{0}{2}   & \blankrow{}{}{} & \blankrow{}{}{}
\\ \blankrow{2}{1}{5}   & \blankrow{}{}{} & \blankrow{}{}{}
\\ \blankrow{5}{0}{10}  & \blankrow{}{}{} & \blankrow{}{}{}
\\ \blankrow{10}{0}{\dec{20}} & \blankrow{}{}{} & \blankrow{}{}{}
\end{array}
$$
\egroup

\vspace{-1ex}