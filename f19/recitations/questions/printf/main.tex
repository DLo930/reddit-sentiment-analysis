\section*{\lstinline'printf'%
\TAGS{c-string}}

\enlargethispage{5ex}
C0 and C1 had different print functions for each type:
\lstinline'printint', \lstinline'printbool', \lstinline'println', etc.
In C, there is just one main print function:
\lstinline'printf'. It always takes in a string,
but you can use \emph{format specifiers} to print other types.
%\footnote{\url{http://cplusplus.com/reference/cstdio/printf/} is a useful reference. Note that it was written for C\texttt{++}, so there may be some minor differences}
%\footnote{Feel free to
%search for more. \url{http://cplusplus.com/reference/cstdio/printf/}
%is for C\texttt{++}, but is a useful reference}.
Feel free to search online for format specifiers for more types.\footnote{The C\texttt{++} document
  \url{http://cplusplus.com/reference/cstdio/printf} is a good
  reference (C behaves similarly).}

\checkpoint*{}
\bgroup
\smalllistings
\lstinputlisting[belowskip=0pt]{\code/printing.c}
\egroup

What is the output of this program?

\begin{solution}
This program will print:\\
\lstinline'64'\\
\lstinline'40'\\
\lstinline'd'\\
\lstinline'100'\\
\lstinline'hello'\\
\lstinline'0xdeadbeef'\\
\lstinline'8'\\
\end{solution}
