\section*{Hexadecimal notation%
\TAGS{ints}}

\bgroup
\newcommand{\base}[1]{\textcolor{Brown}{#1}}
\newcommand{\bin}[1]{\textcolor{blue}{#1}}
\newcommand{\dec}[1]{\textcolor{red}{#1}}
\newcommand{\hex}[1]{\textcolor{Purple}{#1}}

Hex is useful because every hex digit corresponds to exactly 4 binary
digits (bits).  Base 8 (octal) is similarly useful: each octal digit
corresponds to exactly 3 bits. However, hex more evenly divides up a
32-bit integer.  In C0 we indicate we are using base \base{16} with an
\lstinline'0x' prefix, so $\hex{7f2c}_{[\base{16}]}$ is
\hex{\lstinline'0x7f2c'}.

{\footnotesize
\begin{tabular}{@{}l||*{16}{p{15.4px}|}@{}}
  \hex{Hex} & 0 & 1 & 2 & 3 & 4 & 5 & 6 & 7 & 8 & 9 & a & b & c & d & e & f
\\\hline
  \bin{Bin.} & 0000 & 0001 & 0010 & 0011 & 0100 & 0101 & 0110 & 0111 & 1000 & 1001 & 1010 & 1011 & 1100 & 1101 & 1110 & 1111
\\\hline
  \dec{Dec.} & 0 & 1 & 2 & 3 & 4 & 5 & 6 & 7 & 8 & 9 & 10 & 11 & 12 & 13 & 14 & 15
\end{tabular}}


Convert the binary number $\bin{1011111010101101}_{[\base{2}]}$ to hex.
\answerline{$~~~\hex{BEAD}_{[\base{16}]}~~~$}

Convert the hexadecimal number \hex{\lstinline'0x20'} to decimal.  \answerline{$~~~\dec{32}_{[\base{10}]}~~~$}

Why wouldn't it make sense to write a C0 function that converts hex
numbers to decimal numbers?

%\answerline{see below}
\begin{solution}
  Hex and decimal numbers are just ways of representing integers in
  C0.  They're in fact identical; they're just notational
  differences.(There is an int2hex function that returns a string, but
  that's not exactly the same).
\end{solution}
\egroup
