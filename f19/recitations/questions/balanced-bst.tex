\section*{Balanced search tree%
\TAGS{bst, complexity}}

Let's take a look at two binary trees that contain the same elements.

\begin{center}
\begin{tabular}{l | r}
  \begin{tikzpicture}
    \node (-6) {-6};
    \node[below right of=-6] (-3) {-3};
    \node[below right of=-3] (0) {0};
    \node[below right of=0] (2) {2};
    \node[below right of=2] (5) {5};
    \node[below right of=5] (7) {7};
    \node[below right of=7] (9) {9};

    \path
    (-6) edge (-3)
    (-3) edge (0)
    (0) edge (2)
    (2) edge (5)
    (5) edge (7)
    (7) edge (9);
  \end{tikzpicture}
& \begin{tikzpicture}
    \node (2) {2};

    \node[below left of=2, node distance=1.5cm] (-3) {-3};
    \node[below right of=2, node distance=1.5cm] (7) {7};

    \node[below left of=-3] (-6) {-6};
    \node[below right of=-3] (0) {0};


    \node[below left of=7] (5) {5};
    \node[below right of=7] (9) {9};

    \path
    (2) edge (-3)
    edge (7)
    (-3) edge (-6)
    edge (0)

    (7) edge (5)
    edge (9);
  \end{tikzpicture}
\end{tabular}
\end{center}

The height of the left tree is 7, while the height of the right tree
is only 3.  Say we want to access the element 9. In the left tree, we
need to travel down 7 levels, while on the right we only need to go
down 3. Remember that we have to do a comparison at each level, so
we'd like our trees to be as short as possible.
