\section*{Stack and queue interfaces%
\TAGS{interface, queue, stack}}

In lecture we discussed four functions exposed by the stack interface:
\begin{itemize}
  \itemsep=0pt
  \topsep=0pt
  \item \lstinline'stack_new': %
    Creates and returns a new stack
  \item \lstinline'stack_empty': %
    Given a stack, returns true if it is empty, and false otherwise
  \item \lstinline'push': %
    Given a stack and a string, puts the string on the top of the
    stack
  \item \lstinline'pop': %
    Given a stack, removes and returns the string on the top of the
    stack
\end{itemize}

Similarly, we discussed four functions exposed by the queue interface:
\begin{itemize}
  \itemsep=0pt
  \topsep=0pt
  \item \lstinline'queue_new': %
    Creates and returns a new queue
  \item \lstinline'queue_empty': %
    Given a queue, returns true if it is empty, and false otherwise
  \item \lstinline'enq': %
    Given a queue and a string, puts the string at the end of the queue
  \item \lstinline'deq': %
    Given a queue, removes and returns the string at the beginning of
    the queue
\end{itemize}
