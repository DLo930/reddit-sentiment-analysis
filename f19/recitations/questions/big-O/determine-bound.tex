\enlargethispage{5ex}
\hfill\TAGS{complexity}
\vspace{-1.5cm}% \smalllistings
\hspace*{-2em}%
\begin{minipage}[b]{0.5\linewidth}
\checkpoint*{}

For the following two functions, determine the big-O bound:

\bigskip\medskip
\begin{lstlisting}[numbers=left, showlines]
int bigO_1(int n) {
  int[] A = alloc_array(int, n);
  for (int i = 0; i < n; i++) {
    for (int j = 0; j < n; j++) {
      A[i] += j;
    }
  }
  f(A, n); //assume f takes O(log(n)) time
  return A[n-1];
}
\end{lstlisting}
\begin{solution}
	Nested for loop leads to $O(n^{2})$
\end{solution}
\end{minipage}%
\hfill\rule[1.5ex]{0.01em}{41ex}~~~~~%
\begin{minipage}[b]{0.39\linewidth}
\begin{lstlisting}[numbers=left]
int bigO_2(int[] L, int n) {
  int[] A = alloc_array(int, n);

  for (int i = 0; i < n; i++)
    A[i] = L[i];

  for (int i = 0; i < n; i++) {
    c = n;
    while (c > 0) {
      L[i] += 122;
      c /= 4;
    }
  }
  return L[n/2];
}
\end{lstlisting}
\begin{solution}
	The second for loop is $O(n)$, with a $log_{4}(n)$ runtime for
        the inner for loops. Together this is $O(n \log n)$.
\end{solution}
\end{minipage}
