\checkpoint*{\TAGS{big-o}}

Simplify the following big-O bounds without changing the sets they
represent:

\smallskip
$O(3n^{2.5} + 2n^2)$ can be written more simply as
\answerline{$O(n^{2.5})\qquad\qquad\qquad$}

\smallskip
$O(\log_{10} n + \log_{2}(7n))$ can be written more simply as
\answerline{$O(\log n)\qquad\qquad~$}

\bigskip
One interesting consequence of the second result in Checkpoint 2 is that
$O(\log_i n) = O(\log_j n)$ for all $i$ and $j$ (as long as they're both
greater than 1), because of the change of base formula:
$$
\log_i n = \frac{\log_j n}{\log_j i}
$$
But $\frac{1}{\log_j i}$ is just a constant! So,
it doesn't matter what base we use for logarithms in big-O notation.

When we ask for the \emph{simplest, tightest bound} in big-$O$, we'll usually
take points off if you write, for instance, $O(\log_2 n)$ instead of the simpler
$O(\log n)$.
