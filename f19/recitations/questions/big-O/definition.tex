\section*{Big-O definition%
\TAGS{big-o}}

The definition of big-O has a lot of mathematical symbols in it, and
so can be very confusing at first.  Let's familiarize ourselves with
the formal definition and get an intuition behind what it's saying.

$O(g(n))$ is a \emph{set} of functions, where
$f(n) \in O(g(n))$ if and only if:

\medskip
there is some \shortanswerline{$c \in \mathbb{R}^+$\hspace*{6em}} and some
\shortanswerline{$n_0 \in \mathbb{N}$\hspace*{6em}}

\medskip
such that \shortanswerline{$f(n) \leq c \, g(n)$\hspace*{4.5em}}
for all \shortanswerline{$n \geq n_0$\hspace*{5.8em}},

\bigskip
Although it isn't technically correct set notation, it is also common to write
$f(n) = O(g(n))$.
