\section*{Freeing memory%
\TAGS{c-memory}}

After you are done using any memory referenced by a pointer returned
by \lstinline[language=C]'malloc' or \lstinline[language=C]'calloc',
you must free it or your program will have ``memory leaks'' (in C0 and
C1, something called a garbage collector does this automatically). You
can free such memory by passing a pointer to it to
\lstinline[language=C]'free'. Once you free it, it is undefined to
access that memory. Also, don't free memory that was previously
freed. That also results in undefined behavior.

When we design libraries for data structures like stacks and BST's,
it's important to specify whether the client or library is responsible
for freeing each piece of memory that is
\lstinline[language=C]'malloc'ed. Usually, whoever allocates the
memory ``owns'' it and is responsible for freeing it, but in data
structures like BST's, we may want to transfer ownership of the memory
to the data structure. Thus, we ask the client to supply a ``freeing''
function in \lstinline[language=C]'bst_new' for BST elements, so that
when the client calls \lstinline[language=C]'bst_free', we can call
their ``freeing'' function on every element in the BST. If the
function pointer the client gives us is \lstinline[language=C]'NULL',
then we don't free the BST elements.

\input{bst_new}
