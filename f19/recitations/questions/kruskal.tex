\section*{Kruskal's algorithm%
\TAGS{spanning-tree}}

Kruskal's algorithm is an algorithm to find a \emph{minimum weight
  spanning tree} (often called a \emph{minimum spanning tree}) on a
graph. Visualization:
\url{http://www.cs.usfca.edu/~galles/visualization/Kruskal.html}.

A spanning tree of a graph is a subgraph that is a tree and that
connects all vertices of the graph.  (Remember that a tree is a
connected graph with no cycles.)

A \emph{minimum} spanning tree (MST) is simply a spanning tree whose
edges have the minimum total weight among all the spanning trees on the graph.
A graph may have multiple different MSTs.

Kruskal's algorithm is an algorithm that finds the minimum spanning
tree for a graph. The algorithm works as follows:
\begin{enumerate}[1.]
\item%
  Sort all edges by weight, from smallest weight to largest weight.
\item%
  Go through the edges in order. If adding an edge would not create a
  cycle in the graph, add it. When we have a minimum spanning tree
  (when the number of edges we've added is one less than the number of
  vertices), we're done.
\end{enumerate}
