\clearpage
\Question{Termination}\TAGS{correctness}

Recall that the standard way to prove termination of loops is by
determining an expression that always moves towards a specific bound
on every iteration of the loop. Reaching the bound implies that the
loop terminates.  For each of these functions, either prove that the
loop terminates by providing a C0 expression that must reach a
specific bound to guarantee termination, or show that the loop may not
terminate by providing arguments (specific values for \lstinline'x' and/or
\lstinline'y') on which the loop does not terminate.

Two examples are given below.

\textbf{Example 1}

\begin{lstlisting}
int move(int x, int y)
{
  while (x % 2 == 0)
  {
    x = x + 2;
  }
  return x+y;
}
\end{lstlisting}
This loop does not terminate when \lstinline'x' equals 0 (or any even
number) since an even number plus 2 is always an even number in C0.

\bigskip
\textbf{Example 2}

\begin{lstlisting}
int move(int x, int y)
{
  while (x > 10)
  {
    x--;
    y++;
  }
  return x;
}
\end{lstlisting}
If $x \leq 10$, the loop terminates immediately without running its body.

If $x > 10$, or equivalently if $x-10 > 0$, the quantity $x - 10$
strictly decreases by 1 for each iteration of the loop which means it
must reach the bound of 0 eventually, causing the loop to terminate.

\newpage

\begin{parts}
\RUBRIC
TAGS: correctness
ENDRUBRIC

\part[1]~\vspace{-2ex}
\begin{lstlisting}
int move(int x, int y)
{
  while (x < 50 || y < 50)
  {
    if (y > x)
      x++;
    else
      y++;
  }
  return x;
}
\end{lstlisting}
\begin{framed}
\ifprintanswers{\color{\answerColor}
\vspace{-2.5ex}
\begin{itemize}
\itemsep=-0.5ex
\item If \lstinline'x >= 50' and \lstinline'y >= 50', the loop never
  runs at all.
\item If \lstinline'x >= 50' and \lstinline'y < 50' (symmetrically for
  \lstinline'x < 50' and \lstinline'y <= 50'), then
  \begin{itemize}
  \itemsep=-0.5ex
  \item The quantity \lstinline'50 - y' is strictly decreasing and
    never goes negative
  \item The quantity \lstinline'y' is strictly increasing and is
    bounded above by 50
  \end{itemize}
\item If \lstinline'x < 50' and \lstinline'y < 50'
  \begin{itemize}
  \itemsep=-0.5ex
  \item The quantity \lstinline'100 - x - y' is strictly decreasing
    and never goes negative
  \item The quantity \lstinline'x+y' is strictly increasing and never
    passes 100
  \end{itemize}
\end{itemize}
\vspace{-2.5ex}
}\else~\vspace{1.5in}\fi
\end{framed}

\RUBRIC
Part (a)

Gradescope rubric:  TODO

Commentary:
Always terminates

Must get reasoning correct for credit, easiest approach is to split in
to multiple cases. As long as theirs a bound and an indication of
getting strictly closer to that bound, they're good.

  - If x >= 50 and y >= 50, the loop never runs at all.
  - If x >= 50 and y < 50 (symmetrically for x < 50 and y <= 50), then
      * The quantity 50 - y is strictly decreasing and never goes negative
      * The quantity y is strictly increasing and is bounded above by 50
  - If x < 50 and y < 50
      * The quantity 100 - x - y is strictly decreasing and never goes
        negative
      * The quantity x+y is strictly increasing and never passes 100

  - Half a point if their reasoning only works for x < 50 and y < 50
    initially, or if they give reasoning that doesn't include a bound
    and a quantity that strictly approaches that bound.

(Some TAs described this as the "251 critera." I don't know if that
may imply more rigor than we're going for, but there does need to be
*a bound* - it's okay if the bound is somewhat implicitly zero - and
*a quantity that strictly approaches that bound*. The reasoning for
proving that it strictly approaches that bound is trivial if they
choose the bound appropriately, so they don't really have to give us
anything there.)
ENDRUBRIC

\part[1]~\vspace{-2ex}
\begin{lstlisting}
int move(int x, int y)
{
  while (y < 50)
  {
    if (y > x)
      x = 2*x;
    else
      y++;
  }
  return x;
}
\end{lstlisting}
\begin{framed}
\ifprintanswers{\color{\answerColor}
The loop will not terminate if \lstinline'y < 50', \lstinline'x == 0',
or \lstinline'x=int_min()'.
}\else~\vspace{1.5in}\fi
\end{framed}

\RUBRIC
Part (b)

Gradescope rubric:  TODO

Commentary:
will not terminate if y < 50, x == 0, or x is -(2^n).
 Some other solutions are possible.

  - All or nothing (either zero or one for this part!), but make sure
    not to discount a valid counterexample.
ENDRUBRIC
\end{parts}
