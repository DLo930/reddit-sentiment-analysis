\part[1]~\vspace{-2ex}
\begin{lstlisting}
int move(int x, int y)
{
  while (x < 50 || y < 50)
  {
    if (y > x)
      x++;
    else
      y++;
  }
  return x;
}
\end{lstlisting}
\begin{framed}
\ifprintanswers{\color{\answerColor}
\vspace{-2.5ex}
\begin{itemize}
\itemsep=-0.5ex
\item If \lstinline'x >= 50' and \lstinline'y >= 50', the loop never
  runs at all.
\item If \lstinline'x >= 50' and \lstinline'y < 50' (symmetrically for
  \lstinline'x < 50' and \lstinline'y <= 50'), then
  \begin{itemize}
  \itemsep=-0.5ex
  \item The quantity \lstinline'50 - y' is strictly decreasing and
    never goes negative
  \item The quantity \lstinline'y' is strictly increasing and is
    bounded above by 50
  \end{itemize}
\item If \lstinline'x < 50' and \lstinline'y < 50'
  \begin{itemize}
  \itemsep=-0.5ex
  \item The quantity \lstinline'100 - x - y' is strictly decreasing
    and never goes negative
  \item The quantity \lstinline'x+y' is strictly increasing and never
    passes 100
  \end{itemize}
\end{itemize}
\vspace{-2.5ex}
}\else~\vspace{1.5in}\fi
\end{framed}

\RUBRIC
Part (a)

Gradescope rubric:  TODO

Commentary:
Always terminates

Must get reasoning correct for credit, easiest approach is to split in
to multiple cases. As long as theirs a bound and an indication of
getting strictly closer to that bound, they're good.

  - If x >= 50 and y >= 50, the loop never runs at all.
  - If x >= 50 and y < 50 (symmetrically for x < 50 and y <= 50), then
      * The quantity 50 - y is strictly decreasing and never goes negative
      * The quantity y is strictly increasing and is bounded above by 50
  - If x < 50 and y < 50
      * The quantity 100 - x - y is strictly decreasing and never goes
        negative
      * The quantity x+y is strictly increasing and never passes 100

  - Half a point if their reasoning only works for x < 50 and y < 50
    initially, or if they give reasoning that doesn't include a bound
    and a quantity that strictly approaches that bound.

(Some TAs described this as the "251 critera." I don't know if that
may imply more rigor than we're going for, but there does need to be
*a bound* - it's okay if the bound is somewhat implicitly zero - and
*a quantity that strictly approaches that bound*. The reasoning for
proving that it strictly approaches that bound is trivial if they
choose the bound appropriately, so they don't really have to give us
anything there.)
ENDRUBRIC
