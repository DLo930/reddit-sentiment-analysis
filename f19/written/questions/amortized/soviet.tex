\clearpage
\Question{Amortized Analysis}

\begin{parts}

\part[1]\TAGS{amortized-cost}
There are $n$ students $\{s_1, s_2, ..., s_{n}\}$ who want to get into
the Gates-Hillman Center after~6pm. Unfortunately, 15-122 TAs control
the building and charge a toll for entrance. The toll policy is the
following: the TAs have picked some $k \leq n$, and each student $s_i$
is charged $k^2$ tokens if $i \equiv 0 \bmod{k}$, or zero
otherwise. In other words, if $i$ is a multiple of $k$, then student
$s_i$ is charged $k^2$ tokens. Otherwise, the student enters for
free. With this policy, how much do the students pay \emph{altogether}?
\begin{framed}
\bigskip
\answer{34em}{$k(n - (n \mod k))$}
\end{framed}

\RUBRIC
Part (a)
TAGS: amortized-cost

Gradescope rubric: TODO

Commentary:
- solution: k(n - (n mod k))
- Full point for $\left\lfloor \frac{n}{k}\right\rfloor k^2$.
- Half point for kn or an unsimplified equivalent.
ENDRUBRIC


\part[1]\TAGS{amortized-cost}
In Soviet Russia, TAs pay you (the TAs really want students at their
office hours).  However, Soviet Russia is also communist, so the
students must split the money evenly between themselves. If the
\mbox{15-122} TAs in Soviet Russia used the same policy for entering
GHC (the Gavrilovich-Hashlov Center) after 6pm --- student $s_i$ is
paid $k^2$ tokens when $i \equiv 0 \bmod{k}$ --- how much will each
student end up with in the end? (i.e., what is the amortized cost
\emph{per student}?)
\begin{framed}
\bigskip
\answer{34em}{$k(n - (n \mod k)) / n$}
\end{framed}

\RUBRIC
Part (b)
TAGS: amortized-cost

Gradescope rubric: TODO

Commentary:
- solution: k(n - (n mod k)) / n
- Full point for their answer to (a) divided by n
- Full point for k
- Half point for O(1) without an explanation showing they found one of
  the solutions above
ENDRUBRIC


\end{parts}
