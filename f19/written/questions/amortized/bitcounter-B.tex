%% bitcounter-A.tex and bitcounter-B.tex differ only by the fact that
%% the variables n and k are swapped.

\clearpage
\Question{Amortized Analysis Revisited}

Consider a special binary counter represented as $n$ bits: $b_{n-1}
b_{n-2} \ldots b_1 b_0$. For this special counter, the cost of
flipping the $i$\textsuperscript{th} bit is $2^i$ tokens. For example, $b_0$
costs 1 token to flip, $b_1$ costs 2 tokens to flip, $b_2$ costs 4
tokens to flip, etc.  We wish to analyze the cost of performing $k =
2^n$ increments of this $n$-bit counter. (Note that $n$ is \emph{not}
a constant.)

Observe that if we begin with our $n$-bit counter containing all 0s,
and we increment $k$ times, where $k = 2^n$, the final value stored in
the counter will again be 0.

\begin{parts}

\part[1]\TAGS{amortized-cost}
The worst case for a single increment of the counter is when every bit
is set to 1. The increment then causes every bit to flip, the cost of
which is
$$
1 + 2 + 2^2 + 2^3 + \ldots + 2^{n-1}
$$

Find a closed form for the formula above.  Using this fact, explain in
one or two sentences why this cost is $O(k)$ --- again recall that $k
= 2^n$.
\begin{framed}
\bigskip
Closed form: \uanswer{27.9em}{$2^n-1$}

\bigskip
The cost is $O(k)$ because \uanswer{23.2em}{$2^n-1 = k-1$ since $k = 2^n$\hfill}

\bigskip
\uanswer{34em}{\hfill}

\bigskip
\uanswer{34em}{\hfill}
\end{framed}

\RUBRIC
Part (a)
TAGS: amortized-cost

Gradescope rubric:
+0.5 Recognizes sum is 2^n-1
+0.5 Observe that this is k-1

Commentary:
  +1/2 point: show/observe that sum is 2**n - 1.
  +1/2 point: show/observe that is k - 1
(A few might say that the sum is less than 2**n which is O(k). That's ok.)
ENDRUBRIC


\part[1\half]\TAGS{amortized-cost}
Now, we will use amortized analysis to show that although the worst
case for a single increment is $O(k)$, the amortized cost of a single
increment is asymptotically less than this. Remember, $k=2^n$.

Over the course of $k$ increments, how many tokens in total does it cost to
flip the $i$\textsuperscript{th} bit the necessary number of times?
\begin{framed}
\medskip
\answer{34em}{$k$}
\end{framed}

Based on your answer to the previous part, what is the total cost in
tokens of performing $k$ increments? (In other words, what is the
total cost of flipping each of the $n$ bits through $k$ increments?)
Write your answer as a function of $k$ \textbf{only}. (Hint: what is
$n$ as a function of $k$?)
\begin{framed}
\medskip
\answer{34em}{$k \log k$}
\end{framed}

\enlargethispage{5ex}
Based on your answer above, what is the amortized cost of a single
increment as a function of $k$ \textbf{only}?
\begin{framed}
\medskip
$O(\uanswer{16em}{$\log k$})$ amortized
\end{framed}

\RUBRIC
Part (b)
TAGS: amortized-cost

Gradescope rubric:
+0.5  1st blank: EITHER -- Answer is exactly k
+0.25 1st blank: OR -- Anwer is close (k-1, 2k, ...) or is O(k)
+0.5  2nd blank: k log(k) (or first blank times n, written as function of k)
+0.5  3rd blank: log(k) (or second blank divided by k)

Commentary:
  +1/2 pt: k tokens (or 2^n = 2^{n-i}/2^i)
     (1/4 point if they're close: k-1, 2k, something in O(k), basically)
  +1/2 pt: Whatever was in the first blank, times n.
      If first blank right: nk = n * 2^n = k log_2 k [ just k log k is fine ]
  +1/2 pt: O(whatever was in the second blank, divided by k)
      If first two blanks right: O(log k)
      Must be in terms of k and in simplest form.

  ANYONE ANSWERING 2^n, n * 2^n, n * 2^n / k or just n, give 1/2 point.

  ALTERNATE GRADING SCHEME: 1/2 for the 2nd blank being
  right, 1/2 for the 3rd blank being right. If that results
  in more points go with that.
ENDRUBRIC

\end{parts}
