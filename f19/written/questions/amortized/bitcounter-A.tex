%% bitcounter-A.tex and bitcounter-B.tex differ only by the fact that
%% the variables n and k are swapped.

\clearpage
\Question{Amortized Analysis Revisited}

Consider a special binary counter represented as $k$ bits: $b_{k-1}
b_{k-2} \ldots b_1 b_0$. For this special counter, the cost of
flipping the $i$\textsuperscript{th} bit is $2^i$ tokens. For example, $b_0$
costs 1 token to flip, $b_1$ costs 2 tokens to flip, $b_2$ costs 4
tokens to flip, etc.  We wish to analyze the cost of performing $n =
2^k$ increments of this $k$-bit counter. (Note that $k$ is \emph{not}
a constant.)

Observe that if we begin with our $k$-bit counter containing all 0s,
and we increment $n$ times, where $n = 2^k$, the final value stored in
the counter will again be 0.

\begin{parts}

\part[1]\TAGS{amortized-cost}
The worst case for a single increment of the counter is when every bit
is set to 1. The increment then causes every bit to flip, the cost of
which is
$$
1 + 2 + 2^2 + 2^3 + \ldots + 2^{k-1}
$$

Find a closed form for the formula above.  Using this fact, explain in
one or two sentences why this cost is $O(n)$ --- again recall that $n
= 2^k$.
\begin{framed}
\bigskip
Closed form: \uanswer{27.9em}{$2^k-1$}

\bigskip
The cost is $O(n)$ because \uanswer{23.2em}{$2^k-1 = n-1$ since $n = 2^k$\hfill}

\bigskip
\uanswer{34em}{\hfill}

\bigskip
\uanswer{34em}{\hfill}
\end{framed}

\RUBRIC
Part (a
TAGS: amortized-cost

Gradescope rubric:
+0.5 Recognizes sum is 2^k-1
+0.5 Observe that this is n-1

Commentary:
  +1/2 point: show/observe that sum is 2**k - 1.
  +1/2 point: show/observe that is n - 1
(A few might say that the sum is less than 2**k which is O(n). That's ok.)
ENDRUBRIC


\part[1\half]\TAGS{amortized-cost}
Now, we will use amortized analysis to show that although the worst
case for a single increment is $O(n)$, the amortized cost of a single
increment is asymptotically less than this. Remember, $n=2^k$.

Over the course of $n$ increments, how many tokens in total does it cost to
flip the $i$\textsuperscript{th} bit the necessary number of times?
\begin{framed}
\medskip
\answer{34em}{$n$}
\end{framed}

Based on your answer to the previous part, what is the total cost in
tokens of performing $n$ increments? (In other words, what is the
total cost of flipping each of the $k$ bits through $n$ increments?)
Write your answer as a function of $n$ \textbf{only}. (Hint: what is
$k$ as a function of $n$?)
\begin{framed}
\medskip
\answer{34em}{$n \log n$}
\end{framed}

\enlargethispage{5ex}
Based on your answer above, what is the amortized cost of a single
increment as a function of $n$ \textbf{only}?
\begin{framed}
\medskip
$O(\uanswer{16em}{$\log n$})$ amortized
\end{framed}

\RUBRIC
Part (b)
TAGS: amortized-cost

Gradescope rubric:
+0.5  1st blank: EITHER -- Answer is exactly n
+0.25 1st blank: OR -- Anwer is close (n-1, 2n, ...) or is O(n)
+0.5  2nd blank: n log(n) (or first blank times k, written as function of n)
+0.5  3rd blank: log(n) (or second blank divided by n)

Commentary:
  +1/2 pt: n tokens (or 2^k = 2^{k-i}/2^i)
     (1/4 point if they're close: n-1, 2n, something in O(n), basically)
  +1/2 pt: Whatever was in the first blank, times k.
      If first blank right: kn = k * 2^k = n log_2 n [ just n log n is fine ]
  +1/2 pt: O(whatever was in the second blank, divided by n)
      If first two blanks right: O(log n)
      Must be in terms of n and in simplest form.

  ANYONE ANSWERING 2^k, k * 2^k, k * 2^k / n or just k, give 1/2 point.

  ALTERNATE GRADING SCHEME: 1/2 for the 2nd blank being
  right, 1/2 for the 3rd blank being right. If that results
  in more points go with that.
ENDRUBRIC

\end{parts}
