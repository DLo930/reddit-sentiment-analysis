\part[4]

When using probing for collision resolution, it is important that we
detect situations where an insertion cannot succesfully be completed and emit an error message (or fail a contract),
rather than loop infinitely trying to do the impossible.  In linear probing, the only
situation where we can't insert is when the whole table is full.  This is easy to detect
since we store the size ($n$) and capacity ($m$) of the table.  However, as we saw in part
(b), there are situation with quadratic probing where the table may not be full, but insertions are still impossible.

It turns out with quadratic probing that only a certain number of attempts at insertion are
necessary before one can ``give up'' and conclude that insertion is impossible.
What is this number?
Prove its correctness.
Your answer should be in terms of $n$ and $m$ (but not necessarily both).

Hint: Find a cycle in the indices which are checked for insertion.  The cycle length should then be the number of
attemps we need to make.

\begin{solution}

\vspace{3.5in}

\end{solution}
