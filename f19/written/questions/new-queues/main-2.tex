\clearpage
\Question{Queues with Stacks}

Famous Fred Hacker's friend, Ned Stacker, loves stacks. He loves them
so much that he implemented a queue using two stacks in the following
way:
\begin{itemize}
\item A Staqueue has an ``in'' and an ``out'' stack.
\item To enqueue an element, the element is pushed on the ``in'' stack.
\item To dequeue an element, there are two cases:
  \begin{itemize}
  \item If the ``out'' stack is non-empty, then we simply pop from the
    ``out'' stack.
  \item Otherwise, we reverse the ``in'' stack onto the ``out'' stack
    by sequentially popping elements from the ``in'' stack and pushing
    them onto the ``out'' stack. We then pop from the ``out'' stack.
  \end{itemize}
\end{itemize}

A token can pay for the $O(1)$ cost of a stack push or pop; your
answers should be in terms of tokens (not in Big-O notation).

\begin{parts}
\part[1]\TAGS{complexity, stack}
Fred Hacker says that Ned's implementation is too slow.  What is the
\emph{worst case} cost for dequeuing an element in terms of the number
of elements in the staqueue, $n$?
\begin{framed}
\bigskip
\answer{34em}{$2n$}
\end{framed}

\RUBRIC
Part (c)
TAGS: complexity, stack

Gradescope rubric: TODO

Commentary:
- Full point for 2n, 2n+1, or 2n-1; or for n or n+1

- If they say 2n+x, their amortized cost should be 3 in part (d). If
  they say n+x, their amortized cost should be 2
ENDRUBRIC


\part[2]\TAGS{amortized-cost}
Cheer up Ned by giving and justifying the amortized cost for the queue
operations (enqueue and dequeue). You'll need to show that it is
possible to account for the eventual cost of a dequeue by paying some
of that cost each time you enqueue. (It may help to consider $n$ calls
to dequeue in a staqueue with $n$ elements.)
\begin{framed}
\ifprintanswers{\color{\answerColor}
Maintain a collection of tokens
on stack1.  In fact, keep 2 tokens for each thing in the
stack.

When we ENQUEUE, we have three tokens to work with.  We use
one to push the element onto stack1, and the other two are
put on the element for future use.  When we have to move
stack1 into stack2, we have enough tokens in the stack to
pay for the move (one POP and one PUSH for each element).
Finally, the last pop done by the DEQUEUE is paid for by the
1 we allocated for it.
}\else~\vspace{2.8in}\fi
\end{framed}

\RUBRIC
Part (d)
TAGS: amortized-cost

Gradescope rubric: TODO

Commentary:
- Amortized constant cost. We will always have performed at least as
  many enqueue operations as dequeue operations, otherwise the queue
  would underflow. If each enqueue deposits an extra token, then there
  will always be enough tokens to flip the in-stack, so dequeue doesn't
  run out of coins. Hence each operation of the stack only needs a
  constant amount of coins.

Sleator's proofs:

Proof 1 (aggregate method): As an element flows through the
two-stack data structure, it's PUSHed at most twice and
POPPed at most twice.  This shows that we can assign an
amortized cost of 4 to ENQUEUE and 0 to DEQUEUE.

To get the desired result, note that if an element is not
DEQUEUED it's only PUSHED twice and POPPED once.  So the
cost of 3 is paid for by the cost of 3 per ENQUEUE.  The
last POP is paid for by the DEQUEUE (if it happens).

Proof 2 (banker's method): Maintain a collection of tokens
on stack1.  In fact, keep 2 tokens for each thing in the
stack.

When we ENQUEUE, we have three tokens to work with.  We use
one to push the element onto stack1, and the other two are
put on the element for future use.  When we have to move
stack1 into stack2, we have enough tokens in the stack to
pay for the move (one POP and one PUSH for each element).
Finally, the last pop done by the DEQUEUE is paid for by the
1 we allocated for it. QED.

First point:
1 point for correct amortized cost (see part (a); they could have a
greater cost, provided it's still constant and they have an
explanation). Their cost of enqueue can be 2 or 3, and their cost of
dequeue can be 1 (or more) or 0, if they used the accounting method.

0 points for Big-O notation.

Second point:
1 point for a proper explanation along one of the lines
above. Students can get this point even if they had their answer in
terms of Big-O. No points if they don't either (1) show a constant
bound on pushes/pops per element, or (2) show the max cost of a
dequeue is bounded by the number of elements on the in stack (previous
enqueues) and paid for by their accrued ``tokens''.

ENDRUBRIC

\end{parts}