%% See also customization at the bottom

\clearpage
\Question{Binary Search Trees}

\begin{parts}

\newcommand{\insseqA}  % Vary from time to time
 {87, 78, 16, 121, 82, 104, 88, 41, 100, 111}
%{93, 86, 71, 115, 88, 99,  94, 77, 95,  109}
%{88, 79, 17, 122, 83, 105, 89, 42, 101, 112}
%{91, 85, 69, 110, 87, 98,  93, 76, 94,  108}
\newcommand{\insseqB}  % Vary from time to time
{76, 51, 103, -12, 88}
%{42, 11, 59, -7, 83}
%{59, 31, 81, -9, 87}
%{58, 29, 80, -8, 88}

\part[1]\TAGS{bst}
Draw the final binary search tree that results from inserting the
following keys in the order given.  Make sure all branches in your tree
are drawn \emph{clearly} so we can distinguish left branches from
right branches.
\bigskip
\begin{center}\tt
\insseqA
\end{center}
\bigskip
\begin{framed} % Paste the appropriate version

\ifprintanswers{
\begin{lstlisting}[basicstyle=\basicstyle\color{\answerColor}]
                        87
                      /    \
                     /      \
                   78        121
                  /  \       /
                16    82    104
                 \         /  \
                 41       88  111
                            \
                            100
\end{lstlisting}}
\else~\vspace{7in}\fi
\end{framed}

\RUBRIC
Part (a)
TAGS: bst

Gradescope rubric:
+1pt Correct

Commentary:
                        87
                      /    \
                     /      \
                   78        121
                  /  \       /
                16    82    104
                 \         /  \
                 41       88  111
                            \
                            100
ENDRUBRIC


\newpage
\part[2]\TAGS{bst}
How many different binary search trees can be constructed using the
following five keys if they can be inserted in any order?

\bigskip
\begin{center}\tt
\insseqB
\end{center}
\bigskip

Show how your answer is derived. We've begun the derivation below;
we've used $t(n)$ to stand for the number of binary search trees with
$n$ nodes.

\emph{Think recursively:} How many trees with 0 nodes can possibly
exist? How many different trees with 1 node can possibly exist?  2
nodes? 3 nodes? 4 nodes? Think about how to build up your answer from
answers to simpler questions. (It might help to come back to this
question after doing the last question on AVL tree height.)

\bigskip
\begin{center}
\newcommand{\ans}[1]{\answer{27em}{#1}}
\renewcommand{\arraystretch}{2.5}
\begin{tabular}{@{}|c|p{32.6em}|@{}}
       \hline $n$ & \multicolumn{1}{c|}{$t(n)$}
\\[5pt]\hline  0  & $t(0) = 1$
\\[5pt]\hline  1  & $t(1) = 1$
\\[5pt]\hline  2  & $t(2) = t(0)\times t(1) + t(1) \times t(0) = 2$
\\[5pt]\hline  3  & $t(3) = \ans{$t(2) \times t(0) + t(1) \times t(1) +
                                 t(0) \times t(2) \;\;=\;\; 2 + 1 + 2 \;\;=\;\; 5$}$
\\[5pt]\hline  4  & $t(4) = \ans{$\sum_{i=0}^3 t(i) \times t(3-i) \;\;=\;\; 5 + 2 + 2 + 5 \;\;=\;\; 14$}$
\\[5pt]\hline  5  & $t(5) = \ans{$\sum_{i=0}^4 t(i) \times t(4-i) \;\;=\;\; 14 + 5 + 4 + 5 + 14 \;\;=\;\; 42$}$
\\[5pt]\hline
\end{tabular}
\end{center}

\bigskip
What is the general formula for $t(n)$ when $n > 0$?
\begin{framed}
\medskip
$t(n) = \answer{31em}{$\sum_{i=0}^n t(i) \times t(n-i)$}$
\end{framed}

\RUBRIC
Part (b)
TAGS: bst

Gradescope rubric:
+1pt Correct answers in table (5, 14, 42)
+1pt Correct general formula

Commentary:
 Derivation:
 t(0): 1
 t(1): 1
 t(2): t(1)*t(0) + t(0)*t(1) = 1 + 1 = 2
 t(3): t(2)*t(0) + t(1)*t(1) + t(0)*t(2) = 2 + 1 + 2 = 5
 t(4): t(3)*t(0) + t(2)*t(1) + t(1)*t(2) + t(0)*t(3) = 5 + 2 + 2 + 5 = 14
 t(5): t(4)*t(0) + t(3)*t(1) + t(2)*t(2) + t(1)*t(3) + t(0)*t(4) = 14 + 5 + 4 + 5 + 14 = 42

It doesn't need to be in exactly the above form but each line should
have the correct number. One point if they make one mistake and
following lines are consistent with it.

General form (many solutions):
- \Sum_{i=0}^{n-1} t(i) \times t(n-i-1)
- \Sum_{i=1}^n t(i-1) \times t(n-i)
- \choose{2n}{n}/(n+1)
- (2n)!/((n+1)! \times n!)
- t(n-1) \times (4n-2)/(n+1)
ENDRUBRIC


\newpage
\begin{EnvUplevel}
  For the next few questions, we consider the implementation of
  dictionaries as binary search trees in the lecture notes.  In
  particular, recall the following declarations:
\end{EnvUplevel}

\begin{lstlisting}[aboveskip=2pt, belowskip=-6pt]
// typedef _______ key;
// typedef _______* entry;
key entry_key(entry e) /*@requires e != NULL; @*/ ;

bool is_bst(tree* T) {
  return is_tree(T) && is_ordered(T, NULL, NULL);
}
\end{lstlisting}
\begin{minipage}[t]{0.5\linewidth}
\begin{lstlisting}[aboveskip=0pt, belowskip=0pt]
struct dict_header {
  tree* root;
};
typedef struct dict_header dict;
\end{lstlisting}
\end{minipage}%
\hfill\rule[-11ex]{0.01em}{10.5ex}\hfill%
\begin{minipage}[t]{0.44\linewidth}
\begin{lstlisting}[aboveskip=0pt, belowskip=0pt]
bool is_dict(dict* D) {
  return D != NULL
      && is_bst(D->root);
}
\end{lstlisting}
\end{minipage}
\begin{EnvUplevel}
Like in class, the
client defines two functions: \lstinline'entry_key(e)' that extracts
the key of entry \lstinline'e', and \lstinline'key_compare(k1,k2)' that
returns \lstinline'-1' if key \lstinline'k1' is ``less than'' key
\lstinline'k2',
\lstinline'0' if \lstinline'k1' is ``equal to'' \lstinline'k2', and
\lstinline'1' if \lstinline'k1' is ``greater than'' \lstinline'k2'.
\end{EnvUplevel}
\vspace{-1ex}

%\input{print} % S19 S18 S17 S16
\part[1]\TAGS{bst, traverse-ds}
Assume the client also provides a function
\lstinline'entry_print(e)' that prints entry \lstinline'e'
in a readable format on one line.  Complete the function
\lstinline'dict_reverseprint' which prints the entries of
the given dictionary on one line in order from largest key to smallest
key.  If the dictionary is empty, nothing is printed. You will need a
\textbf{recursive} helper function \lstinline'tree_reverseprint' to
complete the task.

\emph{Think recursively: if you are at a non-empty node, what are the
  three things you need to print, and in what order?  You should not
  need to examine the keys since the contract guarantees the argument
  is a BST.}

\enlargethispage{5ex}
\begin{framed}
\begin{lstlisting}[aboveskip=0pt, belowskip=0pt]
void tree_reverseprint(tree* T)
//@requires is_bst(T);
{
\end{lstlisting}
\ifprintanswers
\begin{lstlisting}[basicstyle=\basicstyle\color{\answerColor}]
   if (T == NULL) return;
   tree_reverseprint(T->right);
   entry_print(T->data);
   tree_reverseprint(T->left);
\end{lstlisting}
\else~\vspace{3.5cm}\fi
\begin{lstlisting}[aboveskip=0pt, belowskip=0pt]
}

void dict_reverseprint(dict* D)
//@requires is_dict(D);
{
  tree_reverseprint([*\uanswer{24.5em}{D->root}*]);
  print("\n");
}
\end{lstlisting}
\end{framed}

\RUBRIC
Part (c)
TAGS: bst, traverse-ds

Gradescope rubric:
+0.25pt bst_reverseprint answer is correct: return B->root
+0.25pt tree_reverseprint correctly handles the base case (check if T == NULL and return if so)
+0.25pt Recursive calls happen in the correct order whenever T != NULL
+0.25pt Syntax and overall tree_reverseprint function is correct (possibly aside from the base case) and not needlessly complex

Commentary:
 1/2 point for the bst_reverseprint answer and correct handling of the base case
   (It doesn't work to implicitly assume and then enforce T != NULL as
   a precondition, because of the template we gave for bst_reverseprint. -1/2
   point if they try to do this and get it wrong.)
 1/2 point for the recursive calls and call to entry_print (and possibly
   printing spaces before and after; we don't care)
   (It's okay if they're mishandling the NULL case, though, -1/2 point
   if they get this in the wrong order, if they miss one, or if they
   don't correctly call entry_print)

 void tree_reverseprint(tree* T) {
   if (T == NULL) return;
   tree_reverseprint(T->right);
   entry_print(T->data);
   tree_reverseprint(T->left);
 }

 void bst_reverseprint(dict* D) {
   tree_reverseprint(D->root);

 }
ENDRUBRIC
 % F19 F18 F17 F16

\newpage
%%% Second task comes in two variants

\newpage
\begin{EnvUplevel}
The function \lstinline'bst_lookup'  in the lecture notes is recursive,
but it is also possible to implement it iteratively.
\end{EnvUplevel}

\part[1]\TAGS{bst, divide-and-conquer, loop-invariant}
Fill in the missing code.

\begin{lstlisting}[numbers=left, frame=single]
entry bst_lookup(tree* T, key k)
//@requires is_bst(T);[*\label{l:bst_lookup-pre}*]
/*@ensures \result == NULL
        || key_compare(k, entry_key(\result)) == 0; @*/
{
  entry lo = NULL;        // to support the loop invariant[*\label{l:bst_lookup-init1}*]
  entry hi = NULL;        // to support the loop invariant[*\label{l:bst_lookup-init2}*]

  while ([*\uanswer{33em}{T != NULL \&\&\hfill}*][*\label{l:bst_lookup-while}*]

         [*\uanswer{33em}{key\_compare(k, entry\_key(T->data)) != 0\hfill}*])[*\label{l:bst_lookup-while_B}*]
  //@loop_invariant is_ordered(T, lo, hi);[*\label{l:bst_lookup-LI}*]
  {
    if ([*\uanswer{32.3em}{key\_compare(k, entry\_key(T->data)) < 0}*]) {[*\label{l:bst_lookup-if}*]
      hi = T->data;       // to support loop invariant[*\label{l:bst_lookup-inv1}*]
      T = T->left;[*\label{l:bst_lookup-inv2}*]
    }
    else {
      lo = T->data;       // to support loop invariant
      T = T->right;
    }
  }

  if (T == NULL) return NULL;
  return T->data;
}
\end{lstlisting}

\RUBRIC
Part <lookup>-1
TAGS: bst, divide-and-conquer, loop-invariant

Gradescope rubric:
+0.5pt loop guard:  T != NULL && key_compare(k, entry_key(T->data))
+0.5pt if condition expresses that left subtree < T->data

Commentary:
First slot:
  T != NULL && key_compare(k, entry_key(T->data)) != 0

Second slot: either of
- key_compare(k, entry_key(T->data)) < 0
- key_compare(k, entry_key(T->data)) == -1
- key_compare(entry_key(T->data), k) > 0
- key_compare(entry_key(T->data), k) == 1

ENDRUBRIC

\bigskip
\uplevel{%
The rest of this task verifies the safety and correctness of various
parts of the above code using the methodology and format seen in this
course.  No or little credit will be given to answers that are
unclear, verbose or unjustified.
}%

\newpage
%%% VARIANT A
\part[1]\TAGS{bst, safety}
\label{part:correctness}
Prove that the loop guard you wrote on
lines~\ref{l:bst_lookup-while}--\ref{l:bst_lookup-while_B} is safe ---
that pointer dereferences are safe and that the preconditions of any
function you call are satisfied. You may assume that the loop
invariant is correct. \emph{(You may not need all lines.)}
%(Don't forget that \lstinline'entry_key' has a precondition!)
\begin{framed}
\newcommand{\ans}[2]{\item\uanswer{18em}{#1}~~by~~\uanswer{12em}{#2}}

\medskip
\uanswer{27em}{\lstinline'T->data'} is safe because:
\begin{enumerate}[(a)]
\ans{\lstinline'T != NULL'}{line~\ref{l:bst_lookup-while} (short-circuiting)\hfill}
\ans{}{}
%\ans{}{}
\end{enumerate}

\medskip
\uanswer{27em}{\lstinline'entry_key(T->data)'} is safe because:
\begin{enumerate}[(a)]
\ans{\lstinline'is_ordered(T, lo, hi) == true'}{line~\ref{l:bst_lookup-LI}}
\ans{\lstinline'T != NULL'}{def.~of \lstinline'is_ordered'
  and (a)\hfill}
\ans{\lstinline'T->data != NULL'}{by (b)}
%\ans{}{}
\end{enumerate}
\vspace{-2ex}
\end{framed}

\RUBRIC
Part <lookup>-2
TAGS: bst, safety

Gradescope rubric:
+0.5pt  Account for all pointer dereferences/array accesses
+0.5pt  EITHER -- Ensures preconditions of entry_key is met
+0.5pt      OR -- Answers are correct given code on lines 9 and 12

Commentary:
  . 0.5 points for proving safety of T->data
  . 0.5 points for showing that the precondition of entry_key is met.
  . safety of T->data is all or nothing

- In the precondition proof, one minor error (not distinguishing
  between ``by the loop guard'' and ``by short-circuiting'', or the
  like) should be forgiven. Several minor errors, deduct 0.5 points.

- If their answer is wrong but consistent with part (a), give full credit
  (same goes for the other two proofs)

  On line 9, the pointer dereference T->data is safe because this
  code is only run if T != NULL by short circuiting.  Now, we must
  check that the precondition of entry_key is met, namely that T->data
  != NULL.  By the loop invariant, we know that is_ordered(T, lo,
  hi) is true.  If the code on line 9 calls key_compare, then T !=
  NULL by short circuiting.  In this case, is_ordered will return
  false if T->data is NULL.  Thus, T->data is not NULL, so the
  precondition is satisfied.

  An identical argument applies to line 12, except we know that T !=
  NULL by the loop guard, not by short circuiting.
ENDRUBRIC


% %%% VARIANT B
% \part[1]\TAGS{bst, safety}
% \label{part:correctness}
% Prove that the condition you wrote on line~\ref{l:bst_lookup-if} is
% safe --- that pointer dereferences are safe and that the preconditions
% of any function you call are satisfied. You may assume that the loop
% invariant is correct.
% %(Don't forget that \lstinline'entry_key' has a precondition!)
% \begin{framed}
% \newcommand{\ans}[2]{\item\uanswer{18em}{#1}~~by~~\uanswer{12em}{#2}}

% \medskip
% \uanswer{27em}{\lstinline'T->data'} is safe because:
% \begin{enumerate}[(a)]
% \ans{\lstinline'T != NULL'}{line~\ref{l:bst_lookup-while}\hfill}
% \ans{}{}
% %\ans{}{}
% \end{enumerate}

% \medskip
% \uanswer{27em}{\lstinline'entry_key(T->data)'} is safe because:
% \begin{enumerate}[(a)]
% \ans{\lstinline'is_ordered(T, lo, hi) == true'}{line~\ref{l:bst_lookup-LI}}
% \ans{\lstinline'T != NULL'}{def.~of \lstinline'is_ordered'
%   and (a)\hfill}
% \ans{\lstinline'T->data != NULL'}{by (b)}
% %\ans{}{}
% \end{enumerate}
% \vspace{-2ex}
% \end{framed}

% \RUBRIC
% Part <lookup>-2
% TAGS: bst, safety

% Gradescope rubric:
% +0.5pt  Account for all pointer dereferences/array accesses
% +0.5pt  EITHER -- Ensures preconditions of entry_key is met
% +0.5pt      OR -- Answers are correct given code on lines 9 and 12

% Commentary:
%   . 0.5 points for proving safety of T->data
%   . 0.5 points for showing that the precondition of entry_key is met.
%   . safety of T->data is all or nothing

% - In the precondition proof, one minor error (not distinguishing
%   between ``by the loop guard'' and ``by short-circuiting'', or the
%   like) should be forgiven. Several minor errors, deduct 0.5 points.

% - If their answer is wrong but consistent with part (a), give full credit
%   (same goes for the other two proofs)

%   On line 9, the pointer dereference T->data is safe because this
%   code is only run if T != NULL by short circuiting.  Now, we must
%   check that the precondition of entry_key is met, namely that T->data
%   != NULL.  By the loop invariant, we know that is_ordered(T, lo,
%   hi) is true.  If the code on line 9 calls key_compare, then T !=
%   NULL by short circuiting.  In this case, is_ordered will return
%   false if T->data is NULL.  Thus, T->data is not NULL, so the
%   precondition is satisfied.

%   An identical argument applies to line 12, except we know that T !=
%   NULL by the loop guard, not by short circuiting.
% ENDRUBRIC


\part[0\half]\TAGS{bst, correctness} %
Prove that the loop invariant is true initially.
\begin{framed}
\newcommand{\ans}[2]{\item\uanswer{18em}{#1}~~by~~\uanswer{12em}{#2}}
\medskip
\makebox[4em][l]{To show:}~~\uanswer{25.5em}{\lstinline'is_ordered(T, lo, hi)'} initially.

\bigskip
\begin{enumerate}[(a)]
\ans{\lstinline'lo == NULL'}{line~\ref{l:bst_lookup-init1}\hfill}
\ans{\lstinline'hi == NULL'}{line~\ref{l:bst_lookup-init2}\hfill}
\ans{\lstinline'is_bst(T)'}{line~\ref{l:bst_lookup-pre} (precondition)\hfill}
\ans{\lstinline'is_ordered(T, NULL, NULL)'}{definition of \lstinline'is_bst'\hfill}
%\ans{}{}
\end{enumerate}
\vspace{-2ex}
\end{framed}


\RUBRIC
Part <lookup>-3

Gradescope rubric:
+0.1pt To show: is_ordered(T, lo, hi) initially
+0.15pt lo == hi == NULL by lines 6-7
+0.15pt is_bst(T) by line 2/precondition
+0.1pt is_ordered(T, NULL, NULL) by definition of is_bst

Commentary:
- If they note that the invariant follows from is_bst(T) by way
  of is_ordered(T, NULL, NULL), give at least some
  credit. Deduct 0.25 point for any muddles they get into.

  Initially, lo == hi == NULL by lines 6-7.
  Thus we need to show that is_ordered(T, NULL, NULL) is
  true initially.  By the precondition on line 2, is_bst(T) is true.
  As is_bst(T) is true only if is_ordered(T, NULL, NULL) is
  true, the loop invariant is satisfied.
ENDRUBRIC

\enlargethispage{5ex}
\part[1]\TAGS{bst, correctness} %
Prove that the loop invariant is preserved by any iteration of the
loop.  Only show the case where the conditional on
line~\ref{l:bst_lookup-if} evaluates to \lstinline'true'.
\begin{framed}
\newcommand{\ans}[2]{\item\uanswer{18em}{#1}~~by~~\uanswer{12em}{#2}}
\medskip
Assuming \uanswer{12em}{\lstinline'is_ordered(T,lo,hi)'},
show that \uanswer{12em}{\lstinline'is_ordered(T`,lo`,hi`)'}

\bigskip
\begin{enumerate}[(a)]
\ans{\lstinline'hi` == T->data'}{line~\ref{l:bst_lookup-inv1}\hfill}
\ans{\lstinline'T` == T->left'}{line~\ref{l:bst_lookup-inv2}\hfill}
\ans{\lstinline'T != NULL'}{line~\ref{l:bst_lookup-while} (loop guard)\hfill}
\ans{\lstinline'is_ordered(T, lo, hi)'}{assumption\hfill}
\ans{\lstinline'is_ordered(T->left, lo, T->data)'}{def.~of \lstinline'is_ordered'\hfill}
\ans{}{}
%\ans{}{}
\end{enumerate}
\vspace{-2ex}
\end{framed}

\RUBRIC
Part <lookup>-4

Gradescope rubric:
+0.2pt: Assume that is_ordered(T, lo, hi)
+0.2pt: To show: is_ordered(T', lo', hi')
+0.2pt: hi' == T->data and T' == T->left by appropriate lines
+0.2pt: T != NULL by line 9 (hopefully)
+0.2pt: is_ordered(T->left, lo, T->data) by assumption (definition of is_ordered)

Commentary:
- Deduct 0.2 point if they forget to argue that T is not NULL.

By lines 13 and 14, hi' == T->data and T' == T->left.  To consider that
the loop invariant is preserved we need to show that
is_ordered(T->left, lo, T->data) is true at the end of the loop.
By the loop guard, T != NULL.  By the loop invariant, is_ordered(T,
lo, hi).  When T != NULL, is_ordered(T, lo, hi) will only
return true if is_ordered(T->left, lo, T->data) and
is_ordered(T->right, T->data, hi) are both true.  In particular,
the first of these is true, which is what we wanted to show.
ENDRUBRIC
 % F19 S18 S17
%\input{insert} % S19 S16
%\input{delete} % F18 F17 F16
%\input{height} %

\end{parts}
