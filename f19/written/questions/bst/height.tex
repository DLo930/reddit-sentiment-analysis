\part[3]\TAGS{bst, traverse-ds}
Write an implementation of a new dictionary library function,
\lstinline'dict_load', that returns a measure of how long it will take
to lookup or insert an entry.  It does so by returning the height of
the underlying binary search
tree.  The height of a binary search tree is defined as the maximum
number of nodes as you follow a path from the root to a leaf. As a
result, the height of an empty binary search tree is 0.  Your function
must include a \textbf{recursive} helper function
\lstinline'tree_height'.

HINT: In general, the height of a tree rooted at node $T$ is one more
than the height of its deepest subtree.

\begin{framed}
\begin{lstlisting}[aboveskip=0pt, belowskip=0pt]
int tree_height(tree* T)
//@requires is_tree(T);
{
\end{lstlisting}
\ifprintanswers
\begin{lstlisting}[basicstyle=\basicstyle\color{\answerColor}]
  if (T == NULL) return 0;

  return 1 + max(tree_height(T->left), tree_height(T->right));
\end{lstlisting}
\else~\vspace{2.1in}\fi
\begin{lstlisting}[aboveskip=0pt,belowskip=0pt]
}

int dict_load(dict* D)
//@requires is_dict(D);
//@ensures  is_dict(D);
{
  return [*\uanswer{31em}{tree\_height(D->root)}*];
}
\end{lstlisting}
\end{framed}

\RUBRIC
Part (d -- height)
TAGS: bst, traverse-ds

Gradescope rubric:
+1pt tree_heigh: correct base case
+1pt tree_height: correct recursive case
+1pt dict_load: calls tree_height on D->root

Commentary:

ENDRUBRIC