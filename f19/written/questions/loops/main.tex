\clearpage
\Question{Thinking about Loops}

%% Change questions at the bottom of this file

When we think about loops in 15-122, we will always concentrate on a
single iteration of the loop. A loop will almost always modify something;
the following loop modifies the local assignable \lstinline'i'.

\begin{lstlisting}
while (i < n) {
   i = i + 4;
}
\end{lstlisting}

In order to reason about the loop, we have to think about the two
different values stored in the local assignable \lstinline'i' during
an iteration.

We use the variable $i$ to talk about the value stored in the local
\lstinline'i' before the loop runs (before the loop guard is checked
for the first time).

We use the ``primed'' variable $i'$ to talk about the value stored in
the local \lstinline'i' after the loop runs exactly one more time
(before the loop guard is next checked).

\begin{parts}
\newcommand{\ans}[2][7.5em]{\fbox{\rule[-0.25ex]{0em}{3ex}\answer{#1}{#2}}}
\newcommand{\ansB}[1]{\fbox{\rule{0em}{1ex}~\answer{0em}{#1}~}}

%\part[1]\TAGS{testing}
Consider the following loop:

\begin{lstlisting}
while (i < n) {
    k = j + k;
    j = j * 2 + i;
    i = i + 1;
}
\end{lstlisting}

\begin{itemize}

\item
If $i = 7$, $j = 3$, and $k = 9$, then assuming $7 < \texttt{n}$,

\medskip
$i'$ = \ans{8},
$j'$ = \ans{13}, and
$k'$ = \ans{12}

\bigskip
\item
If $i = 2y$, $j = x - y$, and $k = y$, then assuming $2y < \texttt{n}$,
in terms of $x$ and $y$,

\medskip
$i'$ = \ans{$2y + 1$},
$j'$ = \ans{$2x$}, and
$k'$ = \ans{$x$}

\bigskip
\item
If $j = k$, then assuming $i < \texttt{n}$,
in terms of $i$ and $k$,

\medskip
$i'$ = \ans{$i + 1$},
$j'$ = \ans{$2k + i$}, and
$k'$ = \ans{$2k$}

\bigskip
\item In general, assuming $i < n$, then in terms of $i$, $j$, and $k$,

\medskip
$i'$ = \ans{$i + 1$},
$j'$ = \ans{$2j + i$}, and
$k'$ = \ans{$j + k$}
\end{itemize}

\medskip
Note that we always say ``assuming (something) $<$ \texttt{n},''
because if that were not the case the loop wouldn't run, and it
wouldn't make any sense to be talking about the values of the primed
variables.

\RUBRIC
Part (a)
TAGS: testing

Gradescope rubric:
+ 0.25 pts First bullet: i' = 8, j' = 13, k' = 12
+ 0.25 pts Second bullet: i' = 2y + 1, j' = 2x, k' = x
+ 0.25 pts Third bullet: i' = i + 1, j' = 2k + i, k' = 2k
+ 0.25 pts Fourth bullet: i' = i + 1, j' = 2j + i, k' = j + k
ENDRUBRIC
   % S19 S18 S17 F16 S16
\part[1]\TAGS{testing}
Consider the following loop:

\begin{lstlisting}
while (i < n) {
    k = 2 * j - k;
    j = 3 * j + i;
    i = i - 1;
}
\end{lstlisting}

\begin{itemize}

\item
If $i = 6$, $j = 2$, and $k = 7$, then assuming $6 < \texttt{n}$,

\medskip
$i'$ = \ans{5},
$j'$ = \ans{12}, and
$k'$ = \ans{-3}

\bigskip
\item
If $i = 3y$, $j = x + y$, and $k = 2y$, then assuming $3y < \texttt{n}$,
in terms of $x$ and $y$,

\medskip
$i'$ = \ans{$3y - 1$},
$j'$ = \ans{$3x + 6y - 1$}, and
$k'$ = \ans{$2x$}

\bigskip
\item
If $j = k$, then assuming $i < \texttt{n}$,
in terms of $i$ and $k$,

\medskip
$i'$ = \ans{$i - 1$},
$j'$ = \ans{$3k + i$}, and
$k'$ = \ans{$k$}

\bigskip
\item In general, assuming $i < n$, then in terms of $i$, $j$, and $k$,

\medskip
$i'$ = \ans{$i - 1$},
$j'$ = \ans{$3j + i$}, and
$k'$ = \ans{$2j - k$}
\end{itemize}

\medskip
Note that we always say ``assuming (something) $<$ \texttt{n},''
because if that were not the case the loop wouldn't run, and it
wouldn't make any sense to be talking about the values of the primed
variables.

\RUBRIC
Part (a)
TAGS: testing

Gradescope rubric:
+ 0.25 pts First bullet: i' = 5, j' = 12, k' = -3
+ 0.25 pts Second bullet: i' = 3y - 1, j' = 3x + 6y - 1, k' = 2x
+ 0.25 pts Third bullet: i' = i - 1, j' = 3k + i, k' = k
+ 0.25 pts Fourth bullet: i' = i - 1, j' = 3j + i, k' = 2j - k
ENDRUBRIC
   % F19 F18

%\clearpage
\part[0\half]\TAGS{testing}
Consider this loop:

\begin{lstlisting}
while (...) {
    i = i + 3;
    j = j * 2 + i;
    k = k + i - j;
}
\end{lstlisting}

Be careful, it looks similar but is trickier! Give simplified answers.

\begin{itemize}

\item
If $i = 7$, $j = 3$, and $k = 9$, then assuming the loop guard evaluates
to true,

\medskip
$i'$ = \ans{10},
$j'$ = \ans{16}, and
$k'$ = \ans{3}

\bigskip
\item In general, assuming the loop guard evaluates to true, then in terms of $i$, $j$, and $k$,

\medskip
$i'$ = \ans{$i + 3$}
$j'$ = \ans{$2j + i + 3$}, and
$k'$ = \ans{$k - 2j$},

\end{itemize}

\RUBRIC
Part (b)
TAGS: testing

Gradescope rubric:
+ 0.25 pts First bullet: i' = 10, j' = 16, k' = 3
+ 0.25 pts Second bullet: i' = i + 3, j' = 2*j + i + 3, k' = k - 2*j

Commentary:
Showing work (students need not do this)

    * i[ 7] j[ 3] k[ 9]
      i[10] j[ 3] k[ 9]
      i[10] j[16] k[ 9]
      i[10] j[16] k[ 3]

    * i[i]     j[j]          k[k]
      i[i + 3] j[j]          k[k]
      i[i + 3] j[2j + i + 3] k[k - 2j]
ENDRUBRIC
   % S19 S18 S17 F16 S16
\clearpage
\part[0\half]\TAGS{testing}
Consider this loop:

\begin{lstlisting}
while (...) {
    i = i - 1;
    j = 3 * j + i;
    k = 2 * j - k;
}
\end{lstlisting}

Be careful, it looks similar but is trickier! Give simplified answers.

\begin{itemize}

\item
If $i = 6$, $j = 2$, and $k = 7$, then assuming the loop guard evaluates
to true,

\medskip
$i'$ = \ans{5},
$j'$ = \ans{11}, and
$k'$ = \ans{15}

\bigskip
\item In general, assuming the loop guard evaluates to true, then in terms of $i$, $j$, and $k$,

\medskip
$i'$ = \ans{$i - 1$}
$j'$ = \ans{$3j + i - 1$}, and
$k'$ = \ans{$6j + 2i - 2 - k$},

\end{itemize}

\RUBRIC
Part (b)
TAGS: testing

Gradescope rubric:
+ 0.25 pts First bullet: i' = 5, j' = 11, k' = 15
+ 0.25 pts Second bullet: i' = i - 1, j' = 3*j + i - 1, k' = 6*j + 2*i - 2 - k

Commentary:
Showing work (students need not do this)

    * i[ 6] j[ 2] k[ 7]  (entering the body of the loop)
      i[ 5] j[ 2] k[ 7]  (after i = i - 1)
      i[ 5] j[11] k[ 7]  (after j = 3*j + i)
      i[ 5] j[11] k[15]  (after k = 2*j - k)

    * i[i]     j[j]        k[k]          (entering the body of the loop)
      i[i-1]   j[j]        k[k]          (after i = i - 1)
      i[i-1]   j[3j+i-1]   k[k]          (after j = 3*j + i)
      i[i-1]   j[3j+i-1]   k[6j+2i-2-k]  (after k = 2*j - k)

ENDRUBRIC
   % F19 F18

%\part[1]\TAGS{loop-invariant, testing}
Consider this loop:

\begin{lstlisting}
while (a > 0 && b > 0) {
    if (a > b) {
        a = a-b;
    } else {
        b = b-a;
    }
}
\end{lstlisting}

\begin{itemize}
\itemsep=2ex

\item If $a = 94$ and $b = 12$, then

\medskip
$a'$ = \ans{82} and
$b'$ = \ans{12}

\item If $a = x+y$ and $b = x$, where $x$ and $y$ are both positive
  integers, then

\medskip
$a'$ = \ans{$y$} and
$b'$ = \ans{$x$}

\item If $a = x$ and $b = x+z$, where $x$ is a positive integer and
  $z$ is a non-negative integer, then

\medskip
$a'$ = \ans{$x$} and
$b'$ = \ans{$z$}

\item If $a > 0$ and $b > 0$, one of the two cases above will always
  be the case. Therefore, we can conclude which of the following about
  the values stored in \lstinline'a' and \lstinline'b' after an arbitrary
  iteration of the loop? (Check \textbf{\underline{all}} that apply)

\smallskip
\ansB{x} ~ $a' \geq 0$ and $b' \geq 0$

\smallskip
\ansB{x} ~ $a' > 0$ and $b' \geq 0$

\smallskip
\ansB{} ~ $a' \geq 0$ and $b' > 0$

\smallskip
\ansB{} ~ $a' > 0$ and $b' > 0$

\end{itemize}

\RUBRIC
Part (c)
TAGS: loop-invariant, testing

Gradescope rubric:
+ 0.25 pts First bullet: a' = 82, b' = 12
+ 0.25 pts Second bullet: a' = y, b' = x
+ 0.25 pts Third bullet: a' = x, b' = z
+ 0.25 pts Fourth bullet: exactly the first two should be checked (the most precise answer is a' > 0 and b' >= 0, but this implies that a' >= 0 and b' >= 0)
ENDRUBRIC
   % S19 S18 S17 F16 S16
\part[1]\TAGS{testing, loop-invariant}
Consider this loop:

\begin{lstlisting}
while (a > 0 || b > 0) {
    if (a < b) {
        b = b-a;
    } else {
        a = a-b;
    }
}
\end{lstlisting}

\begin{itemize}
\itemsep=2ex

\item If $a = 42$ and $b = 17$, then

\medskip
$a'$ = \ans{25} and
$b'$ = \ans{17}

\item If $a = x+y$ and $b = x$, where $x$ is a small positive integer and
    $y$ is a small non-negative integer, then

\medskip
$a'$ = \ans{$y$} and
$b'$ = \ans{$x$}

\item If $a = x$ and $b = x+z$, where $x$ and $z$ are both small
  positive integers, then

\medskip
$a'$ = \ans{$x$} and
$b'$ = \ans{$z$}

\item If $a > 0$ and $b > 0$, one of the two cases above will always
  be true. Therefore, we can conclude which of the following about
  the values stored in \lstinline'a' and \lstinline'b' after an arbitrary
  iteration of the loop? (Check \textbf{\underline{all}} that apply)

\smallskip
\ansB{x} ~ $a' \geq 0$ and $b' \geq 0$

\smallskip
\ansB{} ~ $a' > 0$ and $b' \geq 0$

\smallskip
\ansB{x} ~ $a' \geq 0$ and $b' > 0$

\smallskip
\ansB{} ~ $a' > 0$ and $b' > 0$

\end{itemize}

\RUBRIC
Part (c)
TAGS: loop-invariant, testing

Gradescope rubric:
+ 0.25 pts First bullet: a' = 25, b' = 17
+ 0.25 pts Second bullet: a' = y, b' = x
+ 0.25 pts Third bullet: a' = x, b' = z
+ 0.25 pts Fourth bullet: exactly the first and third should be checked (the most precise answer is a' >= 0 and b' > 0, but this implies that a' >= 0 and b' >= 0)
ENDRUBRIC
   % F19 F18

\end{parts}
