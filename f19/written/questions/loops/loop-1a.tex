\part[1]\TAGS{testing}
Consider the following loop:

\begin{lstlisting}
while (i < n) {
    k = j + k;
    j = j * 2 + i;
    i = i + 1;
}
\end{lstlisting}

\begin{itemize}

\item
If $i = 7$, $j = 3$, and $k = 9$, then assuming $7 < \texttt{n}$,

\medskip
$i'$ = \ans{8},
$j'$ = \ans{13}, and
$k'$ = \ans{12}

\bigskip
\item
If $i = 2y$, $j = x - y$, and $k = y$, then assuming $2y < \texttt{n}$,
in terms of $x$ and $y$,

\medskip
$i'$ = \ans{$2y + 1$},
$j'$ = \ans{$2x$}, and
$k'$ = \ans{$x$}

\bigskip
\item
If $j = k$, then assuming $i < \texttt{n}$,
in terms of $i$ and $k$,

\medskip
$i'$ = \ans{$i + 1$},
$j'$ = \ans{$2k + i$}, and
$k'$ = \ans{$2k$}

\bigskip
\item In general, assuming $i < n$, then in terms of $i$, $j$, and $k$,

\medskip
$i'$ = \ans{$i + 1$},
$j'$ = \ans{$2j + i$}, and
$k'$ = \ans{$j + k$}
\end{itemize}

\medskip
Note that we always say ``assuming (something) $<$ \texttt{n},''
because if that were not the case the loop wouldn't run, and it
wouldn't make any sense to be talking about the values of the primed
variables.

\RUBRIC
Part (a)
TAGS: testing

Gradescope rubric:
+ 0.25 pts First bullet: i' = 8, j' = 13, k' = 12
+ 0.25 pts Second bullet: i' = 2y + 1, j' = 2x, k' = x
+ 0.25 pts Third bullet: i' = i + 1, j' = 2k + i, k' = 2k
+ 0.25 pts Fourth bullet: i' = i + 1, j' = 2j + i, k' = j + k
ENDRUBRIC
