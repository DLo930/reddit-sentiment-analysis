\part[1]\TAGS{testing, loop-invariant}
Consider this loop:

\begin{lstlisting}
while (a > 0 || b > 0) {
    if (a < b) {
        b = b-a;
    } else {
        a = a-b;
    }
}
\end{lstlisting}

\begin{itemize}
\itemsep=2ex

\item If $a = 42$ and $b = 17$, then

\medskip
$a'$ = \ans{25} and
$b'$ = \ans{17}

\item If $a = x+y$ and $b = x$, where $x$ is a small positive integer and
    $y$ is a small non-negative integer, then

\medskip
$a'$ = \ans{$y$} and
$b'$ = \ans{$x$}

\item If $a = x$ and $b = x+z$, where $x$ and $z$ are both small
  positive integers, then

\medskip
$a'$ = \ans{$x$} and
$b'$ = \ans{$z$}

\item If $a > 0$ and $b > 0$, one of the two cases above will always
  be true. Therefore, we can conclude which of the following about
  the values stored in \lstinline'a' and \lstinline'b' after an arbitrary
  iteration of the loop? (Check \textbf{\underline{all}} that apply)

\smallskip
\ansB{x} ~ $a' \geq 0$ and $b' \geq 0$

\smallskip
\ansB{} ~ $a' > 0$ and $b' \geq 0$

\smallskip
\ansB{x} ~ $a' \geq 0$ and $b' > 0$

\smallskip
\ansB{} ~ $a' > 0$ and $b' > 0$

\end{itemize}

\RUBRIC
Part (c)
TAGS: loop-invariant, testing

Gradescope rubric:
+ 0.25 pts First bullet: a' = 25, b' = 17
+ 0.25 pts Second bullet: a' = y, b' = x
+ 0.25 pts Third bullet: a' = x, b' = z
+ 0.25 pts Fourth bullet: exactly the first and third should be checked (the most precise answer is a' >= 0 and b' > 0, but this implies that a' >= 0 and b' >= 0)
ENDRUBRIC
