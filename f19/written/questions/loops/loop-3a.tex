\part[1]\TAGS{loop-invariant, testing}
Consider this loop:

\begin{lstlisting}
while (a > 0 && b > 0) {
    if (a > b) {
        a = a-b;
    } else {
        b = b-a;
    }
}
\end{lstlisting}

\begin{itemize}
\itemsep=2ex

\item If $a = 94$ and $b = 12$, then

\medskip
$a'$ = \ans{82} and
$b'$ = \ans{12}

\item If $a = x+y$ and $b = x$, where $x$ and $y$ are both positive
  integers, then

\medskip
$a'$ = \ans{$y$} and
$b'$ = \ans{$x$}

\item If $a = x$ and $b = x+z$, where $x$ is a positive integer and
  $z$ is a non-negative integer, then

\medskip
$a'$ = \ans{$x$} and
$b'$ = \ans{$z$}

\item If $a > 0$ and $b > 0$, one of the two cases above will always
  be the case. Therefore, we can conclude which of the following about
  the values stored in \lstinline'a' and \lstinline'b' after an arbitrary
  iteration of the loop? (Check \textbf{\underline{all}} that apply)

\smallskip
\ansB{x} ~ $a' \geq 0$ and $b' \geq 0$

\smallskip
\ansB{x} ~ $a' > 0$ and $b' \geq 0$

\smallskip
\ansB{} ~ $a' \geq 0$ and $b' > 0$

\smallskip
\ansB{} ~ $a' > 0$ and $b' > 0$

\end{itemize}

\RUBRIC
Part (c)
TAGS: loop-invariant, testing

Gradescope rubric:
+ 0.25 pts First bullet: a' = 82, b' = 12
+ 0.25 pts Second bullet: a' = y, b' = x
+ 0.25 pts Third bullet: a' = x, b' = z
+ 0.25 pts Fourth bullet: exactly the first two should be checked (the most precise answer is a' > 0 and b' >= 0, but this implies that a' >= 0 and b' >= 0)
ENDRUBRIC
