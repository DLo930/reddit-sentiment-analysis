\clearpage
\Question[2]{Preconditions and Postconditions}\TAGS{correctness}

\bgroup
\newcommand{\ans}[2][1.5em]{\fbox{\rule[-0.25ex]{0em}{3ex}\answer{#1}{#2}}}

For the following functions, \textbf{\underline{either}} check the box
that says the postcondition always holds when the function is given
inputs that satisfy its preconditions \textbf{\underline{or}} give a
concrete counterexample: specific values of the inputs such that the
preconditions (if there are any) holds and the postcondition does not
hold. You don't have to write any proofs.

\begin{tabular}{p{0.45\textwidth}|@{\hspace{1.5em}}p{0.45\textwidth}}
\vspace{-12pt}
\begin{lstlisting}
int f1(int x, int y)
//@requires 0 < x && x <= y;
//@ensures \result < 0;
{
  return x - y;
}
\end{lstlisting}
\lstinline'@ensures' always true? ~ \ans{}


x = \ans[2.5em]{\texttt{0}} ~~  y = \ans[2.5em]{\texttt{x}}
&
\vspace{-12pt}
\begin{lstlisting}
int f2(int x)
//@requires x % 2 == 0;
//@ensures x >= 0 || \result > x;
{
  return x / 2;
}
\end{lstlisting}
\lstinline'@ensures' always true? ~ \ans{X}

x = \ans[2.5em]{}
\end{tabular}

\bigskip

\begin{tabular}{p{0.45\textwidth}|@{\hspace{1.5em}}p{0.45\textwidth}}
\vspace{-12pt}
\begin{lstlisting}
int f3(int x, int y)
//@requires y < -1;
//@ensures \result > y ;
{
  return x % y;
}
\end{lstlisting}
\lstinline'@ensures' always true? ~ \ans{X}

x = \ans[2.5em]{} ~~ y = \ans[2.5em]{}
&
\vspace{-12pt}
\begin{lstlisting}
int f4(int x, int y)
//@requires x + y == 42;
//@ensures \result - x == y;
{
  return 42;
}
\end{lstlisting}
\lstinline'@ensures' always true? ~ \ans{X}

x = \ans[2.5em]{} ~~ y = \ans[2.5em]{}
\end{tabular}

\bigskip

\begin{tabular}{p{0.45\textwidth}|@{\hspace{1.5em}}p{0.45\textwidth}}
\vspace{-12pt}
\begin{lstlisting}
int f5(int x, int y)
//@ensures \result > 0;
{
  if (x < 0) x = -x;
  if (y < 0) y = -y;
  if (y > x) {
    return y - x;
  } else {
    return x - y;
  }
}
\end{lstlisting}
\lstinline'@ensures' always true? ~ \ans{}

x = \ans[2.5em]{5} ~~ y = \ans[2.5em]{x}
&
\vspace{-12pt}
\begin{lstlisting}
int f6(int x, int y)
//@ensures \result <= 0;
{
  if (x <= 0) x = -x;
  if (y <= 0) y = -y;
  if (y >= x) {
    return y - x;
  } else {
    return x - y;
  }
}
\end{lstlisting}
\lstinline'@ensures' always true? ~ \ans{}

x = \ans[2.5em]{5} ~~ y = \ans[2.5em]{6}
\end{tabular}

\RUBRIC
TAGS: correctness

Gradescope rubric (cap at 2 points):
* 0.35 pts f1 is not always true; counterexample for all x >= 0, x == y
* 0.35 pts f2 is always true
* 0.35 pts f3 is always true
* 0.35 pts f4 is always true
* 0.35 pts f5 is not always true; counterexample is when x == y or x == -y
* 0.35 pts f6 is not always true; counterexample is when x != y and x != -y

Commentary:
Counterexamples are in general form in the rubric ("x != y and x != -y"),
but MUST be specific values when given by students or they don't get
credit for having a correct counterexample.

No credit for a function if they have things written in multiple boxes
(that is, if there's any ambiguity whether they meant to check the box
or write the concrete counterexample).
ENDRUBRIC

\egroup