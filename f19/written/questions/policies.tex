\clearpage
\Question{Policies}

\begin{parts}
\part[1]\TAGS{course-policies}
\textbf{Collaboration Policy}%

Read the collaboration policy on the course website. For each of the following
situations, decide whether or not the students' actions are permitted by the
policy. Explain your answers.

\enlargethispage{5ex}
\begin{enumerate}[(a)]
% \item%
%   Stephanie is really stuck on problem 3, and the deadline is looming.
%   She asks Frank, who took 15-122 two semesters ago, for help.  Frank
%   sends her his solution.  She changes the wording and submits.  Is
%   Stephanie in violation of the policy?  Is Frank?
% \begin{framed}
% \ifprintanswers{\color{\answerColor}
%   Both Stephanie and Frank are in violation.  Copying a homework
%   solution from another student is not allowed.  In all likelihood,
%   both students will be reported.
% }\else~\vspace{0.75in}\fi
% \end{framed}

\item%
  Tom is having a hard time installing C0 on his dorm computer.  The
  deadline for homework 1 is in less an hour.  He Skypes his friend
  Hyrum who walks him through the installation process.  With 15
  minutes to go, they get \lstinline'cc0' working and Tom manages to
  turn in 3 out of 4 exercises.
\begin{framed}
\ifprintanswers{\color{\answerColor}
Tom and Hyrum did not violate the policy: helping each other with the
infrastructure is allowed.
}\else~\vspace{0.75in}\fi
\end{framed}

% \item%
%   Frank and Penny sketch out a solution to Problem 2 on a whiteboard in
%   GHC\@.  Then they erase the whiteboard and go to class.  In the
%   evening, sitting at opposite sides of a computer cluster, each
%   student types up the solution.
% \begin{framed}
% \ifprintanswers{\color{\answerColor}
%   Frank and Penny are in compliance: the whiteboard policy allows to
%   collaborate on a problem as long as nothing is recorded and there is
%   enough before writing down the solution.
% }\else~\vspace{0.75in}\fi
% \end{framed}

% \item%
%   Penny and Tom write the solution code for Problem 2 on a whiteboard
%   in GHC\@.  Once they have the solution, they run to their respective
%   dorm rooms, types it up and submit. %
% \begin{framed}
% \ifprintanswers{\color{\answerColor}
%   Both Penny and Tom are in violation: writing code (or even
%   pseudocode) for a problem with another person is not allowed.
% }\else~\vspace{0.75in}\fi
% \end{framed}

\item%
  Stephanie and Anna are working on problem 5 on a whiteboard in the
  9th floor of GHC\@.  An hour later, they have come up with a complex
  solution.  They go their separate ways but Anna takes a picture so
  that she can improve the solution.  The next day, both type up their
  solution.
\begin{framed}
\ifprintanswers{\color{\answerColor}
  Stephanie is fine but Anna is in violation of the whiteboard policy:
  when working together, recording a solution is not allowed.
}\else~\vspace{0.75in}\fi
\end{framed}

\item%
  Andre is working on a problem alone on a whiteboard in the GHC
  commons.  He accidentally forgets to erase his solution and goes to a
  cluster to write it up.  Later, Tom walks by, reads Andre's notes,
  and then writes the solution when he gets home a few hours later.
  Is Tom in violation of the policy? Is Andre?
\begin{framed}
\ifprintanswers{\color{\answerColor}
  Both Andre and Tom are in violation: Andre was negligent for not
  cleaning up after himself and Tom should have walked away the moment
  he recognized it was a solution for a problem he was working on.
  (Yes, we can tell!)
}\else~\vspace{0.75in}\fi
\end{framed}

\item%
  Iliano is having a really hard time with 122.  So much so that he
  has hired an expert tutor, Rob, who goes over the lecture notes and
  the assignment writeups with him, nearly line by line.  As a result,
  he manage to get 74\% in the first homework. %
\begin{framed}
\ifprintanswers{\color{\answerColor}
  Iliano did not violate the policy: clarifying the phrasing of
  homeworks is allowed.  Had expert Rob done Iliano's homework (which
  is not allowed), he would have gotten much better than 74\%.
}\else~\vspace{0.75in}\fi
\end{framed}

% \item%
%   Rob is working on a tricky question shortly before the deadline and just
%   can't figure it out. To get a hint, he looks at the staff solution to the
%   problem from when his friend Hyrum took it last semester.
% \begin{framed}
% \ifprintanswers{\color{\answerColor}
%   Rob is in violation: looking up solutions from previous semesters is
%   not allowed.
% }\else~\vspace{0.75in}\fi
% \end{framed}

\item%
  Jean realizes that problem 2 is very similar to when she took 15-122
  last time, a semester ago.  She retrieves her old solution, fixes
  the one issue where she lost points, and submits.
\begin{framed}
\ifprintanswers{\color{\answerColor}
  Jean is in violation: reusing one's solution from a previous
  semester is not allowed.
}\else~\vspace{0.75in}\fi
\end{framed}

\end{enumerate}

\RUBRIC
Part (allowed behaviors)
TAGS: course-policies

Gradescope rubric:
+0.2pt Tom and Hyrum did not violate the policy: helping each other with the infrastructure is allowed.
+0.2pt Stephanie is fine but Anna is in violation of the whiteboard policy: when working together, recording a solution is not allowed.
+0.2pt Both Andre and Tom are in violation: Andre was negligent for not cleaning up after himself and Tom should have walked away the moment he recognized it was a solution for a problem he was working on.  (Yes, we can tell!)
+0.2pt Iliano did not violate the policy: clarifying the phrasing of homeworks is allowed.  Had expert Rob done Iliano's homework (which is not allowed), he would have gotten much better than 74\%.
+0.2pt Jean is in violation: reusing one's solution from a previous semester is not allowed.

Commentary:
* 1 points max
* -0.2 for each incorrect answer
ENDRUBRIC
%%% UNUSED PROBLEMS
%+0.2pt  Both Stephanie and Frank are in violation.  Copying a homework solution from another student is not allowed.  In all likelihood, both students will be reported.
%+0.2pt Both Penny and Tom are in violation: writing code (or even pseudocode) for a problem with another person is not allowed.
%+0.2 pts [STRICT POLICY] Both Frank and Penny are in violation: solving a problem with another person is not allowed (unless explicitly stated).
%+0.2 pts [WHITEBOARD POLICY] Frank and Penny are in compliance: the whiteboard policy allows to collaborate on a problem as long as nothing is recorded and there is enough before writing down the solution.
%+0.2 pts  Rob is in violation: looking up solutions from previous semesters is not allowed




\part[0\half]\TAGS{course-policies}
\textbf{MOSS}%

The course relies on a software service called MOSS to check for plagiarized
code.  What does MOSS do exactly?  A quick online search will bring up examples
of MOSS output, possibly for languages other than C/C0.  If a student
is pressed for time, what is likely to be least time consuming: writing
his/her own code or ``borrowing'' code from a friend and modifying it in
such a way that MOSS won't flag them as similar?  Explain.
\begin{framed}
\ifprintanswers{\color{\answerColor}
MOSS is a tool to identify similiarities in student code.

Deceiving MOSS is possible, but it typically takes quite some effort.
}\else~\vspace{1.5in}\fi
\end{framed}

\RUBRIC
Part (MOSS)
TAGS: course-policies

Gradescope rubric: (negative grading)
* -0.25 pts Incorrect or grossly inaccurate description of MOSS.
* -0.25 pts Unconvincing explanation. (It takes a lot of work to fool MOSS.)

Commentary:
* The purpose of this exercise is to get students to gain awareness about MOSS.
* The actual answers don't matter all that much as long has they put some thought into them.
ENDRUBRIC

\begin{comment}
\part[1]\TAGS{course-policies}
\textbf{Bonuses}%

Go to the class web page, read carefully the part that explains how the grades
are calculated, and answer the following questions:
\begin{enumerate}
\item%
  Assume you get 100\% on each homework assignment in the course.  How much
  extra credit will you receive if you submit every assignment three full days
  before the deadline.  Compute it as a percentage of the final grade for the
  course, whose maximum is 100\%.

\item%
  What if you consistently get 85\% in each homework assignments.
\end{enumerate}
Justify your answers

\begin{framed}
\ifprintanswers{\color{\answerColor}
1. 4.5\%

2. 4.49\%
}\else~\vspace{1in}\fi
\end{framed}

\RUBRIC
Part (academic integrity contract)
TAGS: course-policies

Gradescope rubric:
- 0.5 pts  5.4%.
- 0.5 pts  4.59%.

Commentary:
- The goals is for students to understand what they can get out of the bonus points.
- Accept answers that are in the ballpark (+/- 1%, say)
- The general formula is 2% / 12 hours * 2 * 12 hours / day * 3 days * 45\% ovrerall points for the assignmemnts.
ENDRUBRIC
\end{comment}

\bigskip
\part[0]{\bf Academic Integrity Contract}

Now that you had a chance to reflect on the collaboration policy of
the course, we ask you to complete and sign the contract on the next
page.  By doing this, you declare that you understand the course
policy on academic integrity and commit to abide by it.  Like any
contract, read it carefully.  Please reach out to the course staff if
you have any questions.

Although this tasks is worth 0 points, failure to complete and sign
the contract will carry a penalty of \textbf{-500 points}, i.e.,
guaranteed failure in the course.

\RUBRIC
Part (Academic Integrity contract)
TAGS: course-policies

Gradescope rubric:
* -500 pts  Contract is not filled and signed

Commentary:
- If student fails to complete the form, the penalty will be erased if
  he/she submits it within a reasonable amount of time.
ENDRUBRIC
\end{parts}

\newpage
\thispagestyle{empty}
\begin{figure}[p]
\vspace*{-5ex}%
\hspace*{-0.02\linewidth}%
\framebox[1.04\linewidth]{\centering
\begin{minipage}{\linewidth}
\setlength{\parskip}{1ex}

\medskip
\begin{center}\large\bf
15--122 --- Principles of Imperative Computation, \semester{}
\end{center}

The value of your degree depends on the academic integrity of yourself
and your peers in each of your classes. It is expected that, unless
otherwise instructed, the work you submit as your own will be your own
work and not someone else's work or a collaboration between yourself
and other(s).

Please read the
\href{http://www.cmu.edu/policies/documents/Academic\%20Integrity.htm}{University
  Policy on Academic Integrity} carefully to understand the penalties
associated with academic dishonesty at Carnegie Mellon. In this class,
cheating/copying/plagiarism means copying all or part of a program or
homework solution from another student or unauthorized source such as
the Internet, having someone else do a homework or take an exam for
you, knowingly giving such information to another student, reusing
answers or solutions from previous editions of the course, or giving
or receiving unauthorized information during an examination. In
general, \textbf{each solution you submit (quiz, written assignment,
  programming assignment, midterm or final exam) must be your own
  work}.  In the event that you use information written by another
person in your solution, you must cite the source of this information
(and receive prior permission if unsure whether this is permitted). It
is considered cheating to compare complete or partial answers,
% discuss details of %
copy or adapt others' solutions, read other students' code or show
your code to other students, or sit near another person who is taking
the same course and complete the assignment together.  Writing code on
a whiteboard is never permitted.  It is also considered cheating for a
repeating student to reuse one's solutions from a previous semester,
or any instructor-provided sample solution.

\textbf{It is a violation of this policy to hand in work for other students.}

Your course instructor reserves the right to determine an appropriate
penalty based on the violation of academic dishonesty that occurs.
\emph{Penalties are severe: a typical violation of the university policy
results in the student failing this course, but may go all the way to
expulsion from Carnegie Mellon University.}  If you have any questions
about this policy and any work you are doing in the course, please
feel free to contact your instructor for help.

We will be using the \href{https://theory.stanford.edu/~aiken/moss/}{Moss} system to detect software plagiarism.

It is not considered cheating to clarify vague points in the
assignments, lectures, lecture notes, or to give help or receive help
in using the computer systems, compilers, debuggers, profilers, or
other facilities, but you must refrain from looking at other students'
code while you are getting or receiving help for these tools. It is
not cheating to review graded assignments or exams with students in
the same class as you, but it is considered unauthorized assistance to
share these materials between different iterations of the course.  \textbf{Do
not post code from this course publicly (e.g., to Bitbucket or
GitHub).}

\bigskip

By signing below, I have read the statements above and have reviewed
the course policy for cheating and plagiarism, and I will abide by
these policies in this course.

\medskip
Andrew ID \uanswer{20.9em}{bovik}

\medskip
Name (print) \uanswer{20em}{Harry Bovik}\hfill Section \uanswer{8em}{X}

\medskip
Signature \uanswer{21.65em}{H.B.}\hfill Date \uanswer{8em}{1/1/1989}

\medskip
\end{minipage}}
\end{figure}
