\bgroup

\renewcommand{\a}
%{a} % up to S18
{u} % S19
\renewcommand{\b}
%{b} % up to S18
{v} % S19
\newcommand{\x}
%{x} % up to S18
{y} % S19

\begin{lstlisting}
int mystery(int [*\a*], int [*\b*])
//@requires [*\color{\contractColor}\a*] > 0 && [*\color{\contractColor}\b*] > 0;
{
    int [*\x*] = 0;
    for (int i = 1; i < 10; i++) {
        int j = [*\a*];

        while (j > 0) {
            int k = [*\b*];

            while (k > 0) {
                [*\x*] = [*\x*] + i * j * k;
                k = k / 2;
            }

            j--;
        }
    }

    return [*\x*];
}
\end{lstlisting}

\begin{framed}\vspace{0.25in}
$O(\uanswer{16em}{$\a \log \b$})$
\end{framed}

\RUBRIC
Part (mystery A)
TAGS: big-o

Gradescope rubric:
+ 0.5 pts Big-O is in simplest form: (O(u log v), not O(u log_2 v)
+ 0.5 pts Big-O is correct and in tightest form

Commentary:
- check name of variables (varies from semester to semester)
. 1/2 pt if student mixes up a and b or gets part of it right
  (e.g. O(ab)).
. 1/2 pt if student has O(10 * a log b) since 10 doesn't matter
  here, or if they include the base 2 e.g. O(a log_2 b)
ENDRUBRIC

\egroup