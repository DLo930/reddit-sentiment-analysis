\bgroup
\renewcommand{\a}
{a} % up to F18
\renewcommand{\b}
{b} % up to F18
\newcommand{\A}
{A} % up to F18
\newcommand{\n}
{25} % up to F18

\begin{lstlisting}
int mystery(int[] [*\A*], int [*\a*], int [*\b*])
//@requires 0 <= [*\color{\contractColor}\a*] && [*\color{\contractColor}\a*] < \length([*\color{\contractColor}\A*]);
//@requires 0 <= [*\color{\contractColor}\b*] && [*\color{\contractColor}\b*] < \length([*\color{\contractColor}\A*]);
{
    int r = 0;
    for (int i = 1; i < [*\n*]; i++) {
        int j = [*\a*];

        while (j > 0) {
            int k = [*\b*];

            while (k > 0) {
                r = r + i * [*\A*][j] * [*\A*][k];
                k--;
            }

            j = j / 2;
        }
    }

    return r;
}
\end{lstlisting}

\begin{framed}\vspace{0.25in}
$O(\uanswer{16em}{$b \log a$})$
\end{framed}

\RUBRIC
Part (mystery B)
TAGS: big-o

Gradescope rubric:
+0.5pt Big-O is in simplest form: O(b log a), not O(b log_2 a)
+0.5pt Big-O is correct and in tightest form

Commentary:
Grading: 1 pt for correct answer with no unnecessary constants
. 1/2 pt if student mixes up a and b or gets part of it right
  (e.g. O(ab)).
. 1/2 pt if student has O(25 * b log a) since 25 doesn't matter
  here, or if they include the base 2 e.g. O(b log_2 a)
ENDRUBRIC

\egroup