\clearpage
\Question{Timing Code}

The following run times were obtained when using two different
algorithms on a data set of size $n$. You are asked to extrapolate the
asymptotic complexity of these algorithms based on this time data.
Determine the asymptotic complexity of each algorithm as a function of
$n$.  Use big-O notation in its tightest form and briefly explain how
you reached the conclusion.

\begin{parts}
\renewcommand{\arraystretch}{1.2}
\part[1]\TAGS{complexity, testing}
\hspace*{5em}
\begin{tabular}[t]{r@{\hspace{2em}}l}
   $n$   & Execution Time
\\\hline
   1000 & 0.564 milliseconds
\\ 2000 & 2.271 milliseconds
\\ 4000 & 8.992 milliseconds
\\ 8000 & 36.150 milliseconds
\end{tabular}

\smallskip
\begin{framed}\textbf{Asymptotic complexity: }
$O(\uanswer{10em}{$n^2$})$

\medskip
\ifprintanswers{\color{\answerColor}
  The time grows more or less quadratically with $n$.
}\else~\vspace{2in}\fi
\end{framed}

\RUBRIC
Part (a)
TAGS: complexity, testing

Gradescope rubric:
+0.5pt O(n^2)
+0.5pt reasonable explanation

Commentary:
ENDRUBRIC


\enlargethispage{5ex}
\part[1]\TAGS{complexity, testing}
\hspace*{5em}
\begin{tabular}[t]{r@{\hspace{2em}}l}
   $n$        & Execution Time
\\\hline
   1000       & 0.043 milliseconds
\\ 1000000    & 43.68 milliseconds
\\ 1000000000 & 43.9 seconds
\end{tabular}

\smallskip
\begin{framed}\textbf{Asymptotic complexity: }
$O(\uanswer{10em}{$n$})$

\medskip
\ifprintanswers{\color{\answerColor}
  The time grows more or less linearly with $n$.
}\else~\vspace{2in}\fi
\end{framed}

\RUBRIC
Part (b)
TAGS: complexity, testing

Gradescope rubric:
+0.5pt O(n)
+0.5pt reasonable explanation

Commentary:
ENDRUBRIC

\end{parts}
