\clearpage
\Question{The Preservation of Loop Invariants}\TAGS{loop-invariant}

The core of proving the correctness of a function with one loop is
proving that the loop invariant is \emph{preserved} --- that if the loop
invariant holds at the beginning of an iteration (just before
the loop guard is tested), it still holds at the end of that iteration
(just before the loop guard is tested the next time).

For each of the following loops, state whether the loop invariant is
ALWAYS PRESERVED or NOT ALWAYS PRESERVED. If you say that the loop
invariant is always preserved, prove it using point-to reasoning.  If
you say that the loop invariant is not always preserved, give a
\emph{specific counterexample}. When we ask for a counterexample, what
we mean is that we want \emph{specific, concrete} values of the local
variables such that the loop guard and loop invariant will hold before
the loop body executes for some iteration, but where the loop
invariant will not hold after the loop body executes that one
iteration.

Here are two solved examples to give you an idea of how to write your
solutions. Integers are defined as C0's 32-bit signed two's-complement
numbers; be careful about this when you think about counterexamples!

%\paragraph{Example:}~
%\part[0]~\vspace{-2ex}
\begin{lstlisting}[numbers=left]
while (x <= y)
//@loop_invariant x < y;
{
    x = x + 1;
}
\end{lstlisting}
\begin{framed}\textbf{Solution:}
NOT ALWAYS PRESERVED

Counterexample: x=2 and y=3, satisfies loop invariant and loop guard.

After this iteration, x=3 and y=3, violating loop invariant.
\end{framed}
\enlargethispage{5ex}
\begin{lstlisting}[numbers=left]
while (x + 1 < y)
//@loop_invariant x < y + 1;
{
    x = x + 2;
}
\end{lstlisting}
\begin{framed}\textbf{Solution:}
ALWAYS PRESERVED.

\newcounter{lineCounterC}
\newcommand{\aLine}[3][0.5ex]
{\stepcounter{lineCounterC}%
 \noindent\makebox[1em]{\alph{lineCounterC})~~}\makebox[10em]{#2\hfill}~by
 \makebox[20em]{#3\hfill}\\[#1]%
}%

Assume \lstinline'x < y + 1' (by line 2) before an iteration.
We must show \lstinline"x"$'$ \lstinline" < y + 1" after an iteration.

Since \lstinline"x"$'$ \lstinline" = x + 2" (by line 4), we need to show
\lstinline"x + 2 < y + 1".

\smallskip
\setcounter{lineCounterC}{0}
\aLine{\lstinline'x + 1 < y'}{line 1}
\aLine{\lstinline'x + 2 <= y'}{math (because \lstinline'x + 1 < y')}
\aLine{\lstinline'y < y + 1'}{line 2 that lets us know \lstinline'y != int_max()'}
\aLine[-2ex]{\lstinline'x + 2 < y + 1'}{(b) and (c)}

\end{framed}

% %\paragraph{Example:}~
%\part[0]~\vspace{-2ex}
\begin{lstlisting}[numbers=left]
while(j < 10000)
//@loop_invariant 2 * i == j;
{
  i = i+2;
  j = j+4;
}
\end{lstlisting}
\begin{framed}\textbf{Solution:}
ALWAYS PRESERVED.

% \emph{Long version}: From line \lstinline'1', we know that \lstinline'j < 10000'
% at the beginning of the loop and from line \lstinline'2' we know \lstinline'2 * i' =
% \lstinline'j' at the beginning of the loop.

% We use primed variables to refer to the values stored in \lstinline'i' and
% \lstinline'j' at the end of the loop. Therefore, to show that the loop
% invariant is preserved, we need to show that \lstinline"2 * i'" = \lstinline"j'"
% where \lstinline"i'" = \lstinline"i+2" (line \lstinline'4') and \lstinline"j'" =
% \lstinline"j+4" (line \lstinline'5').

% Therefore we have that \lstinline"2 * i'" = \lstinline"2 * (i+2)" =
% \lstinline"2 * (i+2)" = \lstinline"2*i + 4" = \lstinline"2*i + 4" = \lstinline"j + 4" =
% \lstinline"j'", which by transitivity is what we needed to show.

% \emph{Short version}:

Assume \lstinline'2 * i == j' (line 2) before an iteration.
We must show \lstinline'2 * i` == j`' after an iteration.

Since \lstinline'i` = i + 2' (line 4)
and \lstinline'j` = j + 4' (line 5), we need to show that
\lstinline'2 * (i+2) ==  j+4'.

\begin{tabbing}
~ -- ~~ \lstinline'2 * i`' \== \lstinline'2 * (i+2)'\qquad\= (line 4)\\
~ -- ~~ \phantom{\lstinline'2 * (i+2)'}\> = \lstinline'2*i + 4'        \> (distributivity)\\
~ -- ~~ \phantom{\lstinline'2*i + 4'}  \> = \lstinline'j + 4'          \> (line 2)\\
~ -- ~~ \phantom{\lstinline'j + 4'}    \> = \lstinline'j`'             \> (line 5)\\
~ -- ~~ \lstinline'2 * i`'   \> = \lstinline'j`'             \> (transitivity, the four preceding facts)
\end{tabbing}

\end{framed}


\RUBRIC
In general, it's okay if they're clear that it's always preserved
(giving a proof) or not always preserved (giving a counterexample)
even if they don't write that explicitly.

TAGS: loop-invariant
ENDRUBRIC

\begin{parts}

\newpage
\part[0\half]~\vspace{-2ex}
\begin{lstlisting}[numbers=left]
while (x < y && x <= 15122)
//@loop_invariant x <= y;
{
    if (0 <= z && z < 10) {
        x = x + z;
    }
}
\end{lstlisting}
\begin{framed}
\newcommand{\ans}[1]{\fbox{\rule[-0.5ex]{0em}{3ex}\answer{5.5em}{\texttt{#1}}}}
NOT ALWAYS PRESERVED

\bigskip
Counterexample: $x$ = \ans{-4} , $y$ = \ans{4} , $z$ = \ans{9}.

The loop invariant and loop guard are satisfied at the start of the
iteration but the loop invariant is not satisfied at the end of that
iteration.
\end{framed}

\RUBRIC
Part (<n1>)

Gradescope rubric:
+0.5 pts Valid counterexample [Ex: -4, 4, 9]

Commentary:
 - Counterexample: 0 <= z < 10, and x and y differ by less than z
   for example:
     (x = -4, y = 4, z = 9)
     (x = 15121, y = 15122, z = 2)

  All or nothing.
ENDRUBRIC
 % S19 F18 S18 F17 S17 S16
%\newpage
\part[0\half]~\vspace{-2ex}
\begin{lstlisting}[numbers=left]
while (i <= x)[*\label{l:lpres-yes1-lg}*]
//@loop_invariant x < y;[*\label{l:lpres-yes1-LI1}*]
//@loop_invariant i <= y;[*\label{l:lpres-yes1-LI2}*]
{
    i++;[*\label{l:lpres-yes1-ipp}*]
}
\end{lstlisting}
\begin{framed}
\newcommand{\ans}[2]{\fbox{\rule[-0.5ex]{0em}{3ex}\answer{#1}{#2}}}
ALWAYS PRESERVED

\bigskip
The first loop invariant is always preserved because
\uanswer{10.8em}{\texttt{x} and \texttt{y} remain constant\hfill}

\medskip
\uanswer{33.5em}{\hfill}.

\bigskip\medskip
For the second loop invariant, we assume that $i \le y$
and want to show that $i' \le y'$ (or equivalently $i' \le y$ since $y$ does not change in the loop).

\bigskip
Using operational reasoning for one iteration:

\bigskip
By line~\ref{l:lpres-yes1-ipp}, \quad $i' = \ans{6em}{\texttt{i+1}}$.

\bigskip
By line~\ref{l:lpres-yes1-LI1}, \quad $x+1 \le \ans{6em}{\texttt{y}}$.

\bigskip
By line~\ref{l:lpres-yes1-lg}, \quad $ \ans{6em}{\texttt{i+1}} \le x+1$.

\bigskip
The previous three statements taken together imply that $i' \le y$.
\end{framed}

\RUBRIC
Part (<y1>)

Gradescope rubric:
+0.25pt Box 1: x and y remain constant, so the loop invariant is trivially preserved
+0.25pt Boxes 2-4: i+1, (i+1 OR i'), y
-0.1pt Boxes 2-4: one mistake

Commentary:
  First line: x and y remain constant.
  Three lines for i <= y:
  first blank: i + 1
  second blank: i + 1
  third blank: y

  Half point for first line about x < y.
  Half point for three lines about i <= y. All or nothing.
ENDRUBRIC
 % S19 F18 S18 F17 S17 F16 S16

\newpage
%\part[0\half] In this example,
you are using two functions with the following declarations:

\medskip

\begin{lstlisting}[numbers=left]
bool f(int x);
int mid(int lo, int hi)
  /*@requires 0 <= lo && lo < hi; @*/[*\label{l:lpres-yes2-mid-pre}*]
  /*@ensures lo <= \result && \result < hi; @*/ ;[*\label{l:lpres-yes2-mid-post}*]
\end{lstlisting}

\medskip

That is, \lstinline'mid(lo, hi)' takes two integers and returns an
integer in the non-empty range \lstinline'[lo, hi)'.  The function
\lstinline'f(x)' takes an integer and returns a boolean; we don't know
anything about its return value, so we reason about both cases.

Now consider the following code that uses functions \lstinline'f' and
\lstinline'mid':

\medskip
\begin{lstlisting}[numbers=left, firstnumber=11]
while (lo < hi)
//@loop_invariant 0 <= lo && lo <= hi;[*\label{l:lpres-yes2-LI}*]
{
    m = mid(lo, hi);
    if (f(m)) {[*\label{l:lpres-yes2-if}*]
        lo = m+1;[*\label{l:lpres-yes2-lo}*]
    } else {
        hi = m;[*\label{l:lpres-yes2-hi}*]
    }
}
\end{lstlisting}

\begin{framed}
ALWAYS PRESERVED (Complete the indicated parts of the proof)

\bigskip
Assume: \hfill\uanswer{4.8in}{\lstinline'0 <= lo && lo <= hi'}

\medskip
To show: \hfill\uanswer{4.8in}{\lstinline'0 <= lo` && lo` <= hi`'}

\bigskip
\textbf{Case 1:} \lstinline'f(m)' returns \lstinline'true'

\medskip
By lines~\ref{l:lpres-yes2-if} and~\ref{l:lpres-yes2-lo}, %
\lstinline'lo`' = \hfill\uanswer{3.8in}{\lstinline'm+1'}

\medskip
By line~\ref{l:lpres-yes2-if}, \lstinline'hi`' = %
\hfill\uanswer{4.3in}{\lstinline'hi'}

\medskip
Therefore

\smallskip
\ifprintanswers{\color{\answerColor}
\begin{itemize}
\item
  \lstinline'lo <= m' by line~\ref{l:lpres-yes2-mid-post}, %
  \lstinline'0 <= m' by line~\ref{l:lpres-yes2-LI}, and %
  \lstinline'0 <= m+1' by line~\ref{l:lpres-yes2-mid-post}.  %
  Thus, \lstinline'0 <= lo`'.
\item
  \lstinline'm < hi' by line~\ref{l:lpres-yes2-mid-post} and %
  \lstinline'm+1 <= hi' by math. %
  Thus, \lstinline'lo` <= hi`'.
\end{itemize}
}\else~\vspace{1.0in}\fi

\textbf{Case 2:} \lstinline'f(m)' returns \lstinline'false'

\medskip
By line~\ref{l:lpres-yes2-if}, \lstinline'lo`' =
\hfill\uanswer{4.3in}{\lstinline'lo'}

\medskip
By lines~\ref{l:lpres-yes2-if} and~\ref{l:lpres-yes2-hi}, %
\lstinline'hi`' = \hfill\uanswer{3.7in}{\lstinline'm'}

\medskip
Therefore\ldots
\emph{(Skip this, as it looks much like the previous case)}
\end{framed}

\RUBRIC
Part (<y2>)

Gradescope rubric:
+ 0.1 pts Correct assumption (0 <= lo && lo <= hi) and to-show (0 <= lo' && lo' <= hi')
+ 0.1 pts Case 1: Correct lo' = m+1, hi' = hi
+ 0.2 pts Case 1: Proof correctly combines the postcondition of mid (line 4) and the loop invariant (line 12)
+ 0.1 pts Case 2: Correct lo' = lo, hi' = m

Commentary:
 Blanks:
   1/2 point for these two together:
   Assume:                       0 <= lo && lo <= hi
   To show:                     0 <= lo' && lo' <= hi'

  1/2 point for these two lines together:
   By lines 15 and 16, lo' = m+1
   By line 15, hi' =         hi

  1/2 point for completing "Therefore..."
   BOTH the loop invariant (line 12) and the postcondition to mid (line 4) are
   necessary to show that the postcondition holds. It's okay if they
   identify them unambiguously as "postcondition to mid" and "loop
   invariant" even if they don't refer to the line numbers.

  1/2 point for these two lines together:
   By line 15, lo' =         lo
   By lines 15 and 18, hi' = m
ENDRUBRIC
 % F18 F17 F16 S16
\part[0\half]~\vspace{-2ex}
\begin{lstlisting}[numbers=left, firstnumber=11]
while (a != b)
//@loop_invariant a > 0 && b > 0;[*\label{l:lpres-yes6-LI}*]
{
  if (a > b) {[*\label{l:lpres-yes6-if}*]
    a = a - b;[*\label{l:lpres-yes6-seta}*]
  } else {
    b = b - a;
  }
}
\end{lstlisting}

\begin{framed}
ALWAYS PRESERVED (Complete the indicated parts of the proof)

\bigskip
We reason by case analysis on the relationship between the integers
\lstinline'a' and \lstinline'b'.

\begin{description}
\item[Case 1, ] \lstinline'(a > b)':
\ifprintanswers{\color{\answerColor}
\begin{enumerate}[(a)]
\newcommand{\ans}[2]{\item\makebox[10em][l]{#1} by #2}
\ans{\lstinline'a` = a - b'}{lines~\ref{l:lpres-yes6-if}, \ref{l:lpres-yes6-seta}}
\ans{\lstinline'b` = b'}{line~\ref{l:lpres-yes6-if}}
\ans{\lstinline'b > 0'}{line~\ref{l:lpres-yes6-LI}}
\ans{\lstinline'a` > 0'}{assumption and (1)}
\ans{\lstinline'b` > 0'}{(b) and (c)}
\ans{\lstinline'a` > 0 \&\& b` > 0'}{math}
\end{enumerate}
}\else~\vspace{3in}\fi


\item[Case 2, ] \lstinline'(a < b)': similar (trust us!)

\item[Case 3, ] \lstinline'(a == b)':

  Because we know \lstinline'a != b' (line \lstinline'11'), this case
  is impossible.
\end{description}
\end{framed}

\RUBRIC
Part (<y6>)

Gradescope rubric:
+0.5pts Case 1 is correct except for (*)
-0.25pts (*) no mention of b'

Commentary:
  - We give them the answer here, so roughly 0.25 point for each proof.

  - 0.25 point overall if they say *nothing* about b' in Case 1 or
    nothing about a' in Case 2 but otherwise have correct reasoning
    about a' in Case 1 and b' in Case 2.

  - It's fine if they don't include line 4 as part of the
    justification.

  - It's fine if they don't make the final step of saying that a' > 0
    && b' > 0 as long as they've justified that a' > 0 and b' > 0.

  - Case 1 (a > b)
       - a' = a - b          (line 4, line 5)
       - b' = b              (line 4)
       - b > 0               (line 2)
       - a' > 0              (a > b) && (a' = a - b) => a' > 0
       - b' > 0              (b > 0) && (b' = b) => b' > 0
       - a' > 0 && b' > 0    (previous two facts)

  - Case 2 (a < b)
       - a' = a              (line 4)
       - b' = b - a          (line 4, line 7)
       - a > 0               (line 2)
       - a' > 0              (a > 0) && (a' = a) => a' > 0
       - b' > 0              (a < b) && (b' = b - a) => b' > 0
       - a' > 0 && b' > 0    (previous two facts)
ENDRUBRIC
 % S19 S18 S17

\newpage
%\part[0\half]~\vspace{-2ex}
\begin{lstlisting}[numbers=left]
while (x < y)
//@loop_invariant x <= y;
{
    x = x + z;
}
\end{lstlisting}
\begin{framed}
\ifprintanswers{\color{\answerColor}
  NOT ALWAYS PRESERVED.

  Counterexample: x=2, y=3, z=2.
}\else~\vspace{2.5in}\fi
\end{framed}

\RUBRIC
Part (<n1b>)

Gradescope rubric:
+ 1 pt Valid counterexample [Ex: 2, 3, 2]

Commentary:
  not always preserved.  x = 24, y = 25, z = 2 (for example)
ENDRUBRIC

%%%%%%% break and continue are not supported any more in C0.
\part[0\half]~\vspace{-2ex}
\begin{lstlisting}[numbers=left]
while (i < 24)
//@loop_invariant i == 2*j;
{
  if (i % 5 == 0){
    break;
  }
  i++;
  if(i % 7 == 0){
    continue;
  }
  j += 2;
}
\end{lstlisting}
\enlargethispage{5ex}
\begin{framed}
\ifprintanswers{\color{\answerColor}
  NOT ALWAYS PRESERVED.

  Counterexample: i=7, j=14.
}\else~\vspace{2.1in}\fi
\end{framed}

\RUBRIC
Part (<n5>)

Gradescope rubric:
+ 1 pt Valid counterexample [Ex: 14, 7]

Commentary:
  not always preserved.  i = 7, j = 14
ENDRUBRIC
 %%% uses break and continue!

%\newpage
%\part[0\half]~\vspace{-2ex}
\begin{lstlisting}[numbers=left]
while (a > 1)
//@loop_invariant a >= 1;
{
    if (a % 2 == 0) {
       a = a / 2;
    } else {
       a = 3 * a + 1;
    }
}
\end{lstlisting}
\begin{framed}
\ifprintanswers{\color{\answerColor}
  NOT ALWAYS PRESERVED.

  Counterexample: 715,827,883 <= a <= 1,431,655,765
}\else~\vspace{2.0in}\fi
\end{framed}

\RUBRIC
Part (<n2>)

Gradescope rubric:
+ 0.25 pts NOT ALWAYS PRESERVED
+ 0.25 pts Valid counterexample [Ex: 1 billion and 1]

Commentary:
  - Counterexample:
    715,827,883 <= a <= 1,431,655,765
    (0x2AAAAAAB <= a <= 0x55555555)
    int_max() is NOT a counterexample!

 Half a point for NOT ALWAYS PRESERVED, half a point for the right
 counterexample.
ENDRUBRIC
 % S18 S17 S16
\part[0\half]~\vspace{-2ex}
\begin{lstlisting}[numbers=left]
while (i < 24)
//@loop_invariant 0 <= i;
//@loop_invariant 2*i == j;
{
  i++;
  if (i % 7 > 0) {
    j += 2;
  }
}
\end{lstlisting}
\begin{framed}
\ifprintanswers{\color{\answerColor}
  NOT ALWAYS PRESERVED.

  Counterexample: i=6, j=12.
}\else~\vspace{2.1in}\fi
\end{framed}

\RUBRIC
Part (<n3>)

Gradescope rubric:
+ 0.25 pts NOT ALWAYS PRESERVED
+ 0.25 pts Valid conterexample [Ex i=6, j=12]

Commentary:
  ONLY i = 6,  j = 12
       i = 13, j = 26
       i = 20, j = 40

  Half a point for NOT ALWAYS PRESERVED, half a point for counterexample
ENDRUBRIC
 % F19 S18 S17 S16

%\newpage
%\part[0\half]~\vspace{-2ex}
\begin{lstlisting}[numbers=left]
while (i < 24)
//@loop_invariant 2*i == j;
{
    i++;
    if (i % 7 != 4) {
        j += 2;
    }
}
\end{lstlisting}
\begin{framed}
\ifprintanswers{\color{\answerColor}
  NOT ALWAYS PRESERVED.

  Counterexample: i=3, j=6.
}\else~\vspace{2.1in}\fi
\end{framed}

\RUBRIC
Part (<n3b>)

Gradescope rubric:
+0.25pt NOT ALWAYS PRESERVED
+0.25pt Valid conterexample [Ex i=3, j=6]

Commentary:
NOT ALWAYS PRESERVED
  ONLY i = 3,  j = 6
       i = 10, j = 20
       i = 17, j = 34

  Half a point for NOT ALWAYS PRESERVED, half a point for
  counterexample
ENDRUBRIC
  % F18 F17 F16
\part[0\half]~\vspace{-2ex}
\begin{lstlisting}[numbers=left]
while (k <= n)
//@loop_invariant i*i == k;[*\label{l:lpres-yes5-LI}*]
{
  k = k + 2*i + 1;[*\label{l:lpres-yes5-setk}*]
  i = i + 1;[*\label{l:lpres-yes5-seti}*]
}
\end{lstlisting}
\begin{framed}
\newcommand{\ans}[2]{\item\makebox[16em][l]{#1} by #2}
\ifprintanswers{\color{\answerColor}
  ALWAYS PRESERVED.

\begin{enumerate}[(a)]
  \ans{\lstinline'k`' = \lstinline'k + 2*i + 1'}{line~\ref{l:lpres-yes5-setk}}
  \ans{\lstinline'k + 2*i + 1' = \lstinline'i*i + 2*i + 1'}{line~\ref{l:lpres-yes5-LI}}
  \ans{\lstinline'i*i + 2*i + 1' = \lstinline'(i+1)*(i+1)'}{math}
  \ans{\lstinline'(i+1)*(i+1)' = \lstinline'i`*i`'}{line~\ref{l:lpres-yes5-seti}}
  \ans{so \lstinline'i`*i` == k`'}{(a--d)}
  \end{enumerate}
}\else~\vspace{2.5in}\fi
\end{framed}

\RUBRIC
Part (<y5>)

Gradescope rubric:
+0.25pt ALWAYS PRESERVED
+0.25pt Valid proof showing that i'*i' == k'

Commentary:
The loop invariant is always preserved.
  - k' = k + 2*i + 1                 (line 4)
  - k + 2*i + 1 = i*i + 2*i + 1      (line 2)
  - i*i + 2*i + 1 = (i + 1)*(i + 1)  (math)
  - (i + 1)*(i + 1) = i'*i'          (line 5)
  - i'*i' = k                        (transitivity, the last 4 facts)
ENDRUBRIC
 % S19 F17 F16

%\newpage
%\part[0\half]~\vspace{-2ex}
\begin{lstlisting}[numbers=left]
while (a != b)
//@loop_invariant a > b || b > a;
{
    if (a > b) {
        a = a - b;
    } else {
        b = b - a;
    }
}
\end{lstlisting}
\begin{framed}
\ifprintanswers{\color{\answerColor}
  NOT ALWAYS PRESERVED.

  Counterexample: a=2b or b=2a.
}\else~\vspace{2.1in}\fi
\end{framed}

\RUBRIC
Part (<n4>)

Gradescope rubric:
+ 0.25 pts NOT ALWAYS PRESERVED
+ 0.25 pts Valid conterexample [Ex a=2b, b=2a]

Commentary:
  If a and b are positive, only counterexamples a = 2b or b = 2a
  (may be wonky for counterexamples involving negative numbers...)

  Half a point for NOT ALWAYS PRESERVED, half a point for counterexample
ENDRUBRIC
  % F18 F16
%\part[0\half]~\vspace{-2ex}
\begin{lstlisting}[numbers=left]
while (0 <= b && b < a)
//@loop_invariant a % 17 == 0 && b % 17 == 0;[*\label{l:lpres-yes3-LI}*]
{
    a = a - b;[*\label{l:lpres-yes3-seta}*]
}
\end{lstlisting}
\begin{framed}
\ifprintanswers{\color{\answerColor}
  ALWAYS PRESERVED.

  \begin{itemize}
  \item \lstinline'a' = $17n$ by line~\ref{l:lpres-yes3-LI}
  \item \lstinline'b' = $17m$ by line~\ref{l:lpres-yes3-LI}
  \item \lstinline'a`' = $17(m - n)$ by line~\ref{l:lpres-yes3-seta}
  \item so \lstinline'a` % 17' = $17(m-n) \% 17 == 0$
  \end{itemize}

  \lstinline'b`' = \lstinline'b',
  so \lstinline'b` % 17' = 0 follows directly from line~\ref{l:lpres-yes3-LI}.
}\else~\vspace{2.1in}\fi
\end{framed}

\RUBRIC
Part (<y3>)

Gradescope rubric:
+ 0.5 pts ALWAYS PRESERVED
+ 0.5 pts Valid proof

Commentary:
  a = 17n (line 2)
  b = 17m (line 2)
  a' = 17(m - n) (line 4)
   ==> a' % 17 == 17(m-n) % 17 == 0

  b' = b, so b' % 17 == 0 follows directly from line 2 (if they leave
  this out it's okay)

  Half a point for ALWAYS PRESERVED, half a point for proof

  Proof can be pretty loose, as long as it seems like they're trying
  to say somthing like "both are multiples of 17, so their difference
  must also be a multiple of 17", it is fine.
ENDRUBRIC
 % S18 S16

\newpage
\part[0\half]~\vspace{-2ex}
\begin{lstlisting}[numbers=left]
while(0 < j && j <= 1000)
//@loop_invariant j % 2 == (1 - i);[*\label{l:lpres-yes4-LI}*]
{
  j = j + 3;[*\label{l:lpres-yes4-seti}*]
  i = (i + 1) % 2;[*\label{l:lpres-yes4-setj}*]
}
\end{lstlisting}
\begin{framed}
\newcommand{\ans}[2]{\item\makebox[10em][l]{#1} by #2}
\ifprintanswers{\color{\answerColor}
  ALWAYS PRESERVED.

  Case \lstinline'j' is even at the beginning of the loop iteration:
  \begin{enumerate}[~~~~~~(a)]
  \itemsep=-0.5ex

  \ans{\lstinline'1 - i' = 0}{line~\ref{l:lpres-yes4-LI}}
  \ans{\lstinline'i' = 1}{math}
  \ans{\lstinline'j`' = \lstinline'j+3'}{line~\ref{l:lpres-yes4-setj}}
  \ans{\lstinline'j`\%2' = 1}{assumption and math}
  \ans{\lstinline'i`' = \lstinline'(i+1)\%2' = 1}{line~\ref{l:lpres-yes4-seti}}
  \ans{so \lstinline'i`' = \lstinline'j`\%2' = 1}{(d) and (e)}
  \end{enumerate}

  Case \lstinline'j' is odd at the beginning of the loop iteration:
  \begin{enumerate}[~~~~~~(a)]
  \itemsep=-0.5ex
  \ans{\lstinline'1 - i' = 1}{line~\ref{l:lpres-yes4-LI}}
  \ans{\lstinline'i' = 0}{math}
  \ans{\lstinline'j`' = \lstinline'j+3'}{line~\ref{l:lpres-yes4-setj}}
  \ans{\lstinline'j`\%2' = 0}{assumption and math}
  \ans{\lstinline'i`' = \lstinline'(i+1)\%2' = 0}{line~\ref{l:lpres-yes4-seti}}
  \ans{so \lstinline'i`' = \lstinline'j`\%2' = 0}{(d) and (e)}
  \vspace{-2ex}
  \end{enumerate}

}\else~\vspace{2.5in}\fi
\end{framed}

\RUBRIC
Part (<y4>)

Gradescope rubric:
+ 0.5 pts ALWAYS PRESERVED
+ 0.5 pts Valid proof

Commentary:
The loop invariant is always preserved.
(Case on j odd and j even at the beginning of the loop iteration)
Case 1: j%2 = 0
   1 - i = 0 (Line 2)
=> i = 1
   j' = j + 3 (Line 4)
=> j'%2 = 1
   i' = (i + 1)%2 (Line 5)
=> i' = 1 = j'%2

Case 2: j%2 = 1
   1 - i = 1 (Line 2)
=> i = 0
   j' = j + 3 (Line 4)
=> j'%2 = 0
   i' = (i + 1)%2 (Line 5)
=> i' = 0 = j'%2
ENDRUBRIC
 % S19 F17
\part[0\half]%~\vspace{-2ex}
In this task, you know nothing about what \lstinline'f' computes.
\begin{lstlisting}[numbers=left]
while (a < 700)
//@loop_invariant a + b == f(a,b);
{
    int c = f(a,b);
    a += 1;
    b = c - a - 1;
}
\end{lstlisting}
\begin{framed}
\ifprintanswers{\color{\answerColor}
  NOT ALWAYS PRESERVED.

  Counterexample: a=0, b=0, f(a,b)=a*b.
}\else~\vspace{2.5in}\fi
\end{framed}

\RUBRIC
Part (<n5>)

Gradescope rubric:
+ 1 pt Valid counterexample [Ex: 0, 0, f(a,b)=a*b]

Commentary:
  NOT always preserved: a = 0, b = 0, f(a,b) = a. a' = 1, b' = -1.
  For any counterexample, it must be the case that f(a,b) = a+b, that
  f(a+1,b-1) != a+b.
ENDRUBRIC
  % S19 S18

%\newpage
%\part[0\half]~\vspace{-2ex}
\begin{lstlisting}[numbers=left]
while (e > 0)
//@loop_invariant e > 0 || accum == POW(x,y);
{
  accum = accum * x;
  e = e - 1;
}
\end{lstlisting}
\begin{framed}
\ifprintanswers{\color{\answerColor}
  NOT ALWAYS PRESERVED.

  Counterexample: 3=1, accum=0, x=10, y=10.
}\else~\vspace{2.5in}\fi
\end{framed}

\RUBRIC
Part (<n7>)

Gradescope rubric:
+0.5pt Valid counterexample (ex: e = 1, accum = 0, x = 10, y = 10)

Commentary:
  NOT always preserved
  - Almost any case where e = 1 produces a counterexample.

   - If they figure that out but don't give values for accum and x and
     y, only give half a point.
ENDRUBRIC
  % F18 F17
%\part[0\half]~\vspace{-2ex}
\begin{lstlisting}[numbers=left]
while (x == 2*y)[*\label{l:lpres-yes7-lg}*]
//@loop_invariant i == 4*j;[*\label{l:lpres-yes7-LI}*]
{
  i = i+2*x;[*\label{l:lpres-yes7-seti}*]
  j = j+y;[*\label{l:lpres-yes7-setj}*]
  x = f(i);[*\label{l:lpres-yes7-setx}*]
}
\end{lstlisting}
\begin{framed}
\ifprintanswers{\color{\answerColor}
  ALWAYS PRESERVED.

  \begin{itemize}
  \item \lstinline'i`' = \lstinline'i + 2*x' by line~\ref{l:lpres-yes7-seti}
  \item \lstinline'i`' = \lstinline'i + 4*y' by line~\ref{l:lpres-yes7-lg}
  \item \lstinline'i`' = \lstinline'4*j + 4*y' by line~\ref{l:lpres-yes7-LI}
  \item \lstinline'j`' = \lstinline'j + y' by line~\ref{l:lpres-yes7-setj}
  \item so \lstinline'i` == 4*j`'
  \end{itemize}
}\else~\vspace{2.5in}\fi
\end{framed}

\RUBRIC
Part (<y7>)

Gradescope rubric:
+ 0.25 pts ALWAYS PRESERVED
+ 0.25 pts Valid proof

Commentary:
The loop invariant is always preserved.
  - i' = i + 2*x             (line 4)
  - i + 2*x = 4*j + 2*x      (line 2)
  - 4*j + 2*x = 4*j + 4*y    (line 1)
  - 4*j + 4*y = 4*(j + y)    (distributivity)
  - 4*(j + y) = 4*j'         (line 5)
  - i' = 4*j'                (transitivity, the last 5 facts)
ENDRUBRIC
 % F18 S17

\end{parts}