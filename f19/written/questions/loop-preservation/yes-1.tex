\part[0\half]~\vspace{-2ex}
\begin{lstlisting}[numbers=left]
while (i <= x)[*\label{l:lpres-yes1-lg}*]
//@loop_invariant x < y;[*\label{l:lpres-yes1-LI1}*]
//@loop_invariant i <= y;[*\label{l:lpres-yes1-LI2}*]
{
    i++;[*\label{l:lpres-yes1-ipp}*]
}
\end{lstlisting}
\begin{framed}
\newcommand{\ans}[2]{\fbox{\rule[-0.5ex]{0em}{3ex}\answer{#1}{#2}}}
ALWAYS PRESERVED

\bigskip
The first loop invariant is always preserved because
\uanswer{10.8em}{\texttt{x} and \texttt{y} remain constant\hfill}

\medskip
\uanswer{33.5em}{\hfill}.

\bigskip\medskip
For the second loop invariant, we assume that $i \le y$
and want to show that $i' \le y'$ (or equivalently $i' \le y$ since $y$ does not change in the loop).

\bigskip
Using operational reasoning for one iteration:

\bigskip
By line~\ref{l:lpres-yes1-ipp}, \quad $i' = \ans{6em}{\texttt{i+1}}$.

\bigskip
By line~\ref{l:lpres-yes1-LI1}, \quad $x+1 \le \ans{6em}{\texttt{y}}$.

\bigskip
By line~\ref{l:lpres-yes1-lg}, \quad $ \ans{6em}{\texttt{i+1}} \le x+1$.

\bigskip
The previous three statements taken together imply that $i' \le y$.
\end{framed}

\RUBRIC
Part (<y1>)

Gradescope rubric:
+0.25pt Box 1: x and y remain constant, so the loop invariant is trivially preserved
+0.25pt Boxes 2-4: i+1, (i+1 OR i'), y
-0.1pt Boxes 2-4: one mistake

Commentary:
  First line: x and y remain constant.
  Three lines for i <= y:
  first blank: i + 1
  second blank: i + 1
  third blank: y

  Half point for first line about x < y.
  Half point for three lines about i <= y. All or nothing.
ENDRUBRIC
