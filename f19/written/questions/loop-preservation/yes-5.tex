\part[0\half]~\vspace{-2ex}
\begin{lstlisting}[numbers=left]
while (k <= n)
//@loop_invariant i*i == k;[*\label{l:lpres-yes5-LI}*]
{
  k = k + 2*i + 1;[*\label{l:lpres-yes5-setk}*]
  i = i + 1;[*\label{l:lpres-yes5-seti}*]
}
\end{lstlisting}
\begin{framed}
\newcommand{\ans}[2]{\item\makebox[16em][l]{#1} by #2}
\ifprintanswers{\color{\answerColor}
  ALWAYS PRESERVED.

\begin{enumerate}[(a)]
  \ans{\lstinline'k`' = \lstinline'k + 2*i + 1'}{line~\ref{l:lpres-yes5-setk}}
  \ans{\lstinline'k + 2*i + 1' = \lstinline'i*i + 2*i + 1'}{line~\ref{l:lpres-yes5-LI}}
  \ans{\lstinline'i*i + 2*i + 1' = \lstinline'(i+1)*(i+1)'}{math}
  \ans{\lstinline'(i+1)*(i+1)' = \lstinline'i`*i`'}{line~\ref{l:lpres-yes5-seti}}
  \ans{so \lstinline'i`*i` == k`'}{(a--d)}
  \end{enumerate}
}\else~\vspace{2.5in}\fi
\end{framed}

\RUBRIC
Part (<y5>)

Gradescope rubric:
+0.25pt ALWAYS PRESERVED
+0.25pt Valid proof showing that i'*i' == k'

Commentary:
The loop invariant is always preserved.
  - k' = k + 2*i + 1                 (line 4)
  - k + 2*i + 1 = i*i + 2*i + 1      (line 2)
  - i*i + 2*i + 1 = (i + 1)*(i + 1)  (math)
  - (i + 1)*(i + 1) = i'*i'          (line 5)
  - i'*i' = k                        (transitivity, the last 4 facts)
ENDRUBRIC
