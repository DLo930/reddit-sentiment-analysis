\part[0\half] In this example,
you are using two functions with the following declarations:

\medskip

\begin{lstlisting}[numbers=left]
bool f(int x);
int mid(int lo, int hi)
  /*@requires 0 <= lo && lo < hi; @*/[*\label{l:lpres-yes2-mid-pre}*]
  /*@ensures lo <= \result && \result < hi; @*/ ;[*\label{l:lpres-yes2-mid-post}*]
\end{lstlisting}

\medskip

That is, \lstinline'mid(lo, hi)' takes two integers and returns an
integer in the non-empty range \lstinline'[lo, hi)'.  The function
\lstinline'f(x)' takes an integer and returns a boolean; we don't know
anything about its return value, so we reason about both cases.

Now consider the following code that uses functions \lstinline'f' and
\lstinline'mid':

\medskip
\begin{lstlisting}[numbers=left, firstnumber=11]
while (lo < hi)
//@loop_invariant 0 <= lo && lo <= hi;[*\label{l:lpres-yes2-LI}*]
{
    m = mid(lo, hi);
    if (f(m)) {[*\label{l:lpres-yes2-if}*]
        lo = m+1;[*\label{l:lpres-yes2-lo}*]
    } else {
        hi = m;[*\label{l:lpres-yes2-hi}*]
    }
}
\end{lstlisting}

\begin{framed}
ALWAYS PRESERVED (Complete the indicated parts of the proof)

\bigskip
Assume: \hfill\uanswer{4.8in}{\lstinline'0 <= lo && lo <= hi'}

\medskip
To show: \hfill\uanswer{4.8in}{\lstinline'0 <= lo` && lo` <= hi`'}

\bigskip
\textbf{Case 1:} \lstinline'f(m)' returns \lstinline'true'

\medskip
By lines~\ref{l:lpres-yes2-if} and~\ref{l:lpres-yes2-lo}, %
\lstinline'lo`' = \hfill\uanswer{3.8in}{\lstinline'm+1'}

\medskip
By line~\ref{l:lpres-yes2-if}, \lstinline'hi`' = %
\hfill\uanswer{4.3in}{\lstinline'hi'}

\medskip
Therefore

\smallskip
\ifprintanswers{\color{\answerColor}
\begin{itemize}
\item
  \lstinline'lo <= m' by line~\ref{l:lpres-yes2-mid-post}, %
  \lstinline'0 <= m' by line~\ref{l:lpres-yes2-LI}, and %
  \lstinline'0 <= m+1' by line~\ref{l:lpres-yes2-mid-post}.  %
  Thus, \lstinline'0 <= lo`'.
\item
  \lstinline'm < hi' by line~\ref{l:lpres-yes2-mid-post} and %
  \lstinline'm+1 <= hi' by math. %
  Thus, \lstinline'lo` <= hi`'.
\end{itemize}
}\else~\vspace{1.0in}\fi

\textbf{Case 2:} \lstinline'f(m)' returns \lstinline'false'

\medskip
By line~\ref{l:lpres-yes2-if}, \lstinline'lo`' =
\hfill\uanswer{4.3in}{\lstinline'lo'}

\medskip
By lines~\ref{l:lpres-yes2-if} and~\ref{l:lpres-yes2-hi}, %
\lstinline'hi`' = \hfill\uanswer{3.7in}{\lstinline'm'}

\medskip
Therefore\ldots
\emph{(Skip this, as it looks much like the previous case)}
\end{framed}

\RUBRIC
Part (<y2>)

Gradescope rubric:
+ 0.1 pts Correct assumption (0 <= lo && lo <= hi) and to-show (0 <= lo' && lo' <= hi')
+ 0.1 pts Case 1: Correct lo' = m+1, hi' = hi
+ 0.2 pts Case 1: Proof correctly combines the postcondition of mid (line 4) and the loop invariant (line 12)
+ 0.1 pts Case 2: Correct lo' = lo, hi' = m

Commentary:
 Blanks:
   1/2 point for these two together:
   Assume:                       0 <= lo && lo <= hi
   To show:                     0 <= lo' && lo' <= hi'

  1/2 point for these two lines together:
   By lines 15 and 16, lo' = m+1
   By line 15, hi' =         hi

  1/2 point for completing "Therefore..."
   BOTH the loop invariant (line 12) and the postcondition to mid (line 4) are
   necessary to show that the postcondition holds. It's okay if they
   identify them unambiguously as "postcondition to mid" and "loop
   invariant" even if they don't refer to the line numbers.

  1/2 point for these two lines together:
   By line 15, lo' =         lo
   By lines 15 and 18, hi' = m
ENDRUBRIC
