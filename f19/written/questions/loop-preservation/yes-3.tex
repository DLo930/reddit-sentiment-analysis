\part[0\half]~\vspace{-2ex}
\begin{lstlisting}[numbers=left]
while (0 <= b && b < a)
//@loop_invariant a % 17 == 0 && b % 17 == 0;[*\label{l:lpres-yes3-LI}*]
{
    a = a - b;[*\label{l:lpres-yes3-seta}*]
}
\end{lstlisting}
\begin{framed}
\ifprintanswers{\color{\answerColor}
  ALWAYS PRESERVED.

  \begin{itemize}
  \item \lstinline'a' = $17n$ by line~\ref{l:lpres-yes3-LI}
  \item \lstinline'b' = $17m$ by line~\ref{l:lpres-yes3-LI}
  \item \lstinline'a`' = $17(m - n)$ by line~\ref{l:lpres-yes3-seta}
  \item so \lstinline'a` % 17' = $17(m-n) \% 17 == 0$
  \end{itemize}

  \lstinline'b`' = \lstinline'b',
  so \lstinline'b` % 17' = 0 follows directly from line~\ref{l:lpres-yes3-LI}.
}\else~\vspace{2.1in}\fi
\end{framed}

\RUBRIC
Part (<y3>)

Gradescope rubric:
+ 0.5 pts ALWAYS PRESERVED
+ 0.5 pts Valid proof

Commentary:
  a = 17n (line 2)
  b = 17m (line 2)
  a' = 17(m - n) (line 4)
   ==> a' % 17 == 17(m-n) % 17 == 0

  b' = b, so b' % 17 == 0 follows directly from line 2 (if they leave
  this out it's okay)

  Half a point for ALWAYS PRESERVED, half a point for proof

  Proof can be pretty loose, as long as it seems like they're trying
  to say somthing like "both are multiples of 17, so their difference
  must also be a multiple of 17", it is fine.
ENDRUBRIC
