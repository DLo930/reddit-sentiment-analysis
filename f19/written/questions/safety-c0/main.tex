\clearpage
\Question{Safety in C0}\TAGS{safety}

We'll talk a lot in this class about proving that contracts
(preconditions, postconditions, annotations, and loop invariants) will
always evaluate to \lstinline'true'. This is important, because it's how we
prove a function correct.

Before we can even talk about correctness, though, we want to use our
contracts to think about \emph{safety}. There are five kinds of
safety violations that we've talked about so far in class:

\begin{itemize}
\item giving a function call arguments that violate its preconditions;
\item division or modulo by zero;
\item bitshifting an integer left or right by less than zero or more than
  31;
\item allocating an array with negative length; and
\item accessing an array out of bounds.
\end{itemize}

Whenever we have an operation that's potentially unsafe, we must be
able to point to contracts that ensure its safety \emph{without
  reasoning about multiple iterations of any loop at once}. That means
we may use the following facts:
\begin{itemize}
\item%
  When local variables are \emph{untouched} by a loop, statements we
  know to be true about those locals \emph{before} the loop remain
  valid \emph{inside} the loop and \emph{after} the loop.

\item%
  For local variables that are modified by the loop, the loop guard
  and the loop invariants are the only statements we can use.
  \begin{itemize}
  \item%
    Inside of a loop, we know that the loop invariants held just before the
    loop guard was checked and that the loop guard returned \lstinline'true'.
  \item%
    After a loop, we know that the loop invariants held just before the loop
    guard was checked for the last time and that the loop guard returned
    \lstinline'false'.
  \end{itemize}
\end{itemize}
We call this \emph{point-to reasoning}.  That's because we show that
our argument is supported by pointing to some lines of code.

For each of the problems below, \textbf{state whether the safety of each
  potentially unsafe operation is SUPPORTED or UNSUPPORTED given the existing
  contracts, and briefly explain your reasoning}.  You can assume that all
loop invariants are true initially (before the loop guard is checked the first
time) and that they are preserved by any iteration of the loop. If you claim
that the annotation is supported, your answer should be a concise proof; if you
claim that the annotation is unsupported, we only expect an informal argument
to explain why.

We've given two examples below.


\newpage
\begin{lstlisting}[numbers=left, aboveskip=0pt, belowskip=0pt]
void fill(int x, int[] A)
/*@requires x == \length(A); @*/;

int[] f(int x, int[] B)
//@requires 1 <= x && x < \length(B);
{
    int i = 1;
    int[] A = alloc_array(int, x);

    while (i < x)
    //@loop_invariant i >= 1;
    {
        B[i] = B[i] - 3;
        i += 1;
    }

    fill(i, A);
    return A;
}
\end{lstlisting}

\begin{framed}
\newcounter{lineCounterB}
\newcommand{\aLine}[3][0.8ex]
{\stepcounter{lineCounterB}%
 \noindent\makebox[1em]{\alph{lineCounterB})~~}\makebox[10em]{#2\hfill}~by
 \makebox[15em]{#3\hfill}\\[#1]%
}%

Safety of the array access on line 13 is: \underline{SUPPORTED}.

To show: \lstinline'0 <= i && i < \length(B)'

\medskip
\setcounter{lineCounterB}{0}
\aLine{\lstinline'i >= 1'}{line 11 (loop invariant)}
\aLine{\lstinline'0 <= i'}{math on (a) (because \lstinline'i >= 1')}
\aLine{\lstinline'i < x'}{line 10 (loop guard)}
\aLine{\texttt{x < \length(B)}}{line 5 (precondition)}
\aLine[-2ex]{\tt i < \length(B)}{(c) and (d)}
\end{framed}

\begin{framed}
Safety of the function call on line 17 is: \underline{UNSUPPORTED}.

We know by line 8 that \lstinline'\length(A) == x', so for the
function call to be safe we need to know \lstinline'i == x'.

By line 10, we know that the loop guard \lstinline'i < x' is false ---
that is, we know that \lstinline'!(i < x)', which is the same thing as
saying \lstinline'i >= x'. We can't conclude, from this, that \lstinline'i' is
equal to \lstinline'x'.
\end{framed}

With a different loop invariant on line 11, safety of the
function call on line 17 \textbf{would have been
  supported}. You'll demonstrate this in the next question.

\smallskip
That means that, even though we have a function call whose precondition will
never fail, our loop invariants aren't good enough for us to conclude (using
only point-to reasoning based on existing contracts) that the function call is
safe!


\begin{parts}

\newcounter{lineCounter}
\newcommand{\aLine}[3][1.5ex]
{\stepcounter{lineCounter}%
 \noindent\makebox[1em]{\alph{lineCounter})~~}\uanswer{21em}{#2}\hfill~by
 \uanswer{10em}{#3\hfill}\\[#1]%
}%
\newcommand{\noans}[1]{\textbf{#1}}
\newcommand{\ans}[1]{%
  \ifprintanswers%
  \textcolor{\answerColor}{\fbox{\noans{#1}}}%
  \else\noans{#1}\fi}

\RUBRIC
TAGS: safety
ENDRUBRIC

\newpage\part[1]~\vspace*{-2ex}
\begin{lstlisting}[numbers=left]
int f(int a, int b)
//@requires 1 <= a && a <= b;[*\label{l:safe-assert1-prec}*]
{
    int i = 1;
    while (i < a)[*\label{l:safe-assert1-lg}*]
    //@loop_invariant 1 <= i && i <= b;[*\label{l:safe-assert1-LI}*]
    {
        //@assert 0 <= i && i < b;          /*** Assertion A ***/[*\label{l:safe-assert1A}*]
        i += 1;
    }
    //@assert i <= b;                       /*** Assertion B ***/[*\label{l:safe-assert1B}*]
    return i;
}
\end{lstlisting}

\vfill
\begin{framed}
\medskip
\lstinline'Assertion A' is: \uanswer{26.2em}{SUPPORTED}

\medskip
\aLine{\texttt{1 <= i}}{line~\ref{l:safe-assert1-LI}}
\aLine{\texttt{0 <= i}}{math}
\aLine{\texttt{i < a}}{line~\ref{l:safe-assert1-lg}}
\aLine{\texttt{a <= b}}{line~\ref{l:safe-assert1-prec}}
\aLine{\texttt{i < b}}{math}
Therefore we \ans{can}/\noans{cannot} conclude that\\[0.6ex]
\aLine[-2ex]{\texttt{0 <= i \&\& i < b}}{above}
\end{framed}

\RUBRIC
Part 1A

Gradescope rubric:  TODO

Commentary:
Assertion A is SUPPORTED
   - 1 <= i from
   - i < b from lines 5 and 2 - (i < a) && (a <= b) => i < b
ENDRUBRIC

\vfill
\begin{framed}
\medskip
\lstinline'Assertion B' is: \uanswer{26.2em}{SUPPORTED}

\medskip
\aLine{\texttt{i <= b}}{line~\ref{l:safe-assert1-LI}}
\aLine{}{}
\aLine{}{}
\aLine{}{}
\aLine{}{}
Therefore we \ans{can}/\noans{cannot} conclude that\\[0.6ex]
\aLine[-2ex]{\texttt{i <= b}}{line~\ref{l:safe-assert1B}}
\end{framed}

\RUBRIC
Part 1B

Gradescope rubric:  TODO

Commentary:
Assertion B is SUPPORTED
   - i <= b from line 6
ENDRUBRIC % S16 (Goes together with example-1)
\newpage\part[1]~\vspace*{-2ex}
\begin{lstlisting}[numbers=left, belowskip=0pt]
int gap(int x, int y)
/*@requires x <= y; @*/ ;

int g(int a)
//@requires 0 <= a;
{
    int i = 2*a;

    while (i > a)[*\label{l:gapex2-lg}*]
    //@loop_invariant i >= a;[*\label{l:gapex2-LI}*]
    {
        int[] A = alloc_array(int, i-1);[*\label{l:gapex2-alloc_array}*]
        a += 2;[*\label{l:gapex2-upd-a}*]
        i += 1;[*\label{l:gapex2-upd-i}*]
    }

    return gap(i, a);[*\label{l:gapex2-gap}*]
}
\end{lstlisting}

\vfill
\begin{framed}
\medskip
Safety of the array allocation on line~\ref{l:gapex2-alloc_array} is:
\uanswer{15em}{UNSUPPORTED}

\medskip
\setcounter{lineCounter}{0}
\aLine{\texttt{i >= a}}{line~\ref{l:gapex2-LI}}
\aLine{\texttt{i} and \texttt{a} are modified in the
  loop}{lines~\ref{l:gapex2-upd-i} and~\ref{l:gapex2-upd-a}}
\aLine{}{}
\aLine{}{}
Therefore we \noans{can}/\ans{cannot} conclude that\\[0.6ex]
\aLine[-2ex]{\texttt{i > 0}}{~~~~N/A}
\end{framed}

\vfill
\begin{framed}
\medskip
Safety of the function call on line~\ref{l:gapex2-gap} is: \uanswer{16.5em}{SUPPORTED}

\medskip
\setcounter{lineCounter}{0}
\aLine{\texttt{i <= a}}{line~\ref{l:gapex2-lg}}
\aLine{}{}
\aLine{}{}
\aLine{}{}
Therefore we \ans{can}/\noans{cannot} conclude that\\[0.6ex]
\aLine[-2ex]{\texttt{i <= a}}{line~\ref{l:gapex2-gap}}
\end{framed}

\RUBRIC
Part (c)

Gradescope rubric:
+0.25pt First box: UNSUPPORTED
+0.25pt First box: correctly justified (array size could be negative)
+0.25pt Second box: SUPPORTED
+0.25pt Second box: correctly justified (negation of loop guard on line 9)
-0.1pt Second box: unnecessarily use of line 10

Commentary:
- Line 12 is UNSUPPORTED
  . i >= a (line 10)
  . i and a are modified in the loop (lines 13 and 14)
  . This does not allow us to conclude that i > 0 at line 12.
- Line 17 is SUPPORTED
  . i <= a by line 9 (negation of the exit condition)
    [If they cite line 10 unnecessarily in their justification, this is
    incorrect: we want people to give us the minimal set of line
    numbers necessary to justify a conclusion.]

ENDRUBRIC
 % S16
\newpage\part[1]~\vspace*{-2ex}
\begin{lstlisting}[numbers=left]
int f(int n) {
    int x = 0;
    while (n >= 2) {[*\label{l:safe-assert3-lg}*]
        x += 1;
        if (n % 2 == 1) {
            n = 3*n + 1;
        } else {
            n = n/2;
        }
    }
    //@assert n <= 2;   /*** Assertion A ***/[*\label{l:safe-assert3A}*]
    //@assert x >= 0;   /*** Assertion B ***/[*\label{l:safe-assert3B}*]
    return x;
}
\end{lstlisting}

\vfill
\begin{framed}
\medskip
\lstinline'Assertion A' is: \uanswer{26.2em}{SUPPORTED}

\medskip
\aLine{\texttt{n < 2}}{line~\ref{l:safe-assert3-lg}}
\aLine{}{}
\aLine{}{}
\aLine{}{}
\aLine{}{}
Therefore we \ans{can}/\noans{cannot} conclude that\\[0.6ex]
\aLine[-2ex]{\texttt{n <= 2}}{math}
\end{framed}

\RUBRIC
Part 3A

Gradescope rubric:
+ 0.2 pts SUPPORTED
+ 0.15 pts n < 2 by line 3 (negation of the loop guard)
+ 0.15 pts Therefore n <= 2

Commentary:
Assertion A is SUPPORTED
  - The negation of the loop guard gives n < 2, which implies n <= 2
ENDRUBRIC

\vfill
\begin{framed}
\medskip
\lstinline'Assertion B' is: \uanswer{26.2em}{UNSUPPORTED}

\medskip
\aLine{\texttt{x} is modifed by the loop}{}
\aLine{}{}
\aLine{}{}
\aLine{}{}
\aLine{}{}
Therefore we \noans{can}/\ans{cannot} conclude that\\[0.6ex]
\aLine[-2ex]{\texttt{x >= 0}}{line~\ref{l:safe-assert3B}}
\end{framed}

\RUBRIC
Part 3B

Gradescope rubric:
+ 0.2 UNSUPPORTED
+ 0.3 x is modified by the loop and there are no loop invariants: logical reasoning gives no information about it.

Commentary:
Assertion B is UNSUPPORTED
  - Because x is modified by the loop and unmentioned by the loop
    guard and the (trivial) loop invariant, we don't know anything
    about x after the loop body runs.
ENDRUBRIC
 % S16

\end{parts}
