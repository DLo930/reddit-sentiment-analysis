\part[0\half]
This is the \textbf{exact same code} from the previous example, but
because there is a different loop invariant, the answers, which are
given, have changed. You just have to provide explanations.

\smallskip
\begin{lstlisting}[numbers=left, aboveskip=0pt, belowskip=0pt]
void fill(int x, int[] A)
/*@requires x == \length(A); @*/;

int[] f(int x, int[] B)
//@requires 1 <= x && x < \length(B);
{
    int i = 1;
    int[] A = alloc_array(int, x);[*\label{l:fillex1-alloc_array}*]

    while (i < x)[*\label{l:fillex1-lg}*]
    //@loop_invariant i <= \length(A);[*\label{l:fillex1-LI}*]  // MODIFIED
    {
        B[i] = B[i] - 3;[*\label{l:fillex1-A[i]}*]
        i += 1;
    }

    fill(i, A);[*\label{l:fillex1-fill}*]
    return A;
}
\end{lstlisting}

\enlargethispage{5ex}\vfill
\begin{framed}
Safety of the array access on line~\ref{l:fillex1-A[i]} is:
\underline{UNSUPPORTED}.

\medskip
\setcounter{lineCounter}{0}
\aLine{We have no direct support for \lstinline'0 <= i'.\hfill}{}
\aLine{}{}
\aLine{}{}
\aLine{}{}
Therefore we \textbf{cannot} conclude that\\[0.6ex]
\aLine[-2.5ex]{\lstinline'0 <= i'}{~~~~N/A}
\end{framed}

\vfill
\begin{framed}
Safety of the function call on line~\ref{l:fillex1-fill} is:
\underline{SUPPORTED}.

\medskip
\setcounter{lineCounter}{0}
\aLine{i >= x}{line~\ref{l:fillex1-lg}}
\aLine{x = \length{(A)}}{line~\ref{l:fillex1-alloc_array}}
\aLine{i <= \length{(A)}}{line~\ref{l:fillex1-LI}}
\aLine{}{}
Therefore we \textbf{can} conclude that\\[0.6ex]
\aLine[-2.5ex]{i == \length{(A)}}{line~\ref{l:fillex1-fill}}
\end{framed}

\vfill

\RUBRIC
Part (b)

Gradescope rubric:
+0.25pt First: we have no support for 0 <= i
+0.25pt Second box: right idea (combines <= and >= to get ==)
+0.25pt Second box: fully correct, uses lines 8, 10 ("loop guard"), and 11 ("loop invariant")

Commentary:
- Line 13 is UNSUPPORTED
  . We have no direct support for 0 <= i
    [it must be the case that 0 <=i on the basis of operational
    reasoning, but not logical reasoning.]
- Line 17 is SUPPORTED
  . i >= x by line 10 (negation of loop invariant)
  . x = \length(A) by line 8
  . i <= \length(A) by line 11
  . So i == \length(A) on line 17

1 point per justification. 1/2 pt credit if partially correct.
ENDRUBRIC
