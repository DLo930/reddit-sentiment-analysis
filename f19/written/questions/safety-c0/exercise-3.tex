\part[1]~\vspace*{-2ex}
\begin{lstlisting}[numbers=left, belowskip=0pt]
int h(int n) {
    int x = 1;
    while (n >= 25) {[*\label{l:hex3-lg}*]
        x += 1;
        if (n % 2 == 1) {
            n = 3*n + 1;
        } else {
            n = n/2;
        }
    }

    if (n >= 0) {[*\label{l:hex3-if}*]
        return x << n;[*\label{l:hex3-shift}*]
    }

    return 1000000 / x;[*\label{l:hex3-div}*]
}
\end{lstlisting}

\vfill
\begin{framed}
\medskip
Safety of the left-shift on line~\ref{l:hex3-shift} is:
\uanswer{18em}{SUPPORTED}

\medskip
\setcounter{lineCounter}{0}
\aLine{\texttt{0 <= n}}{line~\ref{l:hex3-if}}
\aLine{\texttt{n < 25}}{line~\ref{l:hex3-lg}}
\aLine{}{}
\aLine{}{}
Therefore we \ans{can}/\noans{cannot} conclude that\\[0.6ex]
\aLine[-2ex]{\texttt{0 <= n < 32}}{line~\ref{l:hex3-shift}}
\end{framed}

\vfill
\begin{framed}
\medskip
Safety of the division on line~\ref{l:hex3-div} is:
\uanswer{18.5em}{UNSUPPORTED}

\medskip
\setcounter{lineCounter}{0}
\aLine{We know nothing about \texttt{x} after the loop.}{N/A}
\aLine{}{}
\aLine{}{}
\aLine{}{}
Therefore we \noans{can}/\ans{cannot} conclude that\\[0.6ex]
\aLine[-2ex]{\texttt{x != 0}}{~~~~N/A}
\end{framed}

\RUBRIC
Part (d)

Gradescope rubric:
+0.25pt First box: SUPPORTED
+0.25pt First box: Correctly justified (Negation of loop guard on line 3 and condition on line 12)
+0.25pt Second box: UNSUPPORTED
+0.25pt Second box: Correctly justified (There isn't evidence that tells us that x != 0)

Commentary:
- Line 13 is SUPPORTED
  . 0 <= n by line 12
  . n < 25 by line 3
  . So 0 <= n < 32 on line 13
- Line 16 is UNSUPPORTED
  . We know nothing about x after the loop.
  . So we can't conclude that x != 0
  . [Concrete counterexample: make a call to h(1)

Half point per (UN)SUPPORTED, if they get that right they can get
a full point for the justification (so each part is 1.5 points).
ENDRUBRIC
