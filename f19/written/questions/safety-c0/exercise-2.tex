\part[1]~\vspace*{-2ex}
\begin{lstlisting}[numbers=left, belowskip=0pt]
int gap(int x, int y)
/*@requires x <= y; @*/ ;

int g(int a)
//@requires 0 <= a;
{
    int i = 2*a;

    while (i > a)[*\label{l:gapex2-lg}*]
    //@loop_invariant i >= a;[*\label{l:gapex2-LI}*]
    {
        int[] A = alloc_array(int, i-1);[*\label{l:gapex2-alloc_array}*]
        a += 2;[*\label{l:gapex2-upd-a}*]
        i += 1;[*\label{l:gapex2-upd-i}*]
    }

    return gap(i, a);[*\label{l:gapex2-gap}*]
}
\end{lstlisting}

\vfill
\begin{framed}
\medskip
Safety of the array allocation on line~\ref{l:gapex2-alloc_array} is:
\uanswer{15em}{UNSUPPORTED}

\medskip
\setcounter{lineCounter}{0}
\aLine{\texttt{i >= a}}{line~\ref{l:gapex2-LI}}
\aLine{\texttt{i} and \texttt{a} are modified in the
  loop}{lines~\ref{l:gapex2-upd-i} and~\ref{l:gapex2-upd-a}}
\aLine{}{}
\aLine{}{}
Therefore we \noans{can}/\ans{cannot} conclude that\\[0.6ex]
\aLine[-2ex]{\texttt{i > 0}}{~~~~N/A}
\end{framed}

\vfill
\begin{framed}
\medskip
Safety of the function call on line~\ref{l:gapex2-gap} is: \uanswer{16.5em}{SUPPORTED}

\medskip
\setcounter{lineCounter}{0}
\aLine{\texttt{i <= a}}{line~\ref{l:gapex2-lg}}
\aLine{}{}
\aLine{}{}
\aLine{}{}
Therefore we \ans{can}/\noans{cannot} conclude that\\[0.6ex]
\aLine[-2ex]{\texttt{i <= a}}{line~\ref{l:gapex2-gap}}
\end{framed}

\RUBRIC
Part (c)

Gradescope rubric:
+0.25pt First box: UNSUPPORTED
+0.25pt First box: correctly justified (array size could be negative)
+0.25pt Second box: SUPPORTED
+0.25pt Second box: correctly justified (negation of loop guard on line 9)
-0.1pt Second box: unnecessarily use of line 10

Commentary:
- Line 12 is UNSUPPORTED
  . i >= a (line 10)
  . i and a are modified in the loop (lines 13 and 14)
  . This does not allow us to conclude that i > 0 at line 12.
- Line 17 is SUPPORTED
  . i <= a by line 9 (negation of the exit condition)
    [If they cite line 10 unnecessarily in their justification, this is
    incorrect: we want people to give us the minimal set of line
    numbers necessary to justify a conclusion.]

ENDRUBRIC
