\clearpage
\Question{AVL Trees}

\begin{parts}
\newcommand{\insseqA}{B}%{E}{C}%{3}%{11}%{15}
\newcommand{\insseqB}{G}%{J}{H}%{16}%{42}%{53}
\newcommand{\insseq}  % Vary from time to time
%{\insseqA, \insseqB, 71, 58, 129, 37, 97}
%{\insseqA, \insseqB, 61, 56, 122, 47, 99}
%{\insseqA, \insseqB, 68, 65,  83, 57, 74}
 {\insseqA, \insseqB,  M,  K,   W,  J,  S}
%{\insseqA, \insseqB,  N,  L,   X,  K,  T}
%{\insseqA, \insseqB,  O,  M,   Y,  L,  U}


%%%%% change the details !!!!!!!! %%%%%%%%%

\part[2]\TAGS{avl}
 Draw the AVL trees that result after successively
inserting the following keys into an initially empty tree, in the
order shown:

\bigskip
\begin{center}\tt
\insseq
\end{center}
\bigskip

Show the tree after each insertion and subsequent re-balancing (if
any) is completed: the tree after the first element,
\texttt{\insseqA}, is inserted into an empty tree, then the tree after
\texttt{\insseqB} is inserted into the first tree, and so on for a
total of seven trees.  Make it clear what order the trees are in.

Be sure to maintain and restore the BST invariants and the additional
balance invariant required for an AVL tree after each insert.
\begin{framed} % Paste the appropriate version
\ifprintanswers
\begin{lstlisting}[basicstyle=\smallbasicstyle\color{\answerColor}]
(1) (2)     (3)      (4)          (5)           (6)           (7)

                      G           G             K            K
                    /   \       /   \         /   \        /   \
      B      G     B     M     B     M       G     M      G     S
       \    / \         /           / \     / \     \    / \   / \
 B      G  B   M       K           K   W   B   J     W  B   J M   W
\end{lstlisting}
\else~\vspace{6in}\fi
\end{framed}

\RUBRIC
Part (a)
TAGS: avl

Gradescope rubric:
+2pt   EITHER -- All correct
+1.5pt     OR -- One mistake
+1pt       OR -- 2 or 3 small mistakes
+0.5pt     OR -- More than 3 mistakes, but first 3 trees are correct

Commentary:
Don't sweat too much trying to give points to grossly
wrong answers. If you can easily identify a correct rotation, fine,
give them some credit (see below).

(1) (2)     (3)      (4)          (5)           (6)           (7)

                      G           G             K            K
                    /   \       /   \         /   \        /   \
      B      G     B     M     B     M       G     M      G     S
       \    / \         /           / \     / \     \    / \   / \
 B      G  B   M       K           K   W   B   J     W  B   J M   W

[Older variant with numbers]
(1) (2)     (3)      (4)          (5)           (6)           (7)

                      16          16             58            58
                    /   \        /  \           /  \          /  \
     3      16     3    71      3    71       16    71      16    97
       \    /  \        /           /  \     /  \     \    /  \   / \
 3     16  3   71     58           58  129  3   37   129  3   37 71 129

 2pt:   all correct
 1.5pt:   clearly just one mistake
 1pt:   2-3 small mistakes
 0.5pt: Many mistakes, but got as far as (3) above
ENDRUBRIC

\medskip
\newpage
\part\TAGS{avl, complexity}
Recall our definition for the height $h$ of a tree:
\begin{quote}\em
  The height of a tree is the maximum number of nodes on a path from the root
  to a leaf. So the empty tree has height 0, the tree with one node has height
  1, and a balanced tree with three nodes has height 2.
\end{quote}
The minimum and maximum number of nodes $m$ in a valid AVL tree is
related to its height. The goal of this question is to quantify this
relationship.

\begin{subparts}
\subpart[1] %
Let $m(h)$ be the minimum number of nodes in an AVL tree
of height $h$. Fill in the table below relating $h$ and $m(h)$:

\medskip\medskip
\begin{center}
\newcommand{\ans}[1]{\answer{5em}{\centering\ensuremath #1}}
\renewcommand{\arraystretch}{1.3}
\begin{tabular}{|@{\hspace{1.5em}}c@{\hspace{1.5em}}|c|}
       \hline $h$ & $m(h)$
\\[5pt]\hline   0 & $0$
\\[5pt]\hline   1 & $1$
\\[5pt]\hline   2 & $2$
\\[5pt]\hline   3 & \ans{4}
\\[5pt]\hline   4 & \ans{7}
\\[5pt]\hline   5 & \ans{12}
\\[5pt]\hline   6 & \ans{20}
\\[5pt]\hline
\end{tabular}
\end{center}

\RUBRIC
Part (b i)
TAGS: avl, complexity

Gradescope rubric:
+1pt -- EITHER -- All correct: 4, 7, 12, 20
+0.5pt -- OR -- One error (may result in more than one incorrect number)

Commentary:
Two points. Give one point if one arithmetic error (one
arithmetic error could result in more than one incorrect number.)
 0 0
 1 1
 2 2
 3 4
 4 7
 5 12
 6 20
ENDRUBRIC


\medskip
\subpart[1]
Guided by the table in part (i), give an expression for $m(h)$.

\enlargethispage{5ex}
Here's a hint: recall that the $n$th Fibonacci number $F(n)$ is defined by:
\bgroup\small
\begin{align*}
F(0) &= 0 \\
F(1) &= 1 \\
F(n) &= F(n-1) + F(n-2), \qquad n > 1
\end{align*}
\egroup
You may find it useful to use the Fibonacci function $F(n)$ in your
answer.  Your answer \underline{does not} need to be a closed form
expression; it could be a recursive definition like the one for
$F(n)$.
\begin{framed}
\ifprintanswers{\color{\answerColor}%
\vspace{-2ex}%
\begin{align*}
m(0) &= 0 \\
m(1) &= 1 \\
m(h) &= 1 + m(h - 1) + m(h-2), \qquad h > 1
\end{align*}
\vspace{-3ex}%
}\else~\vspace{2.5cm}\fi
\end{framed}

\RUBRIC
Part (b ii)

Gradescope rubric:
+1pt -- EITHER -- m(h) = 1 + m(h - 1) + m(h-2)
+1pt -- OR -- m(h) = m(h-1) + F(h)
+1pt -- OR -- m(h) = F(h+2) - 1

Commentary:
All or nothing
  m(h) = m(h-2) + m(h-1) + 1
ENDRUBRIC


\subpart[0\half] %
Give a closed form expression (non-recursive) for $M(h)$, the
\emph{maximum} number of nodes in a valid AVL tree of height $h$.

\begin{framed}
\bigskip
$M(h) = \answer{28em}{$2^h-1$}$
\end{framed}

\RUBRIC
Part (b iii)

Gradescope rubric:
+0.5pt 2^h - 1

Commentary:
All or nothing
  M(h) = (2**h)-1
ENDRUBRIC

\end{subparts}
\end{parts}
