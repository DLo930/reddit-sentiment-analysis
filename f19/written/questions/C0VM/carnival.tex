\bgroup
\newcommand{\ans}[1]{\uanswer{25em}{#1\hfill}}
\newcommand{\n}{\textbackslash{n}}
\part[2]\TAGS{c0vm}
\label{g_bytecode}
(Note that the bytecode continues on the following page.)
\enlargethispage{2ex}
\begin{lstlisting}[language=C0VM, basicstyle=\smallbasicstyle]
C0 C0 FF EE       # magic number
00 13             # version 9, arch = 1 (64 bits)

00 00             # int pool count
# int pool

00 11             # string pool total size
# string pool
48 61 70 70 79 20 43 61 72 6E 69 76 61 6C 21 0A 00

00 02             # function count
# function_pool

#<main>
00 00             # number of arguments = 0
00 03             # number of local variables = 3
00 11             # code length = 17 bytes
14 00 00 # aldc 0          # [*\ans{s[0] = "Happy Carnival!\n"}*]
B7 00 00 # invokenative 0  # [*\ans{print("Happy Carnival!\n")}*]
57       # pop             # (ignore result)
10 00    # bipush 0        # [*\ans{0}*]
10 01    # bipush 1        # [*\ans{1}*]
10 0A    # bipush 10       # [*\ans{10}*]
B8 00 01 # invokestatic 1  # [*\ans{g(0, 1, 10)}*]
B0       # return          #

#<g>
00 03             # number of arguments = 3
00 03             # number of local variables = 3
00 30             # code length = 48 bytes
15 02    # vload 2         # [*\ans{n}*]
10 00    # bipush 0        # [*\ans{0}*]
9F 00 06 # if_cmpeq +6     # [*\ans{if (n == 0) goto {\tiny\ref{l:c0vm-cvl-00:then}}}*]
A7 00 09 # goto +9         # [*\ans{goto {\tiny\ref{l:c0vm-cvl-01:else}}}*]
15 00    # vload 0         # [*\ans{fn2}*][*\label{l:c0vm-cvl-00:then}*]
B0       # return          #
A7 00 03 # goto +3         # [*\ans{goto {\tiny\ref{l:c0vm-cvl-02:endif}}}*]
15 02    # vload 2         # [*\ans{n}*][*\label{l:c0vm-cvl-01:else}*][*\label{l:c0vm-cvl-02:endif}*]
10 01    # bipush 1        # [*\ans{1}*]
9F 00 06 # if_cmpeq +6     # [*\ans{if (n == 1) goto {\tiny\ref{l:c0vm-cvl-03:then}}}*]
A7 00 09 # goto +9         # [*\ans{goto {\tiny\ref{l:c0vm-cvl-04:else}}}*]
15 01    # vload 1         # [*\ans{fn3}*][*\label{l:c0vm-cvl-03:then}*]
B0       # return          #
A7 00 03 # goto +3         # [*\ans{goto {\tiny\ref{l:c0vm-cvl-05:endif}}}*]
15 01    # vload 1         # [*\ans{fn3}*][*\label{l:c0vm-cvl-04:else}*][*\label{l:c0vm-cvl-05:endif}*]
15 00    # vload 0         # [*\ans{fn2}*]
15 01    # vload 1         # [*\ans{fn3}*]
60       # iadd            # [*\ans{(fn2 + fn3)}*]
15 02    # vload 2         # [*\ans{n}*]
10 01    # bipush 1        # [*\ans{1}*]
64       # isub            # [*\ans{(n - 1)}*]
B8 00 01 # invokestatic 1  # [*\ans{g(fn3, (fn2 + fn3), (n - 1))}*]
B0       # return          #

00 01             # native count
# native pool
00 01 00 06       # print
\end{lstlisting}
\begin{framed}
\ifprintanswers
\begin{lstlisting}[basicstyle=\basicstyle\color{\answerColor}]
#use <conio>

int g(int fn2, int fn3, int n) {
   if (n == 0) return fn2;
   if (n == 1) return fn3;
   return g(fn3, fn2+fn3, n-1);
}

int main() {
   print("Happy Carnival!\n");
   return g(0, 1, 10);
}
\end{lstlisting}
\else~\vspace{6in}\fi
\end{framed}

\RUBRIC
Part (b)
TAGS: c0vm

Gradescope rubric:
-0 pt WARNING: NEGATIVE GRADING
-0.25pt main: incorrect string
-0.25pt main: incorrect print statement
-0.5pt main: incorrect call to function g
-0.25pt g: incorrect if statement
-0.25pt g: incorrect recursive call
-0.25pt g: incorrect body in rest of if/else
-0.5pt altogether: doesn't match structure of byte code in other ways

Commentary:
- If student writes equivalent program but in reduced/simplified form,
  give half credit.
- No points off for omitting the #use.

#use <conio>

int g(int fn2, int fn3, int n) {
   if (n == 0) return fn2;
   if (n == 1) return fn3;
   return g(fn3, fn2+fn3, n-1);
}

int main() {
   print("Happy Carnival!\n");
   return g(0, 1, 10);
}
ENDRUBRIC
\egroup


\newpage
\part[1]\TAGS{c0vm}
This question has to do with the function \lstinline'g' in the bytecode
given in part~(\ref{g_bytecode}) above.

When execution reaches the instruction on line 52,
there are three values on the operand stack; assume they are
\lstinline'0x0000000D', \lstinline'0x00000015' and
\lstinline'0x00000009', with the last being at the top of the stack.

You will now trace back execution and determine the four operand stack states before each of lines 48--51 is
executed.  The elements in your stack should be 32-bit hexadecimal
numbers. The top of your stack should be on the right-hand
side.  You may not need all the provided spaces

\begin{framed}
\newcommand{\ans}[1]{\uanswer{8em}{\tt #1}}

Immediately before executing line 51: \lstinline[language=opsem]'isub'

\bigskip
\ans{0x0000000D}, \ans{0x00000015}, \ans{0x0000000A}, \ans{0x00000001}

\bigskip
\bigskip
Immediately before executing line 50: \lstinline[language=opsem]'bipush 1'

\bigskip
\ans{0x0000000D}, \ans{0x00000015}, \ans{0x0000000A}, \ans{}

\bigskip
\bigskip
Immediately before executing line 49: \lstinline[language=opsem]'vload 2'

\bigskip
\ans{0x0000000D}, \ans{0x00000015}, \ans{}, \ans{}

\bigskip
\bigskip
Immediately before executing line 48: \lstinline[language=opsem]'iadd'

\bigskip
\ans{0x0000000D}, \ans{0x00000008}, \ans{0x0000000D}, \ans{}

\bigskip
\bigskip

\end{framed}

\RUBRIC
Part (c)
TAGS: c0vm

Gradescope rubric:
+0.25pt Line 51:  0x0000000D 0x00000015 0x0000000A 0x00000001
+0.25pt Line 50:  0x0000000D 0x00000015 0x0000000A
+0.25pt Line 49:  0x0000000D 0x00000015
+0.25pt Line 48:  0x0000000D 0x00000008 0x0000000D

Commentary:
  1/12 pt for each answer overall rounded to the next 1/4, all or nothing
ENDRUBRIC
