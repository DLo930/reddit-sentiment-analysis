%% See bottom of file

\clearpage
\Question{Generic Algorithms}

A generic comparison function might be given a type as follows in C1:
\begin{lstlisting}[numbers=none]
typedef int compare_fn(void* x, void* y)
  //@ensures -1 <= \result && \result <= 1;
\end{lstlisting}
(Note: there's no precondition that \lstinline'x'
and \lstinline'y' are necessarily non-\lstinline'NULL'.)

If we're given such a function, we can treat \lstinline'x' as being less than
\lstinline'y' if the function returns \lstinline'-1', treat \lstinline'x' as
being greater than \lstinline'y' if the function returns \lstinline'1', and
treat the two arguments as being equal if the function returns \lstinline'0'.

Given such a comparison function, we can write a function to check
that an array is sorted even though we don't know the type of its
elements (as long as it is a pointer type):
\begin{lstlisting}[numbers=none]
bool is_sorted(void*[] A, int lo, int hi, compare_fn* cmp)
  //@requires 0 <= lo && lo <= hi && hi <= \length(A) && cmp != NULL;
\end{lstlisting}

\begin{parts}
\enlargethispage{5ex}
\part[1]\TAGS{function-pointer, genericity, void-star}
Complete the generic binary search function below. You don't have
access to generic variants of \lstinline'lt_seg' and
\lstinline'gt_seg'. Remember that, for sorted integer arrays,
\lstinline'gt_seg(x, A, 0, lo)' was equivalent to %
\lstinline'lo == 0 || A[lo - 1] < x'.

\medskip
\begin{framed}
\begin{lstlisting}[aboveskip=0pt, belowskip=0pt]
[*\hspace{-0.5em}*]int binsearch_generic(void* x, void*[] A, int n, compare_fn* cmp)
//@requires 0 <= n && n <= \length(A) && cmp != NULL;
//@requires is_sorted(A, 0, n, cmp);
{
   int lo = 0;
   int hi = n;

   while (lo < hi)
   //@loop_invariant 0 <= lo && lo <= hi && hi <= n;

   //@loop_invariant lo == [*\uanswer{3em}{0}*] || [*\uanswer{12em}{(*cmp)(A[lo - 1], x)}*] == -1;

   //@loop_invariant hi == [*\uanswer{3em}{n}*] || [*\uanswer{12em}{(*cmp)(A[hi], x)}*] == 1;
   {
      int mid = lo + (hi - lo)/2;

      int c = [*\uanswer{24em}{(*cmp)(A[mid], x)}*];

      if (c == 0) return mid;
      if (c < 0) lo = mid + 1;
      else hi = mid;
   }
   return -1;
}
\end{lstlisting}
\end{framed}

\RUBRIC
Part (a)
TAGS: function-pointer, genericity, void-star

Gradescope rubric:
+0.5  Invokes function pointers correctly in all 3 instances (*cmp)(A[indx], x)
+0.25 Loop invariants correctly check bounds and attempt to call cmp function with right arguments
+0.25 Attempts to call cmp function with correct arguments (A[mid], x)

Commentary:
  - Solutions: (lo == 0 || (*cmp)(A[lo - 1], x) < 0)
                           (*cmp)(A[lo - 1], x) == -1
                           (*cmp)(x, A[lo - 1]) > 0
                           (*cmp)(x, A[lo - 1]) == 1
               (hi == n || (*cmp)(x, A[hi]) < 0)
                           (*cmp)(x, A[hi]) == -1
                           (*cmp)(A[hi], x) > 0
                           (*cmp)(A[hi], x) == 1
               int c = (*cmp)(A[mid], x)
     ** 1/4 point is for edge case/correctness
     ** 1/4 point is for correctly invoking function pointer
     ** 1/4 point is for getting directionality right
ENDRUBRIC

\newpage
\begin{EnvUplevel}
  Suppose you have a generic sorting function, with the following contract:
\begin{lstlisting}[numbers=none]
void sort_generic(void*[] A, int lo, int hi, compare_fn* cmp)
  //@requires 0 <= lo && lo <= hi && hi <= \length(A) && cmp != NULL;
  //@ensures is_sorted(A, lo, hi, cmp);
\end{lstlisting}
\end{EnvUplevel}

%\part[1]\TAGS{void-star}
Write an integer comparison function \lstinline'compare_ints' that can
be used with this generic sorting function.  The contracts on your
\lstinline'compare_ints' function \emph{must} be sufficient to ensure
that no precondition-passing call to \lstinline'compare_ints' can
possibly cause a memory error.
\begin{framed}
\begin{lstlisting}
int compare_ints(void* x, void* y)

//@requires x != NULL && \hastag([*\uanswer{16.7em}{int*, x}*]);

//@requires y != NULL && \hastag([*\uanswer{16.7em}{int*, y}*]);
//@ensures -1 <= \result && \result <= 1;
{

   if ([*\uanswer{21em}{*(int*)x < *(int*)y}*]) return [*\uanswer{6.5em}{-1}*];

   if ([*\uanswer{21em}{*(int*)x > *(int*)y}*]) return [*\uanswer{6.5em}{1}*];

   return [*\uanswer{31em}{0}*];
}
\end{lstlisting}
\end{framed}

\RUBRIC
Part (b)
TAGS: void-star

Gradescope rubric:
+0.25pt Correctly invokes \hastag
+0.25pt Correctly casts
+0.25pt Correctly dereferences int pointers
+0.25pt Correct returns

Commentary:
     ** 1/4 point for correctly handling the possibility of NULL
     ** 1/4 point for correctly handing \hastag
     ** 1/4 point for casting and dereferencing int pointer
     ** 1/4 point for getting the return (-1, 0, 1) correct

     int compare_helper(void* x, void* y)
     //@requires x != NULL && \hastag(__int*, x__);
     //@requires y != NULL && \hastag(__int*, y__);
     {
        if (*(int*)x < *(int*)y) return -1;
        if (*(int*)x > *(int*)y) return 1;
        return 0;
     }
ENDRUBRIC


\newpage
\part[2]\TAGS{array, function-pointer, genericity, void-star}
Using \lstinline'sort_generic' (which you may assume has already been
written) and \lstinline'compare_ints', fill in the body of the
\lstinline'sort_ints' function below so that it will sort the array
\lstinline'A' of integers.  You can omit loop invariants. But of
course, when you call \lstinline'sort_generic', the preconditions of
\lstinline'compare_ints' must be satisfied by any two elements of the
array \lstinline'B'.

\begin{framed}
\begin{lstlisting}[belowskip=0pt]
void sort_ints(int[] A, int n)
//@requires \length(A) == n;
{
   // Allocate a temporary generic array of the same size as A

   void*[] B = [*\uanswer{20em}{alloc\_array(void*, n)}*];

   // Store a copy of each element in A into B
\end{lstlisting}
\ifprintanswers
\begin{lstlisting}[basicstyle=\basicstyle\color{\answerColor}]
   for (int i = 0; i < n; i++) {
     int* p = alloc(int);
     *p = A[i];
     B[i] = (void*)p;
   }
\end{lstlisting}
\else~\vspace{2.0in}\fi
\begin{lstlisting}[aboveskip=0pt, belowskip=0pt]
   // Sort B using sort_generic and compare_ints from task 2
\end{lstlisting}
\ifprintanswers
\begin{lstlisting}[basicstyle=\basicstyle\color{\answerColor}]
   sort_generic(B, 0, n, &compare_helper);
\end{lstlisting}
\else~\vspace{0.5in}\fi
\begin{lstlisting}[aboveskip=0pt, belowskip=0pt]
   // Copy the sorted ints in your generic array B into array A
\end{lstlisting}
\ifprintanswers
\begin{lstlisting}[basicstyle=\basicstyle\color{\answerColor}]
   for (int i = 0; i < n; i++) {
      A[i] = *(int*)B[i];
   }
\end{lstlisting}
\else~\vspace{2.0in}\fi
\begin{lstlisting}[aboveskip=0pt]
}
\end{lstlisting}
\end{framed}

\RUBRIC
Part (c)
TAGS: array, function-pointer, genericity, void-star

Gradescope rubric:
+ 0.5 pts Correctly allocate array
+ 0.5 pts Correctly copying A elements into B
+ 0.5 pts Correctly calls sort_generic
+ 0.5 pts Correctly copies sorted ints back into A

Commentary:
     void*[] B = alloc_array(void*, n);  // 1/2 point
     for (int i = 0; i < n; i++) {
        int* p = alloc(int);             // 1/2 point, correctly initializing
        *p = A[i];                       // an INT pointer in the loop,
        B[i] = (void*)p;                 // then casting to void
     }
     sort_generic(B, 0, n, &compare_helper);
                                         // 1/2 point, correct call to sort
                                         // (MUST include address-of &)
     for (int i = 0; i < n; i++) {
        A[i] = *(int*)B[i];              // 1/2 point for casting back
     }


Alternative first loop body:
      B[i] = (void*) alloc(int);
      int* x = (int*) B[i];
      *x = A[i];
ENDRUBRIC % S19 F17 F16 S16
\part[1]\TAGS{void-star}
Write a character comparison function \lstinline'compare_chars' that
can be used with this generic sorting function.  Characters are
compared based on their ASCII value.  The contracts on your
\lstinline'compare_chars' function \emph{must} be sufficient to ensure
that no precondition-passing call to \lstinline'compare_chars' can
possibly cause a memory error.  Depending on how you write your
solution, you may or may not need the extra space at the beginning of
this function.

\begin{framed}
\begin{lstlisting}
int compare_chars(void* x, void* y)

//@requires x != NULL && \hastag([*\uanswer{17em}{char*, x}*]);

//@requires y != NULL && \hastag([*\uanswer{17em}{char*, y}*]);
//@ensures -1 <= \result && \result <= 1;
{
                                       // Extra space if needed
  [*\answer{33em}{int x\_ascii = char\_ord(*(char*)x);\hfill}*]
  [*\answer{33em}{int y\_ascii = char\_ord(*(char*)y);\hfill}*]



  if ([*\uanswer{17em}{x\_ascii < y\_ascii}*]) return [*\uanswer{11em}{-1}*];

  if ([*\uanswer{17em}{x\_ascii > y\_ascii}*]) return [*\uanswer{11em}{1}*];

  return [*\uanswer{31.5em}{0}*];
}
\end{lstlisting}
\end{framed}


\RUBRIC
Part (b)
TAGS: void-star

Gradescope rubric:
+0.25 Correctly invokes \hastag
+0.25 Correctly casts
+0.25 Correctly dereferences char pointers
+0.25 Correct returns

Commentary:
* 1/4 point for correctly handling the possibility of NULL
* 1/4 point for correctly handing \hastag
* 1/4 point for casting and dereferencing int pointer
* 1/4 point for getting the return (-1, 0, 1) correct

int compare_helper(void* x, void* y)
//@requires x != NULL && \hastag(char*, x);
//@requires y != NULL && \hastag(char*, y);
{
  int x_ascii = char_ord(*(char*)x);
  int y_ascii = char_ord(*(char*)y);

  if (x_ascii < y_ascii) return -1;
  if (x_ascii > y_ascii) return 1;
  return 0;
}
ENDRUBRIC


\newpage
\part[2]\TAGS{array, function-pointer, genericity, void-star}
Using \lstinline'sort_generic' (which you may assume has already been
written) and \lstinline'compare_chars', fill in the body of the
\lstinline'sort_chars' function below so that it will sort the array
\lstinline'A' of characters.  You can omit loop invariants.  But of
course, when you call \lstinline'sort_generic', the preconditions of
\lstinline'compare_chars' must be satisfied by any two elements of the
array \lstinline'B'.

\begin{framed}
\begin{lstlisting}[belowskip=0pt]
void sort_chars(char[] A, int n)
//@requires \length(A) == n;
{
  // Allocate a temporary generic array of the same size as A

  void*[] B = [*\uanswer{20em}{alloc\_array(void*, n)}*];

  // Store a copy of each element in A into B
\end{lstlisting}
\ifprintanswers
\begin{lstlisting}[basicstyle=\basicstyle\color{\answerColor}]
   for (int i = 0; i < n; i++) {
     char* p = alloc(char);
     *p = A[i];
     B[i] = (void*)p;
   }
\end{lstlisting}
\else~\vspace{2.0in}\fi
\begin{lstlisting}[aboveskip=0pt, belowskip=0pt]
  // Sort B using sort_generic and compare_chars from task 2
\end{lstlisting}
\ifprintanswers
\begin{lstlisting}[basicstyle=\basicstyle\color{\answerColor}]
   sort_generic(B, 0, n, &compare_helper);
\end{lstlisting}
\else~\vspace{0.5in}\fi
\begin{lstlisting}[aboveskip=0pt, belowskip=0pt]
  // Copy the sorted chars in your generic array B into array A
\end{lstlisting}
\ifprintanswers
\begin{lstlisting}[basicstyle=\basicstyle\color{\answerColor}]
   for (int i = 0; i < n; i++) {
      A[i] = *(char*)B[i];
   }
\end{lstlisting}
\else~\vspace{2.0in}\fi
\begin{lstlisting}[aboveskip=0pt]
}
\end{lstlisting}
\end{framed}

\RUBRIC
Part (c)
TAGS: array, function-pointer, genericity, void-star

Gradescope rubric:
+0.5 Correctly allocate array
+0.5 Correctly copying A elements into B
+0.5 Correctly calls sort_generic
+0.5 Correctly copies sorted chars back into A

Commentary:
     void*[] B = alloc_array(void*, n);  // 1/2 point
     for (int i = 0; i < n; i++) {
        char* p = alloc(char);           // 1/2 point, correctly initializing
        *p = A[i];                       // an INT pointer in the loop,
        B[i] = (void*)p;                 // then casting to void
     }
     sort_generic(B, 0, n, &compare_helper);
                                         // 1/2 point, correct call to sort
                                         // (MUST include address-of &)
     for (int i = 0; i < n; i++) {
        A[i] = *(char*)B[i];             // 1/2 point for casting back
     }


Alternative first loop body:
      B[i] = (void*) alloc(char);
      char* x = (char*) B[i];
      *x = A[i];
ENDRUBRIC
 % F19 S18 S17
%\bgroup
\newcommand{\channel}
{blue}  % F18
%{red}
%{green}
%{alpha}

\begin{EnvUplevel}
Recall also the pixel interface seen early in the course:
\begin{lstlisting}
//typedef _____ pixel_t;

int get_red(pixel_t p)   /*@ensures 0 <= \result && \result < 256; @*/ ;
int get_green(pixel_t p) /*@ensures 0 <= \result && \result < 256; @*/ ;
int get_blue(pixel_t p)  /*@ensures 0 <= \result && \result < 256; @*/ ;
int get_alpha(pixel_t p) /*@ensures 0 <= \result && \result < 256; @*/ ;

pixel_t make_pixel(int alpha, int red, int green, int blue)
/*@requires 0 <= alpha && alpha < 256; @*/
/*@requires 0 <= red   &&   red < 256; @*/
/*@requires 0 <= green && green < 256; @*/
/*@requires 0 <= blue  &&  blue < 256; @*/ ;
\end{lstlisting}
\end{EnvUplevel}

\part[1]\TAGS{void-star}
Write a pixel comparison function \texttt{compare\_\channel} that can be
used with the above generic sorting function, which you should assume is
already written.  The function  \texttt{compare\_\channel} compares
pixels based uniquely on the intensity of their \channel{} component.  For
example, pixel \lstinline'p1' with \channel{} component 122 is considered
smaller than pixel \lstinline'p2' with \channel{} component 210,
irrespective of the values of their other components.

As you write this function, the contracts on your
\texttt{compare\_\channel} function \emph{must} be sufficient to ensure
that no precondition-passing call to \texttt{compare\_\channel} can
possibly cause a memory error.
\begin{framed}
\begin{lstlisting}[aboveskip=0pt, belowskip=0pt]
int compare_[*\channel*](void* x, void* y)

//@requires x != NULL && \hastag([*\uanswer{16.7em}{pixel\_t*, x}*]);

//@requires y != NULL && \hastag([*\uanswer{16.7em}{pixel\_t*, y}*]);
//@ensures -1 <= \result && \result <= 1;
{
                                       // Extra space if needed
  [*\answer{33em}{int x\_\channel{} = get\_\channel(*(pixel\_t*)x);\hfill}*]
  [*\answer{33em}{int y\_\channel{} = get\_\channel(*(pixel\_t*)y);\hfill}*]



   if ([*\uanswer{21em}{x\_\channel{} < y\_\channel{}}*]) return [*\uanswer{6.5em}{-1}*];

   if ([*\uanswer{21em}{x\_\channel{} > y\_\channel{}}*]) return [*\uanswer{6.5em}{1}*];

   return [*\uanswer{31em}{0}*];
}
\end{lstlisting}
\end{framed}

\RUBRIC
Part (b)
TAGS: void-star

Gradescope rubric:
+0.25pt Correctly invokes \hastag
+0.25pt Correctly casts
+0.25pt Correctly dereferences int pointers
+0.25pt Correct returns

Commentary:
- set 'channel' to whatever channel this semester's variant uses
     ** 1/4 point for correctly handling the possibility of NULL
     ** 1/4 point for correctly handing \hastag
     ** 1/4 point for casting and dereferencing int pointer
     ** 1/4 point for getting the return (-1, 0, 1) correct

     int compare_helper(void* x, void* y)
     //@requires x != NULL && \hastag(pixel_t*, x);
     //@requires y != NULL && \hastag(pixel_t*, y);
     {
        if (get_channel(*(pixel_t*)x) < get_channel(*(pixel_t*)y)) return -1;
        if (get_channel(*(pixel_t*)x) > get_channel(*(pixel_t*)y)) return 1;
        return 0;
     }
ENDRUBRIC


\newpage
\part[2]\TAGS{function-pointer, genericity, void-star}
Using \lstinline'sort_generic' (which you may assume has already been
written) and \texttt{compare\_\channel}, fill in the body of the
\texttt{sort\_\channel} function below so that it will sort the array
\lstinline'A' of pixels.  You can omit loop invariants. But of course,
when you call \lstinline'sort_generic', the preconditions of
\texttt{compare\_\channel} must be satisfied by any two elements of
the array \lstinline'B'.

\begin{framed}
\begin{lstlisting}[belowskip=0pt]
void sort_[*\channel*](pixel_t[] A, int n)
//@requires \length(A) == n;
{
   // Allocate a temporary generic array of the same size as A

   void*[] B = [*\uanswer{20em}{alloc\_array(void*, n)}*];

   // Store a copy of each element in A into B
\end{lstlisting}
\ifprintanswers
\begin{lstlisting}[basicstyle=\basicstyle\color{\answerColor}]
   for (int i = 0; i < n; i++) {
     pixel_t* p = alloc(pixel_t);
     *p = A[i];
     B[i] = (void*)p;
   }
\end{lstlisting}
\else~\vspace{2.0in}\fi
\begin{lstlisting}[aboveskip=0pt, belowskip=0pt]
   // Sort B using sort_generic and compare_[*\color{\commentColor}\channel*] from task 2
\end{lstlisting}
\ifprintanswers
\begin{lstlisting}[basicstyle=\basicstyle\color{\answerColor}]
   sort_generic(B, 0, n, &compare_helper);
\end{lstlisting}
\else~\vspace{0.5in}\fi
\begin{lstlisting}[aboveskip=0pt, belowskip=0pt]
   // Copy the sorted pixels from generic array B into array A
\end{lstlisting}
\ifprintanswers
\begin{lstlisting}[basicstyle=\basicstyle\color{\answerColor}]
   for (int i = 0; i < n; i++) {
      A[i] = *(pixel_t*)B[i];
   }
\end{lstlisting}
\else~\vspace{2.0in}\fi
\begin{lstlisting}[aboveskip=0pt]
}
\end{lstlisting}
\end{framed}

\RUBRIC
Part (c)
TAGS: array, function-pointer, genericity, void-star

Gradescope rubric:
+ 0.5 pts Correctly allocate array
+ 0.5 pts Correctly copying A elements into B
+ 0.5 pts Correctly calls sort_generic
+ 0.5 pts Correctly copies sorted pixels back into A

Commentary:
     void*[] B = alloc_array(void*, n);  // 1/2 point
     for (int i = 0; i < n; i++) {
        pixel_t* p = alloc(pixel_t);         // 1/2 point, correctly initializing
        *p = A[i];                       // an PIXEL pointer in the loop,
        B[i] = (void*)p;                 // then casting to void
     }
     sort_generic(B, 0, n, &compare_helper);
                                         // 1/2 point, correct call to sort
                                         // (MUST include address-of &)
     for (int i = 0; i < n; i++) {
        A[i] = *(pixel_t*)B[i];            // 1/2 point for casting back
     }


Alternative first loop body:
      B[i] = (void*) alloc(pixel_t);
      pixel_t* x = (pixel_t*) B[i];
      *x = A[i];
ENDRUBRIC

\egroup
 % F18 -- see file

\end{parts}