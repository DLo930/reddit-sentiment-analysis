\clearpage
\Question{Proving the correctness of functions with one loop}

The Pell sequence is shown below:
$$
0, 1, 2, 5, 12, 29, 70, 169, 408, 985, \ldots
$$
Each integer $i_n$ in the sequence for $n \geq 3$ is the sum of
$2i_{n-1}$ and $i_{n-2}$. By definition, $i_1 = 0$ and $i_2 = 1$.
Consider the following implementation for \lstinline'fastpell' that
returns the $n$\textsuperscript{th} Pell number, $n \geq 1$. The body
of the loop is not shown.
\begin{lstlisting}[numbers=left, belowskip=0pt]
int PELL(int n)
//@requires n >= 1;
{
  if (n <= 1) return 0;
  else if (n == 2) return 1;
  else return 2 * PELL(n-1) + PELL(n-2);
}

int fastpell(int n)
//@requires n >= 1;
//@ensures \result == PELL(n);[*\label{l:fastpell-post}*]
{
  if (n <= 1) return 0;[*\label{l:fastpell-if1}*]
  if (n == 2) return 1;[*\label{l:fastpell-if2}*]
  int i = 0;[*\label{l:fastpell-ini1}*]
  int j = 1;[*\label{l:fastpell-ini2}*]
  int k = 2;[*\label{l:fastpell-ini3}*]
  int x = 3;[*\label{l:fastpell-ini4}*]
  while (x < n)[*\label{l:fastpell-lg}*]
    //@loop_invariant 3 <= x && x <= n;[*\label{l:fastpell-LI1}*]
    //@loop_invariant i == PELL(x-2);[*\label{l:fastpell-LI2}*]
    //@loop_invariant j == PELL(x-1);[*\label{l:fastpell-LI3}*]
    //@loop_invariant k == i + 2*j;[*\label{l:fastpell-LI4}*]
    {
\end{lstlisting}
\begin{lstlisting}[numbers=none,aboveskip=-0pt]
      // LOOP BODY NOT SHOWN: modifies i, j, k, and x
    }
  return k;
}
\end{lstlisting}

In this problem, we will reason about the correctness of the
\lstinline'fastpell' function when the argument \lstinline'n' is
greater than or equal to \lstinline'3', and we will complete the
implementation based on this reasoning.

(NOTE: To completely reason about the correctness of
\lstinline'fastpell', we also need to point out that
\lstinline'fastpell(1) == PELL(1)' and that %
\lstinline'fastpell(2) == PELL(2)'. This is straightforward, because
no loops are involved.)

\newpage
\emph{Note: The completed solution below shows you a general format for showing
  that a postcondition holds given a valid loop invariant. The English
  explanation is kept to a minimum and point-to reasoning
  plays a large role.  In the future, you may be asked to write an
  entire solution in a clear, concise manner, and the solution below gives
  you an example of how you might write such a solution.}


\begin{parts}
\newcommand{\ans}[2]{\fbox{\rule[-0.5ex]{0em}{3ex}\answer{#1}{#2}}}

\part[1]\TAGS{correctness, loop-invariant}
{\bf Loop invariant and negation of the loop guard imply postcondition}

Complete the argument that the postcondition is satisfied assuming
valid loop invariant(s) by giving appropriate line numbers. Use
point-to reasoning.

\begin{framed}
  We know \lstinline'x <= n' by line \ans{1.5em}{\ref{l:fastpell-LI1}}
  and we know \lstinline'x >= n' by line
  \ans{1.5em}{\ref{l:fastpell-lg}} , which implies that %
  \lstinline'x == n' by logic.

  \bigskip%
  The returned value \result{} is the value of \lstinline'k' after the
  loop, so to show that the postcondition on
  line~\ref{l:fastpell-post} holds when \lstinline'n >= 3', it
  suffices to show \lstinline'k == PELL(n)' after the loop.

\begin{tabular}{ll}
     \lstinline'k == i + 2*j'
   & by line \ans{1.5em}{\ref{l:fastpell-LI4}}
\\\\ \lstinline'  == i + 2*PELL(x-1)'
   & by line \ans{1.5em}{\ref{l:fastpell-LI3}}
\\\\ \lstinline'  == PELL(x-2) + 2*PELL(x-1)'
   & by line \ans{1.5em}{\ref{l:fastpell-LI2}}
\\\\ \lstinline'  == PELL(x)'
   & by \lstinline'PELL' definition, the commutativity
\\ & \phantom{by }of \lstinline'+', and \lstinline'x >= 1' by line
     \ans{1.5em}{\ref{l:fastpell-LI1}}
\end{tabular}
\end{framed}

\RUBRIC
Make sure to make it clear what the answer(s) were and
which ones they got right or wrong. It's wrong if there are missing lines and
it's wrong if there are extra lines given (except where indicated).

Part (a)
TAGS: correctness, loop-invariant

Gradescope rubric (negative grading):
-0.2pt  Box 1: 20
-0.2pt  Box 2: 19
-0.2pt  Box 3: 23
-0.2pt  Box 4: 22
-0.2pt  Box 5: 21
-0.2pt  Box 6: 20

Commentary:
  - x <= n                                    by line _20_
  - x >= n                                    by line _19_
  - k = i + 2*j                               by line _23_
  - i + 2*j == i + 2*PELL(x-1)                by line _22_
  - i + 2*PELL(x-1) = PELL(x-2) + 2*PELL(x-1) by line _21_
  - x >= 1                                    by line _20_

- 0.2pt off for each mistake up to a maximum of 4 mistakes (1 pt)
ENDRUBRIC

\part[1]\TAGS{correctness, loop-invariant}
{\bf Loop invariant holds initially}

Complete the argument for the loop invariants holding initially by
giving appropriate line numbers.

\begin{framed}
The loop invariant \lstinline'3 <= x' on line~\ref{l:fastpell-LI1}
holds initially by line(s) \ans{6em}{ \ref{l:fastpell-ini4}}.

\bigskip
The loop invariant \lstinline'x <= n' on line~\ref{l:fastpell-LI1}
holds initially by line(s) \ans{6em}{\ref{l:fastpell-if1}, \ref{l:fastpell-if2}, \ref{l:fastpell-ini4}}.

\bigskip
The loop invariant on line~\ref{l:fastpell-LI2} holds initially by
line(s) \ans{6em}{\ref{l:fastpell-ini1}, \ref{l:fastpell-ini4}}.

\bigskip
The loop invariant on line~\ref{l:fastpell-LI3} holds initially by
line(s) \ans{6em}{\ref{l:fastpell-ini2}, \ref{l:fastpell-ini4}}.

\bigskip
The loop invariant on line~\ref{l:fastpell-LI4} holds initially by
lines \ref{l:fastpell-ini3}, \ref{l:fastpell-ini1} and
\ref{l:fastpell-ini2}.
\end{framed}

\RUBRIC
Part (b)
TAGS: correctness, loop-invariant

Gradescope rubric:
+0.25pt  Box 1: line 18
+0.25pt  Box 2: lines 13, 14 and 18
+0.25pt  Box 3: lines 15 and 18 (any mention of lines 1-6 is fine)
+0.25pt  Box 4: lines 16 and 18 (any mention of lines 1-6 is fine)

Commentary:
- 3 <= x: line 18
- x <= n: lines 13, 14, and 18
- i == PELL(x-2): lines 15 and 18 (any mention of lines 1-6 is fine)
- j == PELL(x-1): lines 16 and 18 (and 1-6 is fine)
- k == i + 2*j: lines 15, 16, and 17

- 1/4 point off for each mistake
ENDRUBRIC

\newpage
\part[0\half]\TAGS{correctness, loop-invariant}
{\bf The loop invariant is preserved through any single iteration of the loop}

Based on the given loop invariants, write the body of the loop.
{\bf  DO NOT use the specification function \lstinline'PELL()'. The
specification function is meant to be used in contracts only.
Also, do not call \lstinline'fastpell' recursively, since this isn't fast!}

(NOTE: To check your answer, you would prove that the loop invariants
are preserved by an arbitrary iteration of the loop, but you don't
have to do that for us here --- we'll cover that process in the next
question.)

\begin{lstlisting}[frame=single, numbers=left, firstnumber=18, numberblanklines=false]
while (x < n)
//@loop_invariant 3 <= x && x <= n;
//@loop_invariant i == PELL(x-2);
//@loop_invariant j == PELL(x-1);
//@loop_invariant k == i + 2*j;
{
    i = [*\uanswer{25em}{j}*];

    j = [*\uanswer{25em}{k}*];

    k = [*\uanswer{25em}{2*i + j}*];

    x = [*\uanswer{25em}{x + 1}*];
}

return k;
\end{lstlisting}

\RUBRIC
Part (c)
TAGS: correctness, loop-invariant

Gradescope rubric:
+0.25pt  At least 3 of {i, j, k, x} correct
+0.25pt  All four correct

Commentary:
  - i = j;
  - j = k;
  - k = 2*i + j; // OR 2*k + j
  - x = x + 1;

  - .25 points if they make one mistake (check it's a mistake if you
    haven't seen it before!)
  - 0 points for recursion, calling PELL, anything else
ENDRUBRIC

\part[0\half]\TAGS{correctness}
{\bf The loop terminates}

The postcondition is satisfied only if the loop terminates.
Explain concisely why the function must terminate with the loop body you
gave in part c.
\begin{framed}
The integer quantity \ans{6em}{\texttt{n - x}} is strictly decreasing because

\medskip
\ifprintanswers{\color{\answerColor}
  \texttt{n} remains constant and \texttt{x} gets strictly bigger.
}\else~\vspace{8ex}\fi

\smallskip
Since the loop terminates if this quantity
reaches 0 or less and this quantity is strictly decreasing, the loop must terminate.
\end{framed}

\RUBRIC
Part (d)
TAGS: correctness

Gradescope rubric:
+0.5pt  -- EITHER -- correct quantity that can't go below zero (probably n-x) and reasonable justification (justification does not need to include line 19, the loop guard, even though technically it should)
+0.25pt -- OR -- correctly increasing quantity (like x) and reasonable justification for why it increases
+0.25pt -- OR -- correctly quantity that decreases (like -x) and justification for why it decreases

Commentary:
First blank:  n - x
Second blank: n remains constant and x gets strictly bigger (or always increases by 1).
- Note: i, j, and k get strictly larger is NOT justified due to modular arithmetic

1/4 point for each blank
ENDRUBRIC

\end{parts}
