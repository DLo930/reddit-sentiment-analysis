\clearpage
\bgroup
% Customizations for first task
\newcommand{\IMG}{IMG}%{A}
\newcommand{\ROW}{row}
\newcommand{\COL}{col}
\newcommand{\PIXEL}{P}%{q}{x}{P}
\newcommand{\WIDTH}{width}%{w}{wid}
\newcommand{\HEIGHT}{height}%{h}{hei}

%% Select the last two exercises from the options at the end of this file


\Question{Implementing an Image Type Using a Struct}

In a previous programming assignment, we worked with one-dimensional
arrays that represented two-dimensional images.  Neither the width nor
the height of an image could be 0.  Suppose we want to create a data
type for an image along with an interface that specifies functions to
allow us to get a pixel of the image or set a pixel of the image.
This type should still be implementable as a one-dimensional array
(but allow other choices).

(You may assume that \lstinline'p1 == p2' is an acceptable way of
comparing pixels for equality.)

\begin{parts}
\part[1\half]\TAGS{interface}
Complete the \underline{interface} for the image data type.
Add appropriate preconditions and postconditions for each image
operation \emph{(you may not need all the lines we provided)}.  The
first two functions should have at least one meaningful postcondition,
but you don't have to give every conceivable postcondition.

\begin{framed}
\begin{lstlisting}
// typedef ______* image_t;


[*\uanswer{5.5em}{int}*] image_getwidth([*\uanswer{22em}{image\_t \IMG}*])

  [*\uanswer{36.3em}{//@requires \IMG{} != NULL;\hfill}*]

  [*\uanswer{36.3em}{//@ensures \result{} > 0;\hfill}*]

  [*\uanswer{36.3em}{}*]


[*\uanswer{5.5em}{int}*] image_getheight([*\uanswer{21.5em}{image\_t \IMG}*])

  [*\uanswer{36.3em}{//@requires \IMG{} != NULL;\hfill}*]

  [*\uanswer{36.3em}{//@ensures \result{} > 0;\hfill}*]

  [*\uanswer{36.3em}{}*]

\end{lstlisting}
\end{framed}

\newpage
\begin{framed}
\bigskip
\begin{lstlisting}
[*\uanswer{5.5em}{pixel\_t}*] image_getpixel(image_t [*\IMG*], int [*\ROW*], int [*\COL*])

  [*\uanswer{36.3em}{//@requires \IMG{} != NULL;\hfill}*]

  [*\uanswer{36.3em}{//@requires 0 <= \ROW{} \&\& \ROW{} < image\_getheight(\IMG);\hfill}*]

  [*\uanswer{36.3em}{//@requires 0 <= \COL{} \&\& \COL{} < image\_getwidth(\IMG);\hfill}*]

  [*\uanswer{36.3em}{}*]

  [*\uanswer{36.3em}{}*]


[*\uanswer{4.0em}{void}*] image_setpixel(image_t [*\IMG*], int [*\ROW*], int [*\COL*], pixel_t [*\PIXEL*])

  [*\uanswer{36.3em}{//@requires \IMG{} != NULL;\hfill}*]

  [*\uanswer{36.3em}{//@requires 0 <= \ROW{} \&\& \ROW{} < image\_getheight(\IMG);\hfill}*]

  [*\uanswer{36.3em}{//@requires 0 <= \COL{} \&\& \COL{} < image\_getwidth(\IMG);\hfill}*]

  [*\uanswer{36.3em}{//@ensures image\_getpixel(\IMG, \ROW, \COL) == \PIXEL;\hfill}*]

  [*\uanswer{36.3em}{}*]


[*\uanswer{5.5em}{image\_t}*] image_new([*\uanswer{5.5em}{int}*] [*\WIDTH*], [*\uanswer{5.5em}{int}*] [*\HEIGHT*])

  [*\uanswer{36.3em}{//@requires 0 < \WIDTH{} \&\& 0 < \HEIGHT{};\hfill}*]

  [*\uanswer{36.3em}{//@requires \WIDTH{} <= int\_max()/\HEIGHT;\hfill}*]

  [*\uanswer{36.3em}{//@ensures \result{} != NULL;\hfill}*]

  [*\uanswer{36.3em}{//@ensures image\_getwidth(\result) == \WIDTH{};\hfill}*]

  [*\uanswer{36.3em}{//@ensures image\_getheight(\result) == \HEIGHT;\hfill}*]
\end{lstlisting}
\end{framed}

\RUBRIC
Part (a)
TAGS: interface

Gradescope rubric:
+0.25pts Return types and parameters are correct (int, int, pixel_t, void, image_t)
+0.25pts NULL check is performed
+0.5pts All other necessary contracts on get/set functions are present
+0.5pts All other necessary contracts on image_new are present, and uses height with row and width with column consistently.
-0.5pts doesn't check for NULL as the first precondition
-0.5pts Interface Violation! (Dereferencing a pointer or using is_image.)

Commentary:
- the names of the variables [below IMG, ROW, COL, PIXEL, HEIGHT, WIDTH]
  vary from semester to semester: set the rubrics up appropriately
 * RETURN TYPES AND PARAMETERS: 1/4 point, all or nothing
    int image_getwidth(image_t IMG);
    int image_getheight(image_t IMG);
    void image_setpixel
    pixel_t image_getpixel
    image_t image_new(int WIDTH, int HEIGHT)

 * CONTRACTS: [imageutil.c0 functions ARE allowed if used correctly]
    image_getwidth/image_getheight
      //@requires IMG != NULL
      //@ensures \result > 0
    image_getpixel/image_setpixel
      //@requires IMG != NULL
      //@requires 0 <= ROW && ROW < image_getheight(IMG)
      //@requires 0 <= COL && COL < image_getwidth(IMG)
     OR
      //@requires IMG != NULL
      //@requires is_valid_pixel(image_getwidth(IMG), image_getheight(IMG),
                                 ROW, COL);
     ALSO, FOR SET:
      //@ensures image_getpixel(IMG, ROW, COL) == PIXEL        // OPTIONAL
    image_new
      //@requires 0 < WIDTH && 0 < HEIGHT && WIDTH <= int_max() / HEIGHT
        OR
      //@requires is_valid_imagesize(WIDTH, HEIGHT)
      //@ensures \result != NULL
      //@ensures image_getwidth(\result) == WIDTH          // OPTIONAL
      //@ensures image_getheight(\result) == HEIGHT        // OPTIONAL

  . +1/4 point for preconditions that check for NULL and attempt to
    check ***some*** sort of bounds for getpixel/setpixel/new
    (so row <= image_getwidth(IMG) on getpixel and 0 <= WIDTH for image_new
    still gets this point)
  . +1/2 point for not confusing ROW/COL and for getting the full
    image_new precondition (either with the check or with is_valid_imagesize)
  . +1/2 point for non-optional postconditions

  *** EXCEPTION:
  *** If they break the interface anywhere (using is_image or
  *** dereferencing the pointer), then their score is capped at 1 point
ENDRUBRIC


\newpage
\begin{EnvUplevel}
In the \underline{implementation} of the image data type, we
have the following type definitions:

\begin{lstlisting}
struct image_header {
    int width;
    int height;
    pixel_t[] data;
};
typedef struct image_header image;
typedef image* image_t;

\end{lstlisting}

\noindent
And the following data structure invariant:

\begin{lstlisting}[firstnumber=8]
bool is_image(image* [*\IMG*]) {
  return [*\IMG*] != NULL
      && [*\IMG*]->width > 0
      && [*\IMG*]->height > 0
      && [*\IMG*]->width <= int_max() / [*\IMG*]->height
      && is_arr_expected_length([*\IMG*]->data, [*\IMG*]->width * [*\IMG*]->height);
}
\end{lstlisting}

The client does not need to know about this function, since it is the job
of the implementation to preserve the validity of the image data structure.
But the implementation must use this specification function to assure
that the image is valid before and after any image operation.
\end{EnvUplevel}

\part[1\half]\TAGS{correctness, pointer, safety, struct}
\enlargethispage{5ex}
Write an implementation for \lstinline'image_getpixel', assuming
pixels are stored the same way they were stored in the programming
assignment.  Include any necessary preconditions and postconditions
for the implementation.
\begin{framed}
\ifprintanswers
\begin{lstlisting}[basicstyle=\basicstyle\color{\answerColor}]
pixel_t image_getpixel(image_t IMG, int row, int col)
//@requires is_image(IMG);
//@requires 0 <= row && row < image_getheight(IMG);
//@requires 0 <= col && col < image_getwidth(IMG);
{
    return IMG->data[row*(IMG->width) + col];
}
\end{lstlisting}
\else~\vspace{3.6in}\fi
\end{framed}

\RUBRIC
Part (b -- image_getpixel)
TAGS: correctness, pointer, safety, struct

Gradescope rubric:
+0.25pts is_image(IMG) precondition
+0.25pts bounds checks are present (0 <= row && row < IMG->height, etc)
+0.5pts Consistent with interface contracts
+0.5pts Implementation code is correct

Commentary:
  . pixel_t image_getpixel(image_t IMG, int row, int col)
    //@requires is_image(IMG);
    //@requires 0 <= row && row < image_getheight(IMG);
    //@requires 0 <= col && col < image_getwidth(IMG);
    {
        return IMG->data[row*(IMG->width) + col];
    }
  . 1/4 point for is_image(IMG) precondition
      *** IMG != NULL is okay as an ADDITIONAL check, but mark it redundant
  . 1/4 point for repeating the bounds
      *** IMG->width instead of image_getwidth(IMG), etc is OK
  . 1/4 point for implementation

ENDRUBRIC
 % F19 F18 S18 S17 F16 S16
%\enlargethispage{5ex}
Write an implementation for \lstinline'image_setpixel', assuming
pixels are stored the same way they were stored in the programming
assignment.  Include any necessary preconditions and postconditions
for the implementation.
\begin{framed}
\ifprintanswers
\begin{lstlisting}[basicstyle=\basicstyle\color{\answerColor}]
void image_setpixel(image_t IMG, int row, int col, pixel_t P)
//@requires is_image(IMG);
//@requires 0 <= row && row < image_getheight(IMG);
//@requires 0 <= col && col < image_getwidth(IMG);
{
    IMG->data[row*(IMG->width) + col] = P;
}
\end{lstlisting}
\else~\vspace{3.6in}\fi
\end{framed}

\RUBRIC
Part (b -- image_setpixel)
TAGS: correctness, pointer, safety, struct

Gradescope rubric:
+0.25pts is_image(IMG) precondition
+0.25pts bounds checks are present (0 <= row && row < IMG->height, etc)
+0.5pts Consistent with interface contracts
+0.5pts Implementation code is correct

Commentary:
  . void image_setpixel(image_t IMG, int row, int col, pixel_t P)
    //@requires is_image(IMG);
    //@requires 0 <= row && row < image_getheight(IMG);
    //@requires 0 <= col && col < image_getwidth(IMG);
    {
        IMG->data[row*(IMG->width) + col] = P;
    }
  . 1/4 point for is_image(IMG) precondition
      *** IMG != NULL is okay as an ADDITIONAL check, but mark it redundant
  . 1/4 point for repeating the bounds
      *** IMG->width instead of image_getwidth(IMG), etc is OK
  . 1/4 point for implementation

ENDRUBRIC
 % S19 F17


\newpage
\part[1\half]\TAGS{array, correctness, pointer, safety, struct}
%Write an implementation for \lstinline'image_new'. Include any necessary
preconditions and postconditions for the implementation.
\begin{framed}
\ifprintanswers
\begin{lstlisting}[basicstyle=\basicstyle\color{\answerColor}]
image_t image_new(int w, int height)
//@requires 0 < w && 0 < h && w <= int_max() / h;
//@ensures is_image(\result)
{
    image_t IMG = alloc(image);
    IMG->width = w;
    IMG->height = h;
    IMG->data = alloc_array(pixel_t, w*h);
    return IMG;
}
\end{lstlisting}
\else~\vspace{3.5in}\fi
\end{framed}

\RUBRIC
Part (b -- image_new)
TAGS: array, correctness, pointer, safety, struct

Gradescope rubric:
+0.5pts calls alloc(image) or alloc(struct image_header) as well as alloc_array(pixel_t, width * height)
+0.5pts Initializes fields and returns the image
+0.5pts Postcondition is is_image(\result)

Commentary:
. image_t image_new(int w, int height)
  //@requires 0 < w && 0 < h && w <= int_max() / h;
  //@ensures is_image(\result)
  {
      image_t IMG = alloc(image);
      IMG->width = w;
      IMG->height = h;
      IMG->data = alloc_array(pixel_t, w*h);
      return IMG;
  }
. 1/2 point calling alloc(image) or alloc(struct image_header) and for
  calling alloc_array(pixel_t, w*h) somewhere
. 1/2 point initializing AND returning the image
. 1/4 point postcondition
   *** \result != NULL is okay as an ADDITIONAL check, but mark it redundant
. Don't worry about preconditions, we alreay took off for that in part (a)
ENDRUBRIC
 % F18 S17 F16 S16
%Write an implementation for \lstinline'image_getwidth'. Include any necessary
preconditions and postconditions for the implementation.
\begin{framed}
\ifprintanswers
\begin{lstlisting}[basicstyle=\basicstyle\color{\answerColor}]
int image_getwidth(image_t IMG)
//@requires is_image(IMG);
//@ensures \result > 0;
{
  return IMG->width;
}
\end{lstlisting}
\else~\vspace{3.5in}\fi
\end{framed}

\RUBRIC
Part (b -- image_getwidth)
TAGS: array, correctness, pointer, safety, struct

Gradescope rubric:
+0.5pts Precondition is is_image(\result)
+0.5pts Postcondition is \result > 0 or IMG->width or equivalent OR consistent with interface
+0.5pts Correctly returns width

Commentary:
. int image_getwidth(image_t IMG)
  //@requires is_image(IMG);
  //@ensures \result > 0;
  {
    return IMG->width;
  }
. 1/4 pt precondition
   *** \result != NULL is okay as an ADDITIONAL check, but mark it redundant
. 1/4 pt postcondition
. 1/4 pt implementation

. Write a style comment for postconditions such as
	\result == \length(IMG->data) / IMG->height?
ENDRUBRIC
 % S19 F17
Write an implementation for \lstinline'image_getheight'. Include any necessary
preconditions and postconditions for the implementation.
\begin{framed}
\ifprintanswers
\begin{lstlisting}[basicstyle=\basicstyle\color{\answerColor}]
int image_getheight(image_t IMG)
//@requires is_image(IMG);
//@ensures \result > 0;
{
  return IMG->height;
}
\end{lstlisting}
\else~\vspace{3.5in}\fi
\end{framed}

\RUBRIC
Part (b -- image_getheight)
TAGS: array, correctness, pointer, safety, struct

Gradescope rubric:
+0.5pts Precondition is is_image(\result)
+0.5pts Postcondition is \result > 0 or IMG->height or equivalent OR consistent with interface
+0.5pts Correctly returns width

Commentary:
. int image_getheight(image_t IMG)
  //@requires is_image(IMG);
  //@ensures \result > 0;
  {
    return IMG->height;
  }
. 1/4 pt precondition
   *** \result != NULL is okay as an ADDITIONAL check, but mark it redundant
. 1/4 pt postcondition
. 1/4 pt implementation

. Write a style comment for postconditions such as
	\result == \length(IMG->data) / IMG->width?
ENDRUBRIC
 % F19 S18

\end{parts}

\egroup