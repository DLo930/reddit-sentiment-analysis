\part[1]~\vspace*{-2ex}
\begin{lstlisting}[numbers=left]
int f(int a, int b)
//@requires 0 <= a && 2*a < b;
//@requires a <= int_max()/2;
{
  int i = 0;
  while (i < a) {[*\label{l:safe-assert7-lg}*]
    //@assert i < b;      /*** Assertion A ***/[*\label{l:safe-assert7A}*]
    i += 2;
    a += 1;
  }[*\label{l:safe-assert7A-lg}*]
  //@assert a <= i;       /*** Assertion B ***/[*\label{l:safe-assert7B}*]
  return i;
}
\end{lstlisting}

\vfill
\begin{framed}
\medskip
\lstinline'Assertion A' is: \uanswer{26.2em}{UNSUPPORTED}

\medskip
\aLine{\texttt{a} is modified by the loop}{N/A}
\aLine{}{}
\aLine{}{}
\aLine{}{}
\aLine{}{}
Therefore we \noans{can}/\ans{cannot} conclude that\\[0.6ex]
\aLine[-2ex]{\texttt{i < b}}{line~\ref{l:safe-assert7A}}
\end{framed}

\RUBRIC
Part 5A

Gradescope rubric:
+ 0.2 pts UNSUPPORTED
+ 0.3 pts a and i are modified by the loop and the loop invariant does not mention a: logical reasoning cannot conclude anything about how b and i are related.

Commentary:
Assertion A is UNSUPPORTED
  a is modified by the loop, so we cannot
  use line 2 to reason about the relationship between a and i inside
  of the loop, so we have no way of knowing the relationship between i
  and b.
ENDRUBRIC

\vfill
\begin{framed}
\medskip
\lstinline'Assertion B' is: \uanswer{26.2em}{SUPPORTED}

\medskip
\aLine{\texttt{a <= i}}{line~\ref{l:safe-assert7-lg}}
\aLine{}{}
\aLine{}{}
\aLine{}{}
\aLine{}{}
Therefore we \ans{can}/\noans{cannot} conclude that\\[0.6ex]
\aLine[-2ex]{\texttt{a <= i}}{line~\ref{l:safe-assert7B}}
\end{framed}

\RUBRIC
Part 5B

Gradescope rubric:
+ 0.2 pts SUPPORTED
+ 0.3 a <= i by line 5 (negation of the loop guard)

Commentary:
Assertion B is SUPPORTED
   - The statement a <= i is just the negation
     of the loop guard, line 5.
ENDRUBRIC