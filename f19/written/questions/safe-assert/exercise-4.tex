\part[1]~\vspace*{-2ex}
\begin{lstlisting}[numbers=left]
int f(int a, int b)
//@requires 0 <= a;
{
  int i = 0;
  int accum = 1;
  while (i < a)[*\label{l:safe-assert4-lg}*]
  //@loop_invariant accum == POW(b, i);[*\label{l:safe-assert4-LI}*]
  {
    //@assert i <= a;                      /*** Assertion A ***/[*\label{l:safe-assert4A}*]
    accum = accum * b;
    i = i + 1;
  }
  //@assert accum == POW(b, a);            /*** Assertion B ***/[*\label{l:safe-assert4B}*]
  return accum;
}
\end{lstlisting}

\vfill
\begin{framed}
\medskip
\lstinline'Assertion A' is: \uanswer{26.2em}{SUPPORTED}

\medskip
\aLine{\texttt{i < a}}{line~\ref{l:safe-assert4-lg}}
\aLine{}{}
\aLine{}{}
\aLine{}{}
\aLine{}{}
Therefore we \ans{can}/\noans{cannot} conclude that\\[0.6ex]
\aLine[-2ex]{\texttt{i <= a}}{math}
\end{framed}

\RUBRIC
Part 4A

Gradescope rubric:  TODO

Commentary:
Assertion A is SUPPORTED
  by the loop guard condition.
ENDRUBRIC

\vfill
\begin{framed}
\medskip
\lstinline'Assertion B' is: \uanswer{26.2em}{UNSUPPORTED}

\medskip
\aLine{\texttt{accum == POW(b, i)}}{line~\ref{l:safe-assert4-LI}}
\aLine{\texttt{i >= a}}{line~\ref{l:safe-assert4-lg}}
\aLine{but we don't know that \texttt{i <= a}}{}
\aLine{}{}
\aLine{}{}
Therefore we \noans{can}/\ans{cannot} conclude that\\[0.6ex]
\aLine[-2ex]{\texttt{accum == POW(b, a)}}{line~\ref{l:safe-assert4B}}
\end{framed}

\RUBRIC
Part 4B

Gradescope rubric:  TODO

Commentary:
Assertion B is UNSUPPORTED,
   as we only know that i >= a at the end of the loop.
ENDRUBRIC
