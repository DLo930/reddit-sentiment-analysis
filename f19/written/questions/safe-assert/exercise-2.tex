\part[1]~\vspace*{-2ex}
\begin{lstlisting}[numbers=left]
int f(int a)
//@requires 0 <= a;
{
    int i = 2*a;
    while (i > a)[*\label{l:safe-assert2-lg}*]
    //@loop_invariant i >= a;
    {
        //@assert i > 0;      /*** Assertion A ***/[*\label{l:safe-assert2A}*]
        a += 2;
        i += 1;
    }
    //@assert i <= a;         /*** Assertion B ***/[*\label{l:safe-assert2B}*]
    return i;
}
\end{lstlisting}

\vfill
\begin{framed}
\medskip
\lstinline'Assertion A' is: \uanswer{26.2em}{UNSUPPORTED}

\medskip
\aLine{\texttt{i > a}}{line~\ref{l:safe-assert2-lg}}
\aLine{but \texttt{a} is modified by the loop}{}
\aLine{}{}
\aLine{}{}
\aLine{}{}
Therefore we \noans{can}/\ans{cannot} conclude that\\[0.6ex]
\aLine[-2ex]{\texttt{i > 0}}{line~\ref{l:safe-assert2A}}
\end{framed}

\RUBRIC
Part 2A

Gradescope rubric:  TODO

Commentary:
Assertion A is UNSUPPORTED
  - Because a is modified by the loop, we can't use the lower bound
    from line 2.
ENDRUBRIC

\vfill
\begin{framed}
\medskip
\lstinline'Assertion B' is: \uanswer{26.2em}{SUPPORTED}

\medskip
\aLine{\texttt{i <= a}}{line~\ref{l:safe-assert2-lg}}
\aLine{}{}
\aLine{}{}
\aLine{}{}
\aLine{}{}
Therefore we \ans{can}/\noans{cannot} conclude that\\[0.6ex]
\aLine[-2ex]{\texttt{i <= a}}{line~\ref{l:safe-assert2B}}
\end{framed}

\RUBRIC
Part 2B

Gradescope rubric:  TODO

Commentary:
Assertion B is SUPPORTED
  - This is directly the negation of the loop guard
ENDRUBRIC
