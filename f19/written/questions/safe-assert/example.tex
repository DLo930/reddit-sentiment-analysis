%\part[0]~\vspace*{-2ex}
\begin{lstlisting}[numbers=left]
int f(int a, int b)
//@requires 1 <= a && a < b;
{
    int i = 1;
    while (i < a)
    //@loop_invariant i >= 1;
    {
        //@assert i < b;        /*** Assertion 1 ***/
        i += 1;
    }
    //@assert i == a;           /*** Assertion 2 ***/
    //@assert i != 0;           /*** Assertion 3 ***/
    return i;
}
\end{lstlisting}

%\vfill
\begin{framed}
\topsep=0.25ex
\partopsep=0.25ex
\lstinline'Assertion 1' is: \underline{SUPPORTED}
\begin{tabbing}
--~~\= \lstinline'i < a'\qquad\= (line 5)\\
--\> \lstinline'a < b'\> (line 2) and \lstinline'a' and \lstinline'b' not changed by loop
\end{tabbing}
Therefore, we \textbf{can} conclude that
\begin{tabbing}
--~~\= \lstinline'i < b'\qquad\= since \lstinline'i < a' and \lstinline'a < b' implies \lstinline'i < b'
\end{tabbing}

(\emph{Long version}: Because the assignables \lstinline'a' and
\lstinline'b' are not modified by the loop, the assertion %
\lstinline'a < b' from line 2 can be used at line 6. Because we are inside the
loop, we know the loop guard held at the beginning of the loop, so
line 5 gives us that \lstinline'i < a'. The facts \lstinline'i < a'
and \lstinline'a < b' together imply \lstinline'i < b'.)
\end{framed}

\vfill
\begin{framed}
\topsep=0.25ex
\partopsep=0.25ex
\lstinline'Assertion 2' is: \underline{UNSUPPORTED}

\begin{tabbing}
--~~\= \lstinline'!(i < a)'\qquad\= (line 5)\\
--\>   \lstinline'i >= a'\qquad\> (by math)
\end{tabbing}
Therefore, we \textbf{cannot} conclude that
\begin{tabbing}
--~~\= \lstinline'i == a'\qquad\quad\= (does not follow logically from any fact we know)
\end{tabbing}

% At line 9, we know that the loop guard \lstinline'i < a' is false ---
% that is, we know that \lstinline'!(i < a)', which is the same thing as
% saying \lstinline'i >= a'. We can't conclude, from this, that \lstinline'i' is
% equal to \lstinline'a'.
\end{framed}

\vfill
\begin{framed}
\topsep=0.25ex
\partopsep=0.25ex
\lstinline'Assertion 3' is: \underline{SUPPORTED}

\begin{tabbing}
--~~\= \lstinline'i >=1'\qquad\= (line 6)
\end{tabbing}
Therefore, we \textbf{can} conclude that
\begin{tabbing}
--~~\= \lstinline'i != 0'\qquad\= (by math)
\end{tabbing}

% After the loop, we know that the loop invariant \lstinline'i >= 1'
% still holds. We can conclude from this, that \lstinline'i' is not
% equal to \lstinline'0'.
\end{framed}
