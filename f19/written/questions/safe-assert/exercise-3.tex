\part[1]~\vspace*{-2ex}
\begin{lstlisting}[numbers=left]
int f(int n) {
    int x = 0;
    while (n >= 2) {[*\label{l:safe-assert3-lg}*]
        x += 1;
        if (n % 2 == 1) {
            n = 3*n + 1;
        } else {
            n = n/2;
        }
    }
    //@assert n <= 2;   /*** Assertion A ***/[*\label{l:safe-assert3A}*]
    //@assert x >= 0;   /*** Assertion B ***/[*\label{l:safe-assert3B}*]
    return x;
}
\end{lstlisting}

\vfill
\begin{framed}
\medskip
\lstinline'Assertion A' is: \uanswer{26.2em}{SUPPORTED}

\medskip
\aLine{\texttt{n < 2}}{line~\ref{l:safe-assert3-lg}}
\aLine{}{}
\aLine{}{}
\aLine{}{}
\aLine{}{}
Therefore we \ans{can}/\noans{cannot} conclude that\\[0.6ex]
\aLine[-2ex]{\texttt{n <= 2}}{math}
\end{framed}

\RUBRIC
Part 3A

Gradescope rubric:
+ 0.2 pts SUPPORTED
+ 0.15 pts n < 2 by line 3 (negation of the loop guard)
+ 0.15 pts Therefore n <= 2

Commentary:
Assertion A is SUPPORTED
  - The negation of the loop guard gives n < 2, which implies n <= 2
ENDRUBRIC

\vfill
\begin{framed}
\medskip
\lstinline'Assertion B' is: \uanswer{26.2em}{UNSUPPORTED}

\medskip
\aLine{\texttt{x} is modifed by the loop}{}
\aLine{}{}
\aLine{}{}
\aLine{}{}
\aLine{}{}
Therefore we \noans{can}/\ans{cannot} conclude that\\[0.6ex]
\aLine[-2ex]{\texttt{x >= 0}}{line~\ref{l:safe-assert3B}}
\end{framed}

\RUBRIC
Part 3B

Gradescope rubric:
+ 0.2 UNSUPPORTED
+ 0.3 x is modified by the loop and there are no loop invariants: logical reasoning gives no information about it.

Commentary:
Assertion B is UNSUPPORTED
  - Because x is modified by the loop and unmentioned by the loop
    guard and the (trivial) loop invariant, we don't know anything
    about x after the loop body runs.
ENDRUBRIC
