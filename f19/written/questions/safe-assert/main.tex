\clearpage
\Question{Assertions in Loops}\TAGS{correctness}

This question involves a series of functions \lstinline'f' with one
loop; each contains additional \assert{} statements. None of the
assertions will ever fail --- they will never evaluate to
\lstinline'false' when the function \lstinline'f' is called with
arguments that satisfy the precondition. However, if our loop
invariants aren't up to the task, we may not be able to \emph{prove}
these assertions hold. The distinction between an assertion being
\emph{true} and an assertion being \emph{supported} is a subtle but
important one.

To support an assertion one may use the following facts:
\begin{itemize}
\item%
  When local variables are \emph{untouched} by a loop, statements we
  know to be true about those variables \emph{before} the loop remain
  valid \emph{inside} the loop and \emph{after} the loop.

\item%
  For local variables that are modified by the loop, the loop guard
  and the loop invariants are the only statements we can use.
  \begin{itemize}
  \item%
    Inside of a loop, we know that the loop invariants held just before the
    loop guard was checked and that the loop guard returned \lstinline'true'.
  \item%
    After a loop, we know that the loop invariants held just before the loop
    guard was checked for the last time and that the loop guard returned
    \lstinline'false'.
  \end{itemize}
\end{itemize}

For each of the problems below, \textbf{state whether each assertion
is SUPPORTED or UNSUPPORTED and explain your reasoning}.
You can assume that the
loop invariant is true initially (before the loop guard is checked the
first time) and that it is preserved by any iteration of the loop. If
you claim that the assertion is supported, your answer should be a
concise proof; if you claim that the assertion is unsupported, we only
expect an informal argument to explain why.

We've given one worked out solution below.


\newpage
\section*{Converting between binary and decimal%
\TAGS{ints}}

\bgroup
\newcommand{\base}[1]{\textcolor{Brown}{#1}}
\newcommand{\bin}[1]{\textcolor{blue}{#1}}
\newcommand{\dec}[1]{\textcolor{red}{#1}}
\newcommand{\blankSize}{2.5em}
\newcommand{\blank}[1][]{\underline{\makebox[\blankSize][r]{#1\;\;}}}
\newcommand{\blankrow}[3]{%
  \blank[#1] \,\times\, \base{2} \,+\, \blank[\bin{#2}] \;=\; \blank[#3]}

To easily convert a number represented in binary notation, such as
$\bin{10100}_{[\base{2}]}$, we can employ \emph{Horner's algorithm}. At
each step, we multiply the previous result by \base{2}, and add the next
bit in the number. To convert in the other direction, we divide by \base{2}
and write the remainder at each step from bottom to top. We can see
the conversion between $\bin{10100}_{[\base{2}]}$ and \dec{20} (or
$\dec{20}_{[\base{10}]}$ to be extra-decimaly) below.
$$
\begin{array}{@{}l@{\hspace{0.9em}}
             |@{\hspace{0.9em}}l@{\hspace{0.9em}}
             |@{\hspace{0.9em}}l@{}}
\mathstrut
   \blankrow{}{}{}      & \blankrow{}{}{} & \blankrow{}{}{}
\\ \blankrow{}{}{}      & \blankrow{}{}{} & \blankrow{}{}{}
\\ \blankrow{}{1}{1}    & \blankrow{}{}{} & \blankrow{}{}{}
\\ \blankrow{1}{0}{2}   & \blankrow{}{}{} & \blankrow{}{}{}
\\ \blankrow{2}{1}{5}   & \blankrow{}{}{} & \blankrow{}{}{}
\\ \blankrow{5}{0}{10}  & \blankrow{}{}{} & \blankrow{}{}{}
\\ \blankrow{10}{0}{\dec{20}} & \blankrow{}{}{} & \blankrow{}{}{}
\end{array}
$$
\egroup

\vspace{-1ex}

\RUBRIC
General Instructions:
  - Each assertion is worth half point, possibly with partial credit
  - Justifications for "unsupported" have to exist. They could get full
    credit if they are not aggressively wrong.
  - Leaving off the word "supported" or "unsupported" but making it
    clear what you meant with the justification is fine.
ENDRUBRIC

\begin{parts}
\newcommand{\aLine}[3][1.5ex]
{\noindent\uanswer{21em}{#2}\hfill~(by \uanswer{10em}{#3})\\[#1]}%
\newcommand{\noans}[1]{\textbf{#1}}
\newcommand{\ans}[1]{%
  \ifprintanswers%
  \textcolor{\answerColor}{\fbox{\noans{#1}}}%
  \else\noans{#1}\fi}

\RUBRIC
TAGS: correctness
ENDRUBRIC

%\newpage\part[0\half]
This is the \textbf{exact same code} from the previous example, but
because there is a different loop invariant, the answers, which are
given, have changed. You just have to provide explanations.

\smallskip
\begin{lstlisting}[numbers=left, aboveskip=0pt, belowskip=0pt]
void fill(int x, int[] A)
/*@requires x == \length(A); @*/;

int[] f(int x, int[] B)
//@requires 1 <= x && x < \length(B);
{
    int i = 1;
    int[] A = alloc_array(int, x);[*\label{l:fillex1-alloc_array}*]

    while (i < x)[*\label{l:fillex1-lg}*]
    //@loop_invariant i <= \length(A);[*\label{l:fillex1-LI}*]  // MODIFIED
    {
        B[i] = B[i] - 3;[*\label{l:fillex1-A[i]}*]
        i += 1;
    }

    fill(i, A);[*\label{l:fillex1-fill}*]
    return A;
}
\end{lstlisting}

\enlargethispage{5ex}\vfill
\begin{framed}
Safety of the array access on line~\ref{l:fillex1-A[i]} is:
\underline{UNSUPPORTED}.

\medskip
\setcounter{lineCounter}{0}
\aLine{We have no direct support for \lstinline'0 <= i'.\hfill}{}
\aLine{}{}
\aLine{}{}
\aLine{}{}
Therefore we \textbf{cannot} conclude that\\[0.6ex]
\aLine[-2.5ex]{\lstinline'0 <= i'}{~~~~N/A}
\end{framed}

\vfill
\begin{framed}
Safety of the function call on line~\ref{l:fillex1-fill} is:
\underline{SUPPORTED}.

\medskip
\setcounter{lineCounter}{0}
\aLine{i >= x}{line~\ref{l:fillex1-lg}}
\aLine{x = \length{(A)}}{line~\ref{l:fillex1-alloc_array}}
\aLine{i <= \length{(A)}}{line~\ref{l:fillex1-LI}}
\aLine{}{}
Therefore we \textbf{can} conclude that\\[0.6ex]
\aLine[-2.5ex]{i == \length{(A)}}{line~\ref{l:fillex1-fill}}
\end{framed}

\vfill

\RUBRIC
Part (b)

Gradescope rubric:
+0.25pt First: we have no support for 0 <= i
+0.25pt Second box: right idea (combines <= and >= to get ==)
+0.25pt Second box: fully correct, uses lines 8, 10 ("loop guard"), and 11 ("loop invariant")

Commentary:
- Line 13 is UNSUPPORTED
  . We have no direct support for 0 <= i
    [it must be the case that 0 <=i on the basis of operational
    reasoning, but not logical reasoning.]
- Line 17 is SUPPORTED
  . i >= x by line 10 (negation of loop invariant)
  . x = \length(A) by line 8
  . i <= \length(A) by line 11
  . So i == \length(A) on line 17

1 point per justification. 1/2 pt credit if partially correct.
ENDRUBRIC
 % S16
%\newpage\part[1]~\vspace*{-2ex}
\begin{lstlisting}[numbers=left]
int f(int a)
//@requires 0 <= a;
{
    int i = 2*a;
    while (i > a)[*\label{l:safe-assert2-lg}*]
    //@loop_invariant i >= a;
    {
        //@assert i > 0;      /*** Assertion A ***/[*\label{l:safe-assert2A}*]
        a += 2;
        i += 1;
    }
    //@assert i <= a;         /*** Assertion B ***/[*\label{l:safe-assert2B}*]
    return i;
}
\end{lstlisting}

\vfill
\begin{framed}
\medskip
\lstinline'Assertion A' is: \uanswer{26.2em}{UNSUPPORTED}

\medskip
\aLine{\texttt{i > a}}{line~\ref{l:safe-assert2-lg}}
\aLine{but \texttt{a} is modified by the loop}{}
\aLine{}{}
\aLine{}{}
\aLine{}{}
Therefore we \noans{can}/\ans{cannot} conclude that\\[0.6ex]
\aLine[-2ex]{\texttt{i > 0}}{line~\ref{l:safe-assert2A}}
\end{framed}

\RUBRIC
Part 2A

Gradescope rubric:  TODO

Commentary:
Assertion A is UNSUPPORTED
  - Because a is modified by the loop, we can't use the lower bound
    from line 2.
ENDRUBRIC

\vfill
\begin{framed}
\medskip
\lstinline'Assertion B' is: \uanswer{26.2em}{SUPPORTED}

\medskip
\aLine{\texttt{i <= a}}{line~\ref{l:safe-assert2-lg}}
\aLine{}{}
\aLine{}{}
\aLine{}{}
\aLine{}{}
Therefore we \ans{can}/\noans{cannot} conclude that\\[0.6ex]
\aLine[-2ex]{\texttt{i <= a}}{line~\ref{l:safe-assert2B}}
\end{framed}

\RUBRIC
Part 2B

Gradescope rubric:  TODO

Commentary:
Assertion B is SUPPORTED
  - This is directly the negation of the loop guard
ENDRUBRIC
 % S16
\newpage\part[0\half]~\vspace*{-2ex}
\begin{lstlisting}[numbers=left]
int f(int a, int b)
//@requires 0 <= a && a <= b;[*\label{l:safe-assert5-prec}*]
{
    int i = 0;
    while (i < a)[*\label{l:safe-assert5-lg}*]
    //@loop_invariant 0 <= i;[*\label{l:safe-assert5-LI}*]
    {
        //@assert 0 <= i && i < b;          /*** Assertion A ***/[*\label{l:safe-assert5A}*]
        i += 1;
    }
    //@assert i <= b;                       /*** Assertion B ***/[*\label{l:safe-assert5B}*]
    return i;
}
\end{lstlisting}

\vfill
\begin{framed}
\lstinline'Assertion A' is: \underline{SUPPORTED}
%\medskip
%\lstinline'Assertion A' is: \uanswer{26.2em}{SUPPORTED}

\medskip
\aLine{\texttt{0 <= i}}{line~\ref{l:safe-assert5-LI}}
\aLine{\texttt{i < a}}{line~\ref{l:safe-assert5-LI}}
\aLine{\texttt{a <= b}}{line~\ref{l:safe-assert5-prec}}
\aLine{\texttt{i < b}}{math}
\aLine{}{}
%Therefore we \ans{can}/\noans{cannot} conclude that\\[0.6ex]
Therefore we \textbf{can} conclude that\\[0.6ex]
\aLine[-2ex]{0 <= i \&\& i < b}{math}
\end{framed}

\RUBRIC
Part 5A

Gradescope rubric:
+ 0.1 pts 0 <= i from line 6
+ 0.15 pts i < b from lines 5 and 2

Commentary:
Assertion A is SUPPORTED
  - 0 <= i from line 6
  - i < b from lines 5 and 2 - (i < a) && (a <= b) => i < b
ENDRUBRIC

\vfill
\begin{framed}
\lstinline'Assertion B' is: \underline{UNSUPPORTED}
%\medskip
%\lstinline'Assertion B' is: \uanswer{26.2em}{UNSUPPORTED}

\medskip
\aLine{\texttt{i >= a}}{line~\ref{l:safe-assert5-lg}}
\aLine{\texttt{a <= b}}{line~\ref{l:safe-assert5-prec}}
\aLine{that doesn't tell us how \texttt{i} and \texttt{b} are related}{}
\aLine{}{}
\aLine{}{}
Therefore we \textbf{cannot} conclude that\\[0.6ex]
%Therefore we \textbf{can}/\textbf{cannot} conclude that\\[0.6ex]
\aLine[-2ex]{\texttt{i <= b}}{line~\ref{l:safe-assert5B}}
\end{framed}

\RUBRIC
Part 5B

Gradescope rubric:
+ 0.1 pts i >= a from line 5
+ 0.15 pts Cannot conclude that i <= b

Commentary:
Assertion B is UNSUPPORTED
   - We know i >= a from line 5,
     so i >= a,
      but that doesn't mean it's < b.
ENDRUBRIC % S19 S17
\newpage\part[1]~\vspace*{-2ex}
\begin{lstlisting}[numbers=left, belowskip=0pt]
int h(int n) {
    int x = 1;
    while (n >= 25) {[*\label{l:hex3-lg}*]
        x += 1;
        if (n % 2 == 1) {
            n = 3*n + 1;
        } else {
            n = n/2;
        }
    }

    if (n >= 0) {[*\label{l:hex3-if}*]
        return x << n;[*\label{l:hex3-shift}*]
    }

    return 1000000 / x;[*\label{l:hex3-div}*]
}
\end{lstlisting}

\vfill
\begin{framed}
\medskip
Safety of the left-shift on line~\ref{l:hex3-shift} is:
\uanswer{18em}{SUPPORTED}

\medskip
\setcounter{lineCounter}{0}
\aLine{\texttt{0 <= n}}{line~\ref{l:hex3-if}}
\aLine{\texttt{n < 25}}{line~\ref{l:hex3-lg}}
\aLine{}{}
\aLine{}{}
Therefore we \ans{can}/\noans{cannot} conclude that\\[0.6ex]
\aLine[-2ex]{\texttt{0 <= n < 32}}{line~\ref{l:hex3-shift}}
\end{framed}

\vfill
\begin{framed}
\medskip
Safety of the division on line~\ref{l:hex3-div} is:
\uanswer{18.5em}{UNSUPPORTED}

\medskip
\setcounter{lineCounter}{0}
\aLine{We know nothing about \texttt{x} after the loop.}{N/A}
\aLine{}{}
\aLine{}{}
\aLine{}{}
Therefore we \noans{can}/\ans{cannot} conclude that\\[0.6ex]
\aLine[-2ex]{\texttt{x != 0}}{~~~~N/A}
\end{framed}

\RUBRIC
Part (d)

Gradescope rubric:
+0.25pt First box: SUPPORTED
+0.25pt First box: Correctly justified (Negation of loop guard on line 3 and condition on line 12)
+0.25pt Second box: UNSUPPORTED
+0.25pt Second box: Correctly justified (There isn't evidence that tells us that x != 0)

Commentary:
- Line 13 is SUPPORTED
  . 0 <= n by line 12
  . n < 25 by line 3
  . So 0 <= n < 32 on line 13
- Line 16 is UNSUPPORTED
  . We know nothing about x after the loop.
  . So we can't conclude that x != 0
  . [Concrete counterexample: make a call to h(1)

Half point per (UN)SUPPORTED, if they get that right they can get
a full point for the justification (so each part is 1.5 points).
ENDRUBRIC
 % S19 S16 S17
%\newpage\part[1]~\vspace*{-2ex}
\begin{lstlisting}[numbers=left]
int f(int a, int b)
//@requires 0 <= a;
{
  int i = 0;
  int accum = 1;
  while (i < a)[*\label{l:safe-assert4-lg}*]
  //@loop_invariant accum == POW(b, i);[*\label{l:safe-assert4-LI}*]
  {
    //@assert i <= a;                      /*** Assertion A ***/[*\label{l:safe-assert4A}*]
    accum = accum * b;
    i = i + 1;
  }
  //@assert accum == POW(b, a);            /*** Assertion B ***/[*\label{l:safe-assert4B}*]
  return accum;
}
\end{lstlisting}

\vfill
\begin{framed}
\medskip
\lstinline'Assertion A' is: \uanswer{26.2em}{SUPPORTED}

\medskip
\aLine{\texttt{i < a}}{line~\ref{l:safe-assert4-lg}}
\aLine{}{}
\aLine{}{}
\aLine{}{}
\aLine{}{}
Therefore we \ans{can}/\noans{cannot} conclude that\\[0.6ex]
\aLine[-2ex]{\texttt{i <= a}}{math}
\end{framed}

\RUBRIC
Part 4A

Gradescope rubric:  TODO

Commentary:
Assertion A is SUPPORTED
  by the loop guard condition.
ENDRUBRIC

\vfill
\begin{framed}
\medskip
\lstinline'Assertion B' is: \uanswer{26.2em}{UNSUPPORTED}

\medskip
\aLine{\texttt{accum == POW(b, i)}}{line~\ref{l:safe-assert4-LI}}
\aLine{\texttt{i >= a}}{line~\ref{l:safe-assert4-lg}}
\aLine{but we don't know that \texttt{i <= a}}{}
\aLine{}{}
\aLine{}{}
Therefore we \noans{can}/\ans{cannot} conclude that\\[0.6ex]
\aLine[-2ex]{\texttt{accum == POW(b, a)}}{line~\ref{l:safe-assert4B}}
\end{framed}

\RUBRIC
Part 4B

Gradescope rubric:  TODO

Commentary:
Assertion B is UNSUPPORTED,
   as we only know that i >= a at the end of the loop.
ENDRUBRIC

%\newpage\part[1]~\vspace*{-2ex}
\begin{lstlisting}[numbers=left]
int f(int a, int b)
//@requires 0 <= a && a <= b;[*\label{l:safe-assert6-prec}*]
{
  int i = 0;
  while (i < a)[*\label{l:safe-assert6-lg}*]
    //@loop_invariant i <= a;[*\label{l:safe-assert6-LI}*]
    {
      //@assert i < b;        /*** Assertion A ***/[*\label{l:safe-assert6A}*]
      i += 1;
    }
  //@assert i == a;           /*** Assertion B ***/[*\label{l:safe-assert6B}*]
  return i;
}
\end{lstlisting}

\vfill
\begin{framed}
\medskip
\lstinline'Assertion A' is: \uanswer{26.2em}{SUPPORTED}

\medskip
\aLine{\texttt{i < a}}{line~\ref{l:safe-assert6-lg}}
\aLine{\texttt{a <= b}}{line~\ref{l:safe-assert6-prec}}
\aLine{}{}
\aLine{}{}
\aLine{}{}
Therefore we \ans{can}/\noans{cannot} conclude that\\[0.6ex]
\aLine[-2ex]{\texttt{i < b}}{math}
\end{framed}

\RUBRIC
Part 6A

Gradescope rubric:  TODO

Commentary:
Assertion A is SUPPORTED
  - i < b from lines 5 and 2 - (i < a) && (a <= b) => i < b
ENDRUBRIC

\vfill
\begin{framed}
\medskip
\lstinline'Assertion B' is: \uanswer{26.2em}{SUPPORTED}

\medskip
\aLine{\texttt{i <= a}}{line~\ref{l:safe-assert6-LI}}
\aLine{\texttt{i >= a}}{line~\ref{l:safe-assert6-lg}}
\aLine{}{}
\aLine{}{}
\aLine{}{}
Therefore we \ans{can}/\noans{cannot} conclude that\\[0.6ex]
\aLine[-2ex]{\texttt{i == a}}{math}
\end{framed}

\RUBRIC
Part 6B

Gradescope rubric:  TODO

Commentary:
Assertion B is SUPPORTED
  - i <= a from line 6
  - i >= a from line 5
  - Together these imply i == a
ENDRUBRIC
\newpage\part[1]~\vspace*{-2ex}
\begin{lstlisting}[numbers=left]
int f(int a, int b)
//@requires 0 <= a && 2*a < b;
//@requires a <= int_max()/2;
{
  int i = 0;
  while (i < a) {[*\label{l:safe-assert7-lg}*]
    //@assert i < b;      /*** Assertion A ***/[*\label{l:safe-assert7A}*]
    i += 2;
    a += 1;
  }[*\label{l:safe-assert7A-lg}*]
  //@assert a <= i;       /*** Assertion B ***/[*\label{l:safe-assert7B}*]
  return i;
}
\end{lstlisting}

\vfill
\begin{framed}
\medskip
\lstinline'Assertion A' is: \uanswer{26.2em}{UNSUPPORTED}

\medskip
\aLine{\texttt{a} is modified by the loop}{N/A}
\aLine{}{}
\aLine{}{}
\aLine{}{}
\aLine{}{}
Therefore we \noans{can}/\ans{cannot} conclude that\\[0.6ex]
\aLine[-2ex]{\texttt{i < b}}{line~\ref{l:safe-assert7A}}
\end{framed}

\RUBRIC
Part 5A

Gradescope rubric:
+ 0.2 pts UNSUPPORTED
+ 0.3 pts a and i are modified by the loop and the loop invariant does not mention a: logical reasoning cannot conclude anything about how b and i are related.

Commentary:
Assertion A is UNSUPPORTED
  a is modified by the loop, so we cannot
  use line 2 to reason about the relationship between a and i inside
  of the loop, so we have no way of knowing the relationship between i
  and b.
ENDRUBRIC

\vfill
\begin{framed}
\medskip
\lstinline'Assertion B' is: \uanswer{26.2em}{SUPPORTED}

\medskip
\aLine{\texttt{a <= i}}{line~\ref{l:safe-assert7-lg}}
\aLine{}{}
\aLine{}{}
\aLine{}{}
\aLine{}{}
Therefore we \ans{can}/\noans{cannot} conclude that\\[0.6ex]
\aLine[-2ex]{\texttt{a <= i}}{line~\ref{l:safe-assert7B}}
\end{framed}

\RUBRIC
Part 5B

Gradescope rubric:
+ 0.2 pts SUPPORTED
+ 0.3 a <= i by line 5 (negation of the loop guard)

Commentary:
Assertion B is SUPPORTED
   - The statement a <= i is just the negation
     of the loop guard, line 5.
ENDRUBRIC % S19 S17
\end{parts}
