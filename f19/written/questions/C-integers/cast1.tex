\part[2]\TAGS{c-numbers, implementation-defined}
Suppose that we are working with the usual 2's complement
implementation of unsigned and signed \lstinline'char' (8 bits, one
byte), \lstinline'short' (16 bits, two bytes) and \lstinline'int' (32
bits, four bytes).

We begin with the following declarations:
\begin{quote}
\begin{lstlisting}[aboveskip=2pt, belowskip=2pt]
  signed char the_char = -7;
  unsigned char un_char_1 = 248;
  unsigned char un_char_2 = 5;
  int the_int = -247;
\end{lstlisting}
\end{quote}

Fill in the table below. In the third column, always use two hex
digits to represent a \lstinline'char', four hex digits to represent a
\lstinline'short', and eight hex digits to represent an
\lstinline'int'.  You might find these numbers useful: $2^8 = 256$,
$2^{16} = 65536$ and $2^{32} = 4294967296$.

Most, but not all, of these answers can be derived from the lecture
notes.  If you can't find an answer from the lecture notes, you can
look at online C references or just compile some code.

\enlargethispage{3ex}
\medskip
\begin{center}
\newcommand{\ans}[1]{\answer{8em}{\tt #1}}
\renewcommand{\arraystretch}{1.6}
\begin{tabular}{@{}| l | p{8.5em} | p{8.5em} |@{}}
  \hline
  {\bf C expression} & {\bf Decimal value} & {\bf Hexadecimal}
\\[0.5ex]\hline
  \lstinline'the_char'                     & \lstinline'-7'   & \lstinline'0xF9'
\\[0.5ex]\hline
  \lstinline'(unsigned char) the_char'     & \lstinline'249'    & \lstinline'0xF9'
\\[0.5ex]\hline
  \lstinline'(int) the_char'               & \lstinline'-7'     & \lstinline'0xFFFFFFF9'
\\[0.5ex]\hline
  \lstinline'un_char_1'                    & \lstinline'248'    & \ans{0xF8}
\\[0.5ex]\hline
  \lstinline'(int)(signed char)un_char_1'  & \ans{-8}           & \ans{0xFFFFFFF8}
\\[0.5ex]\hline
  \lstinline'(int)(unsigned int)un_char_1' & \ans{248}          & \ans{0x000000F8}
\\[0.5ex]\hline
  \lstinline'un_char_2'                    & \lstinline'5'      & \lstinline'0x05'
\\[0.5ex]\hline
  \lstinline'(int)(signed char)un_char_2'  & \ans{5}            & \ans{0x00000005}
\\[0.5ex]\hline
  \lstinline'(int)(unsigned int)un_char_2' & \ans{5}            & \ans{0x00000005}
\\[0.5ex]\hline
  \lstinline'the_int'                      & \lstinline'-247'   & \ans{0xFFFFFF09}
\\[0.5ex]\hline
  \lstinline'(unsigned int)the_int'        & \ans{$2^{32}-247$ or $4294967049$} & \ans{0xFFFFFF09}
\\[0.5ex]\hline
  \lstinline'(char)the_int'                & \ans{9}            & \ans{0x09}
\\[0.5ex]\hline
  \lstinline'(short)the_int'               & \ans{-247}         & \ans{0xFF09}
\\[0.5ex]\hline
  \lstinline'(unsigned short)the_int'      & \ans{$2^{16}-247$ or $65289$} & \ans{0xFF09}
\\[0.5ex]\hline
\end{tabular}
\end{center}

\RUBRIC
Part (a)
TAGS: c-numbers, implementation-defined

Gradescope rubric:
-0.0pt  Negative scoring question - incorrect answers are highlighted
-0.25pt Missing leading 0's anywhere
-0.25pt One incorrect answer (except missing leading 0's)
-0.5pt  Two more incorrect answers (except missing leading 0's)
-0.5pt  Two more incorrect answers (except missing leading 0's)
-0.5pt  Two more incorrect answers (except missing leading 0's)
-0.25pt One last incorrect answer  (except missing leading 0's)

Commentary:
Use Gradescope ``draw a box'' widget to highlight incorrect answers,
then apply rubric

This makes more sense as negative grading: 1/4 a point off for each
incorrect blank until all points lost. Only take off 1/4 point once
for omitting leading zeroes.

Be picky about the number of hex-digits, but generous about using
mathematical notation in the "integer" column.

un_char_1                        248     0xF8
(int)(signed char)un_char_1      -8      0xFFFFFFF8
(int)(unsigned int)un_char_1     248     0xFFFFFFF8

(int)(signed char)un_char_2      5       0x00000005
(int)(unsigned int)un_char_2     5       0x00000005
the_int                          -247    0xFFFFFF09
(unsigned int)the_int            2^32-247 (or 4294967049) 0xFFFFFF09
(char)the_int                    9       0x09
(short)the_int                   -247    0xff09
(unsigned short)the_int          2^16-247 (or 65289)      0xFF09
ENDRUBRIC
