\part[1]\TAGS{c-array, c-memory, function-pointer}
Suppose we've defined the following functions:

\medskip
\begin{lstlisting}
int fib(int n);  // returns the nth fibonacci number
int cat(int n);  // returns the nth catalan number
int las(int n);  // returns the nth look-and-say number
\end{lstlisting}

\medskip
Complete the code below such that it will print

\begin{quote}
\begin{lstlisting}
2 3 5 0 1 1
5 14 42 1 1 2
1211 111221 312211 1 11 21
\end{lstlisting}
\end{quote}

(\emph{Hint:} The \lstinline'typedef' on the first line should define
the type \lstinline'int2int_fn'. This type should match the type of a
function such as \lstinline'fib', \lstinline'cat', or
\lstinline'las'.)

\medskip
\begin{framed}
\begin{lstlisting}
typedef [*\uanswer{32em}{int int2int\_fn(int)}*];

void map_print(int2int_fn* f, int* A, size_t n) {
  for (size_t i = 0; i < n; i++) {

    int x = [*\uanswer{30em}{(*f)(A[i])}*];
    printf("%d ", x);
  }
  printf("\n");
}

int main() {
  int A[6] = {3, 4, 5, 0, 1, 2};

  map_print([*\uanswer{26em}{\&fib}*], A, 6);

  map_print([*\uanswer{26em}{\&cat}*], A, 6);

  map_print([*\uanswer{26em}{\&las}*], A, 6);
  return 0;
}
\end{lstlisting}
\end{framed}

\RUBRIC
Part <int2int-1>
TAGS: c-array, c-memory, function-pointer

Gradescope rubric:
+ 0.25pts  typedef int int2int_fn(int);
+ 0.25pts  int x = (*f)(A[i]);
+ 0.5pt  EITHER --- all 3 calls are correct (e.g., map_print(&fib, A, 6);)
+ 0.25pts OR --- at least 2 calls are correct

Commentary:
-0.25 for each line that is wrong, max -2 (0 for the subquestion)

Solution:
typedef int int2int_fn(int);

void map_print(int2int_fn* f, int* A, size_t n) {
  for (size_t i = 0; i < n; i++) {
    int x = (*f)(A[i]);
    printf("%d ", x);
  }
  printf("\n");
}

int main() {
  int A[6] = {0, 1, 2, 3, 4, 5};
  map_print(&fib, A, 6);
  map_print(&cat, A, 6);
  map_print(&las, A, 6);
  return 0;
}
ENDRUBRIC
