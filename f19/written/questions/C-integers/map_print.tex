\part[2]\TAGS{c-array, c-memory, function-pointer}
Suppose we've defined the following functions:

\medskip
\begin{lstlisting}[numbers=none]
int fib(int n);  // returns the nth fibonacci number
int cat(int n);  // returns the nth catalan number
int las(int n);  // returns the nth look-and-say number
\end{lstlisting}

\medskip
Complete the code below such that it will print

\begin{quote}
\begin{lstlisting}[numbers=none]
0 1 1 2 3 5
1 1 2 5 14 42
1 11 21 1211 111221 312211
\end{lstlisting}
\end{quote}

(\emph{Hint:} The \lstinline'typedef' on the first line should define
the type \lstinline'int2int'. This type should match the type of a
function pointer to either \lstinline'fib', \lstinline'cat', or
\lstinline'las'.)
\begin{framed}
\begin{lstlisting}

typedef [*\uanswer{30em}{int*~int2int(int)}*];

void map_print(int2int fn, int* A, size_t n) {
  for (size_t i = 0; i < n; i++) {

    int x = [*\uanswer{28em}{(*fn)(A[i])}*];
    printf("%d ", x);
  }
  printf("\n");
}

int main() {
  int A[6] = {0, 1, 2, 3, 4, 5};

  map_print([*\uanswer{16em}{\&fib}*], A, 6);

  map_print([*\uanswer{16em}{\&cat}*], A, 6);

  map_print([*\uanswer{16em}{\&las}*], A, 6);
  return 0;
}
\end{lstlisting}
\end{framed}

\RUBRIC
Part <map_print>
TAGS: c-array, c-memory, function-pointer

Gradescope rubric:  TODO

Commentary:
-0.5 for each line that is wrong, max -2 (0 for the subquestion)

Solution:
typedef __int* int2int(int)__;

void map_print(int2int fn, int* A, size_t n) {
  for (size_t i = 0; i < n; i++) {
    int x = __(*fn)(A[i])__;
    printf("%d ", x);
  }
  printf("\n");
}

int main() {
  int A[6] = {0, 1, 2, 3, 4, 5};
  map_print(__&fib__, A, 6);
  map_print(__&cat__, A, 6);
  map_print(__&las__, A, 6);
  return 0;
}
ENDRUBRIC
