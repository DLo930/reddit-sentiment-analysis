\clearpage
\bgroup
%% Change values from time to time
\newcommand{\arrTP}{int}%{char}
\newcommand{\arrONE}{X}%{C}{B}{A}
\newcommand{\arrTWO}{AUX}%{TMP}{C}
\newcommand{\arrSize}{27}%{42}{10}
\newcommand{\IO}{3}%{5}{2}
\newcommand{\IA}{5}%{7}{3}{2}
\newcommand{\IB}{13}%{11}{7}{4}
\newcounter{cnt}
\newcommand{\IAA}{\setcounter{cnt}{\IA - \IO}\arabic{cnt}}
\newcommand{\IBB}{\setcounter{cnt}{\IB - \IO}\arabic{cnt}}
\newcommand{\vA}{\setcounter{cnt}{42 - \IA}\arabic{cnt}}
\newcommand{\vB}{\setcounter{cnt}{42 - (\IB - \IO)}\arabic{cnt}}

\Question[4]{Pass by Reference and Arrays versus Pointers in C}%
\TAGS{aliasing, c-array}

The following little program allocates and initializes an array of
numbers, then calls a function \lstinline'f' on two of its elements.
Rewrite these functions in the box below so that it has the same
behavior, uses the same variables at any point, but doesn't use the
pointer arithmetic notation.  You may not insert additional
instructions.

\bigskip
\begin{lstlisting}
#include <stdlib.h>
#include <stdio.h>
#include "lib/xalloc.h"
#include "lib/contracts.h"

void f([*\textbf{\arrTP}*] *x, [*\textbf{\arrTP}*] *y);   // Code omitted

[*\textbf{\arrTP}*] *mk_[*\arrTP*]_array(size_t n) {
  return xmalloc(sizeof([*\textbf{\arrTP}*]) * n);
}

int main() {
  [*\textbf{\arrTP}*] *[*\arrONE*] = mk_[*\arrTP*]_array(1[*\arrSize*]);
  for (int i = 0 ; i < [*\arrSize*] ; i++) {
    ASSERT(0 <= i);
    *([*\arrONE*] + i) = 42 - i;
  }
  [*\textbf{\arrTP}*] *[*\arrTWO*] = [*\arrONE*]+[*\IO*];
  ASSERT(*([*\arrONE*]+[*\IA*]) == [*\vA*]);
  ASSERT(*([*\arrTWO*]+[*\IB*]) == [*\vB*]);
  f([*\arrONE*]+[*\IA*], [*\arrONE*]+[*\IB*]);
  ASSERT(*([*\arrTWO*]+[*\IAA*]) == [*\vB*]);
  ASSERT(*([*\arrONE*]+[*\IBB*]) == [*\vA*]);

  printf("All tests passed.\n");
  free([*\arrONE*]);
  return 0;
}
\end{lstlisting}
\newpage
\begin{framed}
\begin{lstlisting}[belowskip=0pt]
[*\textbf{\arrTP}*] *mk_[*\arrTP*]_array(size_t n) {
\end{lstlisting}
\ifprintanswers
\begin{lstlisting}[basicstyle=\basicstyle\color{\answerColor}]
  return xmalloc(sizeof([*\arrTP*]) * n);      // UNCHANGED
\end{lstlisting}
\else\vspace{0.5in}\fi
\begin{lstlisting}[aboveskip=0pt, belowskip=0pt]
}

int main() {
\end{lstlisting}
\ifprintanswers
\begin{lstlisting}[basicstyle=\basicstyle\color{\answerColor}]
  [*\textbf{\arrTP}*] *A = mk_[*\textbf{\arrTP}*]_array(1[*\arrSize*]);
  for (int i = 0 ; i < [*\arrSize*] ; i++) {
    ASSERT(0 <= i);
    [*\arrONE*][i] = 42 - i;
  }
  [*\textbf{\arrTP}*] *[*\arrTWO*] = &[*\arrONE*][[*\IO*]];
  ASSERT([*\arrONE*][[*\IA*]] == [*\vA*]);
  ASSERT([*\arrTWO*][[*\IB*]] == [*\vB*]);
  f(&[*\arrONE*][[*\IA*]], &[*\arrONE*][[*\IB*]]);
  ASSERT([*\arrTWO*][[*\IAA*]] == [*\vB*]);
  ASSERT([*\arrONE*][[*\IBB*]] == [*\vA*]);

  printf("All tests passed.\n");
  free([*\arrONE*]);
  return 0;
\end{lstlisting}
\else\vspace{10cm}\fi
\begin{lstlisting}[aboveskip=0pt]
}
\end{lstlisting}
\end{framed}

\RUBRIC
TAGS: aliasing, c-array

Gradescope rubric:
+0.5pt Allocation remains the same
+0.5pt Array notation in for loop
+1pt Correctly assigns TMP
+1pt Array notation in assertions
+1pt Array notation when calling f

Commentary:
3/4 point each for the allocation, the assignment, all the assertions, and the call to f.

int *mk_int_array(size_t n) {
  return xmalloc(sizeof(int) * n);      // UNCHANGED
}

int main() {
  int *A = mk_int_array(110);
  for (int i = 0 ; i < 10 ; i++) {
    ASSERT(0 <= i);
    A[i] = 42 - i;
  }
  int *C = &A[2];
  ASSERT(A[3] == 3);
  ASSERT(C[5] == 7);
  f(&A[3], &A[7]);
  ASSERT(C[1] == 7);
  ASSERT(A[7] == 3);

  printf("All tests passed.\n");
  free(A);
  return 0;
}
ENDRUBRIC
\egroup