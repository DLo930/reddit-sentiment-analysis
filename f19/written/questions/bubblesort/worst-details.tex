\part[1]\TAGS{big-o}
You will now justify your last answer.  Call $f(n)$ the function you
wrote in the second blank of the previous question.  Show that $T(n)
\in O(f(n))$ using the formal definition of big-$O$.  That is, find a
$c > 0$ and $n_0 \geq 0$ such that for every $n \geq n_0$, $T(n) \leq
cf(n)$. Show your work.
\begin{framed}
\ifprintanswers{\color{\answerColor}
We are looking for a $c$ such that $n(n-1)/2 \le c n^2$.
Simplifying, $(n-1)/2 \le cn$, or $c \ge 1/2 - 1/2n$.
So, any value of $c \ge 1/2$ and any value of $n_0$ will do.

}\else~\vspace{3.9in}\fi
\end{framed}

\RUBRIC
Part (c)
TAGS: big-o

Gradescope rubric:
+ 0.5 pts  c = 1/2 or more; n = 0 or more
+ 0.5 pts  Show some kind of plausible calculation

Commentary:
. We are looking for c s.t. n(n-1)/2 <= cn^2
. Simplifying, (n-1)/2 <= cn, or c >= 1/2 - 1/2n
. So, any value of c >= 1/2 and any value of n_0 will do.
ENDRUBRIC
