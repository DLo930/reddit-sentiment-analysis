\clearpage
\Question{Another Sort}
\bgroup
\renewcommand{\v}  % To catch blind copy and paste from previous semesters
%{s}
{k}
%{c}
%{count}

Consider the following function that sorts the integers in an array, using
\lstinline'swap' and \lstinline'is_sorted' from \lstinline'arrayutil.c0'.

\begin{lstlisting}[numbers=left]
void sort(int[] A, int n)
//@requires 0 <= n && n <= \length(A);
//@ensures is_sorted(A, 0, n);
{
  for (int i = 0; i < n; i++)
  //@loop_invariant 0 <= i && i <= n;
  //@loop_invariant le_segs(A, 0, n-i, A, n-i, n);
  //@loop_invariant is_sorted(A, __________, __________);
  {
    int [*\v*] = 0;
    for (int j = 0; j < n-i-1; j++)
    //@loop_invariant 0 <= j && j <= n-i-1;
    //@loop_invariant ge_seg(A[j], A, 0, j);
    //@loop_invariant [*\textcolor{\contractColor}{\v}*] > 0 || ([*\textcolor{\contractColor}{\v}*] == 0 && is_sorted(A, 0, j));
    {
      if (A[j] > A[j+1]) {
        swap(A, j, j+1);  // function that swaps A[j] and A[j+1]
        [*\v*] = [*\v*] + 1;
      }
    }
    if ([*\v*] == 0) return;
  }
}
\end{lstlisting}


\begin{parts}
\part[0\half]\TAGS{loop-invariant, sorting}
Complete the missing loop invariant on line 8.

\begin{lstlisting}[frame=single, numbers=left, firstnumber=6, numberblanklines=false]


//@loop_invariant is_sorted(A, [*\uanswer{9em}{n-i}*], [*\uanswer{9em}{n}*]);
\end{lstlisting}

\RUBRIC
Part (a)
TAGS: loop-invariant, sorting

Gradescope rubric:
+ 0.25 pts First box (lower bound): n-i
+ 0.25 pts Second box (upper bound): n

Commentary:
. no partial credit
ENDRUBRIC

\part[1]\TAGS{complexity}
The asymptotic complexity of this function depends on the number of
comparisons made between pairs of array elements. Let $T(n)$ be the
worst-case number of such comparisons made when \lstinline'sort(A, n)'
is called. Give a \emph{closed form} expression for $T(n)$ --- a
simple, non-recursive mathematical expression that doesn't use $\sum$
or similar. Then express $T(n)$ in big-O notation in its simplest,
tightest form.

\begin{framed}
\bigskip
$T(n) \;=\; \uanswer{30em}{$n(n-1)/2$}$

\medskip\bigskip
$T(n) \;\in\; O(\uanswer{28.5em}{$n^2$})$
\end{framed}

\RUBRIC
Part (b)
TAGS: complexity

Gradescope rubric:
+ 0.5 pts First blank: T(n) = n(n-1)/2
+ 0.5 pts Second blank: T(n) in O(n^2)

Commentary:
No partial credit
. 1/2 point: T(n) = n(n-1)/2       n(n+1)/2 is wrong
. 1/2 point: T(n) in O(n^2)        O(n^3) or O(n^2 + n) is wrong
ENDRUBRIC


\input{worst-details}  % Not in S'16


\part[1]\TAGS{big-o}
Using big-$O$ notation, what is the \textbf{best} case asymptotic
complexity of this sort as a function of $n$? Under what condition
does the best case occur?

\begin{framed}
\bigskip
The best case asymptotic complexity of this sort is
$O(\uanswer{10.1em}{$n$})$.

\medskip\bigskip
This occurs when \uanswer{25.6em}{the array is sorted}.
\end{framed}

\RUBRIC
Part (d)
TAGS: big-o

Gradescope rubric:
+ 0.5 pts Best case complexity of this sort is O(n)
+ 0.5 pts Best case occurs when the array is sorted.

Commentary:
. 1/2 point: O(n)
. 1/2 point: occurs when the array is sorted, no swaps are made in
  the first pass. Anything reasonable that implies that no swaps are
  needed/array sorted is fine.
ENDRUBRIC

\end{parts}
\egroup