\usepackage{../misc/latex/edition}  % Course semester
\usepackage{../misc/latex/c0}       % Listings style for c0
\usepackage{amsmath}
\usepackage{enumerate}
\usepackage[normalem]{ulem}
\usepackage{verbatim}
\usepackage[left=1in, right=1in, top=1in, bottom=1in]{geometry}
\usepackage{graphicx}
\usepackage{hyperref}
\usepackage{tikz}     \usetikzlibrary{shapes}
\usepackage{fancybox}
\usepackage[all]{xy}
\usepackage{wrapfig}
\usepackage{fancyvrb}
\usepackage{datetime}
\usepackage{etoolbox}
\usepackage{calc}
\usepackage[nomessages]{fp}
\usepackage{import}  % Like input and include, but respects subdirectories

\newcommand{\defaultQuestionLocation}{questions}
\newcommand{\inputQuestion}[2][\defaultQuestionLocation/]{%
  \subimport{#1}{#2}
}
% Subdirectories of \defaultQuestionLocation containing code and pictures
\newcommand{\code}{code}
\newcommand{\img}{img}


%%% ic: frontmatter macros
\newcommand{\specialInstructions}{}
\newcommand{\HWNUMBER}
{\ifdefempty{\hwnumber}{__}{%
  \ifnumless{\hwnumber}{10}{0\hwnumber}{\hwnumber}}}
\newcommand{\hwtype}{Written Homework}

%%% ic: 'exam' tweaks
\renewcommand{\half}{.5} % Half points

\newcommand{\Question}[2][]
 {\ifstrempty{#1}
    {\question{\bf #2}}
    {\question[#1]{\bf #2}}
  \immediate\write\rubricfile{}%
  \immediate\write\rubricfile{Question \thequestiontitle:}%
  \immediate\write\rubricfile{==========}
 }

%%% ic: Support for editable PDF
% counter name (some viewers misbehave if always the same)
\newcounter{editable}
\newcommand{\nextField}{\addtocounter{editable}{1}q\arabic{editable}}
\newcommand{\NextField}
 {\makebox[0pt][r]{\scalebox{0.1}{\color{White}\nextField}}}

% Color of edit area
\newcommand{\editAreaColor}{red}
% Single line answer:   \editableLine[extra parameters (optional)]{line width}
\newcommand{\editableLine}[2][]
{\textcolor{\editAreaColor}{%
 \underline{\hspace*{-0.25em}%
 \raisebox{-0.5ex}{%
 \TextField[width=#2, borderwidth=0, #1]{\NextField}}}}%
}
% Single line answer for code:  \editableLine[extra parameters (optional)]{line width}
\newcommand{\editableCodeLine}[2][]
{\textcolor{\editAreaColor}{%
 \underline{%
 \TextField[width=#2, height=1.5ex, borderwidth=0, #1]{\NextField}}}}
% Multiline answer:  \editableLine[extra parameters (optional)]{box height}
\newcommand{\editableBox}[2][]
{\leavevmode\hspace*{-0.1em}%
\TextField[height=#2, width=\linewidth,
           multiline=true, borderwidth=0.1, bordercolor=\editAreaColor,
           #1]{\NextField}}

%%%%% Same answer format as exams
\renewcommand{\rmdefault}{ppl}
\renewcommand{\sfdefault}{phv}
\newcommand{\answerColor}{Blue}

\ifprintanswers
\newcommand{\answer}[2]{\makebox[#1][c]{\color{\answerColor}#2}}
\else
\newcommand{\answer}[2]{\makebox[#1][c]{}\makebox[0pt]{\phantom{|}}}
\fi
\newcommand{\uanswer}[2]{\underline{\answer{#1}{#2}}}


%%% Write rubric snippet.  Usage:
% \RUBRIC
% any multi-line text (including \, #, %, whatever)
% ENDRUBRIC
%% (ENDRUBRIC should be on a line by itself)
\makeatletter
\def\RUBRIC
 {%
  \begingroup
  \let\do\@makeother\dospecials
  \endlinechar=`\^^J
  \@tofile%
 }
\def\ENDRUBRIC{ENDRUBRIC}
\def\@tofile#1^^J{%
  \def\@test{#1}%
  \ifx\@test\ENDRUBRIC
    \immediate\write\rubricfile{}  % End with an empty line
    \expandafter\@firstoftwo
  \else
    \expandafter\@secondoftwo
  \fi
  {\endgroup}%
  {\toks@{#1}%
   \begingroup\endlinechar=\m@ne
   \everyeof{\noexpand}%
   \xdef\@temp{\scantokens\expandafter{\the\toks@}}%
   \endgroup
   \immediate\write\rubricfile{\@temp}%
   \@tofile}%
}
\makeatother

%% Displays tags for an exercise in 'answer' mode
\newcommand{\TAGS}[1]
{\ifprintanswers%
  \rule{0em}{0ex}%
  \marginpar{\footnotesize%
    \fcolorbox{black}{Gray!25}{%
      \parbox[t]{2cm}{\raggedright\textbf{TAGS:}\\#1}}}%
  \ignorespaces%
 \fi}%


%% Page layout
\pagestyle{headandfoot}

\headrule
\header{\textbf{\courseNumber{} \hwtype{} \hwnumber}}
       {}
       {\textbf{Page \thepage\ of \numpages}}
\footrule
\footer{}{}{\COPYRIGHT}

\renewcommand{\partlabel}{\textbf{\thequestion.\thepartno}}
%\renewcommand{\partlabel}{\textbf{Task \thepartno}}
\renewcommand{\subpartlabel}{\textbf{\thesubpart.}}
\renewcommand{\thepartno}{\arabic{partno}}
\renewcommand{\thesubpart}{\alph{subpart}}
\pointpoints{pt}{pts}
\pointformat{\raisebox{0ex}[\height][0pt]{\fcolorbox{black}{yellow}{\themarginpoints}}}
\bonuspointformat{\raisebox{0ex}[\height][0pt]{\fcolorbox{black}{red}{\themarginpoints}}}
\marginpointname{\points}
\pointsinmargin
%\boxedpoints

\setlength\answerlinelength{2in}
\setlength\answerskip{0.3in}

\newcommand{\mkWrittenTitle}[1]{#1}
\newcommand{\mkDueDate}[1]{#1}
\newcommand{\mkEvalSummary}[1]{#1}
\newcommand{\mkGradetable}[1]{#1}



% This fixes an issue with the exam package version 2.6 and after,
% where 'framed' has been renamed to 'examframed' to avoid a conflict.
\ifcsmacro{examframed}{%
\newenvironment{framed}
{\begin{examframed}}
{\end{examframed}}
}{}
